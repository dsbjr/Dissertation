\chapter{Electrolyte Gating}

This chapter provides an introduction to electrolyte gating. We discuss the mechanism underlying gating methods, the advantages and disadvantages of gating with an electrolyte, how to choose an electrolyte, and how to design a device for gating. We conclude with a discussion of some important gating measurements.

\section{Mechanism of field effect gating}

Condensed matter physicists and electrical engineers often add charge carriers (either electrons or holes) to a material with the hope of changing its properties - increasing or decreasing conductivity, changing an insulator into a metal, or even changing a material's physical structure. The most common and least intrusive method to add carriers uses the field effect to electrostatically gate a material. The field effect adds carriers using an electric field rather than chemical substitution or intercalation. 

We can define an electrochemical potential $\mu$ that represents the amount of energy required to add an electron to a system. Using this definition

\begin{equation}
\mu = E_{F} + qV
\end{equation}

where $E_{F}$ is the Fermi energy\footnote{This is only correct at T = 0 K. At finite temperatures, $E_{F}$ should be replaced with an electrochemical potential at zero field, which reflects the broadening of the occupation function. However, for most materials, the Fermi temperature is so high that the difference is negligible.}, $q$ is the fundamental charge, and $V$ the applied electric potential. When two systems with different electrochemical potentials are brought into contact, heat and charge will flow across the interface until the electrochemical potentials are equal. We can exploit this effect to add carriers to a material. If we apply a field to a material that decreases $\mu$, then electrically connect that material to a reservoir (i.e., electrical ground), charge carriers will flow into the material until the chemical potentials are equal. We have used the field effect to increase the carrier density in our material. What is probably the most produced device in the world, the metal-oxide-semiconductor field effect transistor (MOSFET), uses a metal ``gate'' and an insulating oxide to introduce carriers in a semiconductor, controllably changing the conductivity of the semiconductor channel in the device.

\begin{centering}
\includegraphics[width=0.5\textwidth]{C:/Users/dsbjr/Documents/GitHub/Dissertation/img/MOSFET-original.png}
  \captionsetup{width=0.75\textwidth}
  \captionof{figure}[MOSFET Diagram]{Cartoon of a metal-oxide-semiconductor field effect transistor (MOSFET). Voltage $V_{g}$ applied to the metal gate electrode creates a field which polarizes the oxide dielectric and induces charge carriers in the channel. When the channel is conducting, $V_{sd}$ applied between the source and drain electrodes drives a large current.} 
  \label{fig:ElecGate-1}
\end{centering}

We can calculate the charge added to the channel by modeling the MOSFET as a parallel plate capacitor

\begin{equation}
Q = CV_{g} = \epsilon \frac{A}{d} V_{g}
\end{equation}

Where $A$ is the area of the electrode and $d$ is the thickness of the oxide, we can divide by area to find the induced charge density $n$

\begin{equation}
n = \frac{Q}{A} = \frac{\epsilon V_{g}}{d}
\end{equation}

The strongest SiO\textsubscript{2} can withstand a field strength of approximately 1 V/nm before breakdown\footnote{Breakdown occurs when being subjected to an electric field causes a dielectric to lose its insulating properties. Breakdown is caused by a combination of how long the dielectric is subjected to the field and how long the dielectric is subjected to the field.} \cite{Palumbo2019}. Using our parallel plate capacitor model, the maximum amount of charge that can be induced this way is approximately 2 x $10^{12}$ \percmsq . For specially grown high-K dielectrics, this maximum induced charge increases to about 5 x $10^{13}$ \percmsq \cite{Robertson2004}.

While these density changes are impressive, there are use cases for field effect gating that require greater changes in density, including reaching the van Hove singularity in graphene or removing all the carriers from a metal thin film. To achieve larger changes in density, we can appeal to the technique of electrolyte gating.

\begin{centering}
\includegraphics[width=0.5\textwidth]{C:/Users/dsbjr/Documents/GitHub/Dissertation/img/ElectrolyteGating-original.png}
  \captionsetup{width=0.75\textwidth}
  \captionof{figure}[Mechanism of Electrolyte Gating]{Electrolyte gating uses an electrically polarized molten salt to accumulate ions at the surface of a material of interest. This layer of ions creates an electic double layer capacitor, inducing charge into the material.} 
  \label{fig:ElecGate-2}
\end{centering}

Electrolyte gating (\ref{fig:ElecGate-2}) replaces an oxide dielectric with an electrolyte (a molten salt) consisting only of positively and negatively charged species. When a voltage is applied to the gate electrode, the applied electric field causes the charged species to migrate: negative charges accumulate at the gate and positive charges accumulate at the channel (or vice-versa, depending on the sign of the gate voltage). This polarization of charge forms an electrical double layer capacitor at the interface of the channel. To get a sense of the possible induced charge, we can use our previous equation for density, replacing $d$ by the ionic radius of the species (typically ~1 nm), $\epsilon$ by the appropriate dielectric constant, and $V_{g}$ by the maximum voltage that can be applied to the liquid. In this case we find an induced density of 5 x $10^{14}$ \percmsq , an order of magnitude higher than that possible with the best oxide dielectrics.

Changing carrier density by an additional order of magnitude beyond oxide gating is a powerful tool, but electrolyte gating also has strengths and weaknesses that must be considered when designing and executing an experiment. These are discussed in the following section.

\section{Considerations for electrolyte gating}

A successful electrolyte gating experiment involves choosing the most appropriate electrolyte, device geometry, and experimental procedure. We'll examine each in turn.

\subsection{Choosing the right electrolyte}

Electrolytes are typically composed of monovalent, complex molecules. A wide variety of cations and anions are available for use. A selection of these are presented in Figure \ref{fig:ElecGate-3}.

\begin{centering}
\includegraphics[width=0.75\textwidth]{C:/Users/dsbjr/Documents/GitHub/Dissertation/img/IonicLiquids-Petach.png}
  \captionsetup{width=0.75\textwidth}
  \captionof{figure}[Structure of Electrolyte Cations]{Selected cations and anions used in anhydrous electrolytes. a. Diethylmethyl(2-methoxyethyl)ammonium, DEME\textsuperscript{+} b. 1-butyl-methylpyrrolidinium, BMPY\textsuperscript{+} c. 1-ethyl-3-methyl-imidazolium, EMI\textsuperscript{+} d. Tetrafluoroborate, BF$_{4}^{-}$ e. Tris(pentafluoroethyl)trifluorophosphate, FAP\textsuperscript{-} f. bis(trisfluoromethylsulfonyl)imide, TFSI\textsuperscript{-}. From \cite{Petach2017}.} 
  \label{fig:ElecGate-3}
\end{centering}

The most important concern when choosing an electrolyte is chemistry. Unlike metal-oxide gating, which uses a chemically inert oxide, electrolyte gating involves intimate contact between the material of interest and the (potentially) chemically active electrolyte. Chemical reactions between the material and the electrolyte will change the nature of the material and disturb the double-layer capacitance, making the gating ineffective at best and destructive at worst. 

Applying a potential across the electrolyte creates the possiblity of not just chemical, but electrochemical (i.e., reduction-oxidation) reactions that can have similar effects to those listed above. A method to characterize the electrochemistry of an electrolyte is to measure its electrochemical stability window - the range of voltages beyond which the ions in the electrolyte will begin to react at the electrodes. The electrochemical stability sets the maximum potential that can be applied across the electrolyte, and therefore the maximum induced carrier density.

The electrochemical stability window is typically measured with cyclic voltammetry - sweeping the potential across the electrolyte at a constant rate and measuring the current. This current should be constant, and deviations suggest chemistry is occurring. A cyclic voltammetry measurement for DEME-TFSI is presented in Figure \ref{fig:ElecGate-4}. As seen in the figure, the behavior of the gate current becomes substantially nonlinear around $\pm 2$ volts. Accordingly, I have assumed the electrochemical stability window for this electrolyte and  measurement to be -2 to +2 volts. The observed electrochemical stability window for DEME-BF4 is similar.

\begin{centering}
\includegraphics[width=0.5\textwidth]{C:/Users/dsbjr/Documents/GitHub/Dissertation/img/DEME-TFSIStabilityWindow-Original.jpg}
  \captionsetup{width=0.75\textwidth}
  \captionof{figure}[Electrochemical stability window for DEME-TFSI]{Cyclic voltammetry measurement of DEME-TFSI (293 K, 1 x $10^{-6}$ Torr). The sweep rate is 13 mV/second. Current density is defined at the working electrode.} 
  \label{fig:ElecGate-4}
\end{centering}

Another concern for electrolyte gating is contamination of the electrolyte. Electrolytes are polar and hygroscopic; therefore, they will collect moisture and other contaminates from the atmosphere. When solvated in the ionic, these contaminants can create undesirable electrochemical interactions. Fortunately, the high vapor pressure of electrolytes means they can withstand vacuum, so some water can be removed by baking at elevated temperature in high vacuum. However, water remains the biggest source of contamination and exposure to air will necessarily reduce the electrochemical stability window. When possible, in my measurements I have attempted to reduce the exposure of the electrolyte to atmosphere.

Finally, electrolytes freeze below room temperature, which offers some challenges and advantages. Ions in an electrolyte must be free to move for the electrolyte to be polarized, so the electric field must be applied above the freezing temperature. The lower the freezing temperature, the lower the temperature at which the field can be applied. Because electrochemical reactions are suppressed at low temperatures, applying the field just above the freezing temperature minimizes There are often changes in the electric double layer near the freezing temperature, so it is important to avoid electrolytes with transition temperatures in a range of interest for the gated material.

\subsection{Thoughtful device design}




\section{Notable Electrolyte Gating Measurements}

Ambipolar gating of WS2. MoS2 metal-insulator transition. Trevor Petach's measurements.