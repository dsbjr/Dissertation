\begin{appendices}

\chapter{Procedures}

I used the following procedures during my time at Stanford. With minor adjustments, they can be used to make transport devices for just about any exfoliated two-dimensional material.

\section{Producing samples}

\subsubsection{Cleaving wafers for exfoliation}
\begin{enumerate}
	\item Find a new 90 nm SiO2 wafer and score it using a scribe along one of the crystal axes.
	\item Prop the scored portion of the wafer on a broken wirebonder tip.
	\item Press on either side of the score using a closed pair of tweezers such that the wafer cleaves along the crystal axis to which the score is parallel.
	\item Repeat until you have several 10 mm x 10 mm chips.
	\item Remove dust from chips using the air gun.
\end{enumerate}

\subsubsection{Exfoliation of graphene}
\begin{enumerate}
	\item Take a 6” piece of high gloss scotch tape (comes in a red box) and fold the ends over to use as handles. Place sticky side-up on the lab bench.
	\item Place the graphite crystal exfoliation-side down near the edge of the sticky side of the tape. Orient any tears or imperfections on the graphite crystal perpendicular to the long direction of the tape. Press GENTLY with closed tweezers to make good contact between the tape and the crystal.
	\item Using the handle, raise the tape and adhered graphite until it’s approximately perpendicular to the lab bench. GENTLY press the edge of the adhered graphite crystal closest to the handle to ensure the edge is well adhered, then use the handle and your tweezers to peel the graphite crystal from the tape. The crystal should leave behind a shiny patch of graphene on the tape. Return the crystal to its case.
	\item Using the handles, fold the tape to copy the exfoliated patch of graphene onto another section of tape.
	\item Make copies from the copied section from step 4 onto clean areas of the tape until the copy from step 4 is slightly hazy. Return the tape to the lab bench sticky side-up.
	\item Place a clean 90 nm SiO2 chip onto a glass slide. Select a hazy region of the copy from step 4 and place this region face down onto the chip. Avoid creating bubbles. Apply pressure with your thumb. Trim excess tape from the glass slide.
	\item Heat the glass slide/chip/tape complex on hot plate at 100C for 2 minutes. Then cool to room temperature (using an air gun or just by setting on the lab bench).
	\item Peel the tape from the glass slide and the chip at a high angle (as close to anti-parallel) as possible. Immobilized the chip using tweezers. Make sure to peel slowly (should take about 60 seconds).
	\item Store exfoliated flakes in dry box.
\end{enumerate}

\subsubsection{Exfoliation of boron nitride}
\begin{enumerate}	
	\item Take a 6” piece of high gloss scotch tape (comes in a red box) and fold the ends over to use as handles. Place sticky side-up on the lab bench.
	\item Sprinkle a small number of hBN crystals onto the tape.
	\item Repeatedly exfoliate the crystals onto the tape until the tape has a uniform covering of sparkly crystals. The tape should look like it’s coated with rainbow glitter (or makeup).
	\item Place several clean 90 nm SiO2 chips onto a glass slide. Press the tape over the chips, applying pressure and avoiding bubbles. The relative position of the tape and the chips is not important for this step. Trim the excess tape.
	\item Heat the glass slide/chip/tape complex on hot plate at 100C for 2 minutes. Then cool to room temperature (using an air gun or just by setting on the lab bench).
	\item Peel the tape from the glass slide and the chip at a high angle (as close to anti-parallel) as possible. Immobilize the chip using tweezers. Make sure to peel slowly (should take about 60 seconds).
	\item Store exfoliated flakes in dry box.
\end{enumerate}

\subsubsection{Exfoliation of ruthenium chloride}
\begin{itemize}
	\item Follow the same instructions as for boron nitride.
\end{itemize}

\subsubsection{Finding and naming flakes}
\begin{enumerate}
	\item Orient the chips so that a cleaved corner is in the bottom right-hand side of the chip.
	\item Look for flakes under the microscope. I’ve found that looking at 10x and changing the F-stop to the smallest setting helps the flakes show up.
	\item Save a 10x and 50x high contrast image of boron nitride flakes using the following naming convention: ``dsbjr-bn-DATE OF FIRST EXFOLIATION IN THIS SERIES-CHIP NUMBER FLAKE LETTER-X LOCATION Y LOCATION-MAGNIFICATION''. For example, the 10x image of the second hBN flake on chip 3 in the exfoliation series begun on 1 April 2016 located two millimeters from the left edge of the chip and one millimeter from the top of the chip would have the name ``dsbjr-bn-2016-04-01-3b-xp2ym1-10x.''
	\item Save a 10x and 50x high contrast image of graphene flakes using the following naming convention: ``dsbjr-gph-DATE OF FIRST EXFOLIATION IN THIS SERIES-NUMBER OF LAYERS (mono/bi/tri)-CHIP NUMBER FLAKE LETTER-X LOCATION Y LOCATION-MAGNIFICATION''. For example, the 50x image of the third flake that is a bilayer on chip 9 in the exfoliation series begun on 1 April 2016 located three millimeters from the right edge of the chip and two millimeters from the bottom of the chip would have the name ``dsbjr-gph-2016-04-01-bi-9c-xm3yp2-50x.''
\end{enumerate}

\subsubsection{Making PPC/GEL-PAK stamps}
\begin{enumerate}
	\item If not available, mix up some PPC solution in anisole. Dissolve Aldrich 389021-25G poly(propylene carbonate) average mn 50k at 11 wt% in anisole. Use a stirrer bar to mix on a hotplate at room temperature. Note that this dissolution is slow and may require over 24 hours.
	\item Spin polypropylene carbonate (11 wt% in anisole) onto 5 mm x 5 mm diced chips at 1500 rpm for 60 seconds. Bake the chips at 80C for 5 minutes or until the edge beads dry.
	\item Clean a glass slide using acetone and isopropyl alcohol.
	\item Cut a small section of gel-pak material and place it onto the cleaned slide near the upper-left corner.
	\item Examine the gel-pak material under the microscope and, using a razor, cut away any areas that are poorly adhered to the glass slide or that have some surface contamination. It’s okay if the remaining gel-pak is small (in fact, a small stamp is probably best). Cover the gel-pak bearing slides to protect from dust.
	\item Warm up the UV-ozone lamp for five minutes.
	\item UV-ozone the gel-pak slide for five minutes. During this time, use a razor to lift the edges of the PPC from a chip.
	\item Remove the gel-pak slide from the UV-ozone. Using tweezers, peel the edge beads of the PPC from the chip and gently lay the clear window of PPC over the gel-pak stamp.
	\item Bake the stamp at 80C for five minutes. Keep covered.
	\item Remove and examine under the microscope. Discard any undesirable stamps.
\end{enumerate}

\subsubsection{Preparing device substrates}
\begin{enumerate}
	\item Use 300 nm SiO2 chips with distance markings.
	\item Sequentially rinse the chips in acetone, isopropyl alcohol, methanol, and deionized water, blowing dry between each solvent.
	\item Anneal the chips in open air at 500C for one hour.
\end{enumerate}

\subsubsection{Stacking graphene and boron nitride}
\begin{enumerate}
	\item Find and select all exfoliated flakes. Compare their sizes to be sure the device will behave as expected (use transparent layers in gimp to select orientations and stacking order).
	\item Place the hBN-bearing chip on the transfer station and heat to 35C. Navigate to the desired flake.
	\item Mount the desired stamp at a slight angle (maybe 1-2 degrees).
	\item Using the coarse and fine focus knobs, roll the stamp into contact with the hBN flake. Roll on near a corner of the stamp, but not at the exact corner. Minimizing the contact area between the stamp and the chip helps avoid delamination.
	\item Peel the stamp from the chip. Note that some group members have experienced delamination when peeling. If your stamp appears to be stuck, you can try to snatch it off by turning the coarse focus to pull it up quickly. Higher temperatures make the stamp less sticky, but also make delamination more likely. Use your discretion and judgement here.
	\item Unmount the stamp and heat on a hotplate for 1 minute at 80C. Take a photo of the flake on the stamp. Label appropriately and save in the dropbox folder.
	\item Repeat the above steps for the subsequent graphene, hBN, and optional graphite backgate. Take photos at each stage.
	\item Place the substrate chip (300 nm oxide) on the transfer station and heat to 80C. SLOWLY roll the assembled stack on the stamp onto the substrate. Remove and discard the stamp.
	\item Anneal the stack/substrate in 10:1 oxygen:argon at 500C for 1 hour.
	\item The direction you roll the stamp onto the flake can be important. I think that sides of the flake with greater edge length can be easier to start the pick-up with.
	\item Graphene flakes with long/thin tails can be a problem. Sometimes during the pick-up these tails can twist and fold on themselves, ruining your perfect device. Try to roll the stamp onto the flake such that these tails come up first – wider graphene areas or multiple layer graphene/thin graphite is more rigid and not subject to the same problem.
	\item To reduce the chance of delamination, I try to minimize the amount of time the stamp is in contact with the chip during peel-off. What I end up doing is rolling on slowly, then rolling off slowly, only to the point that flake has been removed. After the flake is removed but there’s still stamp in contact with the chip, I snatch that part off like mentioned above. Any wrinkles introduced by the snatch typically come out during the anneal at 80C.
	\item If you’re delaminating, the best thing to do is put the stamp back in contact with the surface and try to go slowly. That said, I haven’t really been successful rescuing a stack when the stamp delaminates. It’s in your best interest not to delaminate in the first place.
\end{enumerate}

\subsubsection{Making open graphene samples}	
It’s hard (impossible?) to pick up monolayer graphene with a PPC stamp. So if you want graphene on the top of a device (for scanning probe measurements), you’ll need to flip over your stack. Do this by depositing the hBN-graphene stack onto another film, and make a new stamp using that film to deposit the device on a substrate.
\begin{enumerate}
	\item On a clean Si substrate (5mm x 5mm x 500 um), spin 11 wt% PPC in anisole at 1500 rpm for 1 min, then bake at 80C for a few minutes until the edge beads are dry (they’ll look wrinkly).
	\item One the same substrate, spin 950/A2 PMMA at 4000 rpm for 1 min, then bake at 80C for a few minutes until the film is dry.
	\item Pick up hBN-graphene as described previously.
	\item Place the PMMA on PPC substrate on the transfer station and heat to 60C.
	\item Gently deposit the hBN-graphene stack onto the PMMA on PPC film.
	\item Remove the PMMA on PPC film substrate and scratch along the edges with a razor to lift the edge of the film on all four sides.
	\item Cut a small square of the blue tape slightly larger than the 5 mm x 5 mm substrate. Fold it half in both directions and cut out the center to form a window larger than the flat area of the stamp.
	\item Press the square of tape onto the PMMA on PPC film substrate, making sure the hole in the tape is centered above the flat area of the film and that the tape adheres to the scratched film along the edges.
	\item Peel the tape and the PMMA on PPC film off the substrate. Take care not to wrinkle or crease the film, as it now hosts your precious heterostructure.
	\item Prepare a stamp though the UV ozone step, then lay the film onto the stamp and back at 80C for a few minutes.
	\item Verify the PMMA on PPC film is adhered to the stamp using an optical microscope, and that the heterostructure isn’t caught in a wrinkle or crease.
	\item Place the final substrate on the transfer stage and heat to 160C.
	\item Bring the PMMA on PPC stamp into contact with the substrate for deposition. The higher temperature is necessary to get the PMMA film to adhere.
	\item Once the heterostructure is deposited and the PMMA film adhered, rinse in acetone to remove the PPC, and then anneal in Ar/0¬2 to clean.
\end{enumerate}

\section{Designing devices}

\subsubsection{Atomic force microscopy}
\begin{enumerate}
	\item Take detailed AFMs of the stack. Make sure to take an AFM of the intended active area and the edges to measure dielectric (hbN flake) thickness. Record both topography and phase images. High contrast in phase images indicates that features are on the surface rather than buried in the heterostructure (e.g., leftover ppc residue).
	\item Flatten and correct the images using gwyddion. I typically remove horizontal scars and then use the three point leveling function with points of several pixel radius as close to the graphene as I can. I try to avoid the median height leveling function as it introduces artifacts into the image. You can use the line leveling function if you want to remove whole lines that look out of place. Then change the scale such that there’s contrast where the graphene is.
\end{enumerate}

\subsubsection{Device Design}
Overall Summary: Decide where to put the device based on optical images and AFM. Define an inner region where the sensitive writes will go, and an outer region where we’ll connect the traces from the inner region to bond pads. Drop alignment marks in the inner region to help with the alignment of sensitive writes. Then define a topgate (if desired), an etch, and patterns for ohmic contacts.

\begin{enumerate}
	\item Take images at 10x, 20x, 50x, 100x, and 150x (you only need the 10x and 50x, but the remainder are good to have). Make sure the 50x includes four alignment marks that circumscribe the heterostructure.
	\item Half the size of the images (2048 to 1024), as it’s hard for design cad to handle large images.
	\item Open the 50x and the processed/leveled AFM image as layers in gimp. Scale/translate/rotate the AFM image until it matches the features of the 50x image. Bubbles that remain in the heterostructure are good markers to use to make sure everything is lined up properly.
	\item Flatten the image and export it.
	\item In design cad, open a new file and load the 10x and 50x+AFM images. Scale them as needed (typically the 10x is scaled by 6 and the 50x by 1.2). Align the images as necessary. Note that 1 unit in design cad is 1 um.
	\item In each of layers 1 through 4, place polygons defining the alignment marks on the 50x image. There should be a 15 um x 15 um dashed box around each of the marks, and then a solid polygon that matches the outline of the alignment mark inside this box.
	\item In layer 6, define your beam dump. I typically draw another box over one of the alignment marks.
	\item Select a region of the heterostructure to use for your device. It should be as free of defects as possible (bubbles, specks, etc).
	\item In approximately a 40-50 um x 40-50 um region around the active area, in layer 6, define four new alignment marks (5 um x 500 nm crosses) in the corners.  You may want four more if you’re doing a topgate. Draw a box around this region as a guide to the eye so you can know where to put polygons for the inner write.
	\item Define a topgate (if desired): In layer 7, use the rectangle tool to define the active region. Rotate as necessary and record the angle. Then add a lead ending in an L-shape that goes to the edge of the inner write region.
	\item Define an etch pattern. In layer 8:
	\begin{enumerate}
		\item Define your etch pattern by making two smaller rectangles (or more if you’re ambitious), rotating them to match the topgate, and positioning them as voltage contacts. Then, define another rectangle at 90 degrees relative to the voltage contacts for the current contacts. All these rectangles should extend beyond the bounds of the topgate. Note that the current lead rectangle should fit entirely within the topgate rectangle. Add all these rectangles together using the Boolean add function to get one polygon.
		\item Given the way PMMA works as an e-beam resist, we don’t actually want to write this polygon. We want to write the inverse of it. To do that, we’ll need to define two other polygons, each of which forms a half clamshell around the active area of the device and cuts through the center of the polygon defined in 11a. Make one of these clamshells, then subtract from it the polygon defined in 11a, leaving a clamshell with one edge that’s half of the polygon. Move the copied polygon such that it lines up with the edges in the first clamshell. Then draw another clamshell and subtract the new polygon. The two clamshell polygons form your etch pattern.
		\item Increase the perimeter of what will eventually be the graphene you make ohmic contact to. 
		\item Make sure you remove all the graphene from your device to avoid shorts. If necessary, in layer 9, draw polygons that will cover the rest of the graphene.
	\end{enumerate}
	\item Define an ohmic pattern.
	\begin{enumerate}
		\item In layer 10, draw quadrilaterals that will be where you evaporate gold to make initial contact to the graphene (and graphite backgate if present).
		\item In layer 11, draw leads connecting these quadrilaterals to the ends L-shaped ends that approach the edge of your inner write region. Make sure these are separate polygons.
	\end{enumerate}
	\item Define bond pads.
	\begin{enumerate}
		\item In layer 13, define bond pads around the edges of the device. I typically fit these on the 10x image and make them 100 um x 200 um, with 100 um between them.
		\item In layer 12, define traces that overlap the bond pads and the L-shaped leads in the inner region.
	\end{enumerate}
\end{enumerate}

\subsubsection{Design principles}
\begin{enumerate}
	\item Avoid bumps or specks in the AFM image. If you must, it’s probably okay to put an impurity in the center of a hall bar because we care mostly about the edge states. But make sure there’s nothing on your edges.
	\item Make sure there is at least 1 um of graphene perimeter underneath your ohmic contacts. 
	\item The etch is design to eat hBN, not graphene or graphite. So if you have a thick bit of graphene that you’re trying to etch, all you’ll end up doing is exposing it. The exposed graphite can make things awkward when you define your bond pads as it will short them to each other. Be careful not to short your leads to each other.
	\item Expect the windows for the alignment marks to be covered in gold when you’re doing your ohmic deposition. Make sure there’s enough area between them and the closest lead that two leads won’t get shorted together. In fact, sometimes it’s best to put your ohmic leads through/over an alignment mark windows to avoid shorting.
	\item Hall bar width should be at least 1 um. Shorter than that and you start to run into edge effects. The minimum length is probably about 4 um.
	\item The best resolution you’ll get with the NovaNano is probably around 200 nm. Don’t try to define any features smaller than that, and make sure that polygons in successive layers won’t short if you’re shifted by something of that order.
	\item Use every square nanometer of space on your active area that you can.
	\item Avoid sending any traces over bumps or obvious imperfections.
\end{enumerate}

\section{Fabrication}

\subsubsection{E-beam lithography tips}
\begin{enumerate}
	\item Use a beam dump (a random polygon written somewhere in the write field you don’t care about). Write this as the first polygon in any pattern. I think there might be some random motion that happens when the beam first starts. But I’m not sure – this could be superstition.
	\item Write from the smallest features in a pattern to the largest. For example, in the etch design laid out above, you would want to write the beam dump first, then layer 8, then layer 9, if necessary. For the ohmic contacts/bond pads (typically done in one write), you’d want to write the beam dump first, then the small quadrilaterals (layer 10), then the inner traces (layer 11), then the outer traces (layer 12), and finally the bond pads (layer 13) at the end. Each successive layer is less vulnerable to lateral shifts, so writing them in that order helps make sure things are connected at the very end.
	\item Write at as high a magnification as possible. I’m not sure if this is superstition or it just makes me feel better, but when I added alignment marks and worked at around 1000x – 2000x for my etch and inner ohmics I got better results.
	\item Any time you change the beam (either spot size or magnification), redo an alignment step. This can help take care of transient beam changes that emerge during the write.
	\item For small features (small etch pattern and small ohmics), I use center to center distance and line spacing at 20 nm / 20 nm. This spacing is the most accurate. For larger features (i.e., outer traces and bond pads), I use 100 nm / 100 nm. Otherwise your writes take forever.
	\item Write small features at spot size 3, and larger features at whatever spot size has the highest current. Not sure if this is just superstition, but it’s what I was taught.
	\item Don’t measure beam current on an alignment mark (can be up to 50% lower than you’ll get when writing on your device). I measure beam current in a Faraday cup on the SEM mount. You may read a different specimen current during writing, but it will always be less than what you’ve measured in the cup. I haven’t had any problems measuring beam current that way.
	\item Write each layer at about 50x less than the maximum magnification NPGS allows. By increasing the write field (lowering the magnification), you can avoid errors NPGS throws about polygons being outside the write area.
	\item Check the relative displacement of your markers during the alignment step. For larger patterns, I would expect some kind of self-consistent translation and rotation, rather than some arbitrary displacement. For small patterns, at higher magnification there may be some arbitrary displacements. I’m not sure why, but I suspect that maybe because the alignment in the first step is necessarily bad (the alignment marks are nebulously defined and hard to align to), maybe there’s some shift in where the alignment marks are dropped.
	\item Do a test write each time you go to do a write. This can show you your errors. Pay attention to how fast polygons are written, and which polygons actually show up in the write field. I would expect the alignment marks and topgate to be written in less than 1 second, the etch to be written in up to 10 seconds depending on how big your second etch (layer 9) is, and the bond pads and ohmics to take up to 15 minutes. Something less than this may indicate you have your beam currents set incorrectly.
	\item Make sure the beam is not exposing your sample when you drive to the origin of your pattern (which should be in the center of the pattern to make sure you’re writing at the highest magnification). Watch the specimen current to confirm – it should be less than 10 pA if you aren’t writing. You can do this by switching the beam blanker to EXT when you’re staring at alignment mark with the detector unpaused. The screen should go dark and return only when you switch the beam blanker back to ON. Navigate to the center of your pattern with the beam blanker on EXT. Bear in mind that if you have the beam paused and set the scan mode to external in FEI without the beam blanker on EXT (and sometimes even if the beam blanker is on EXT!), you can expose right in the center of your pattern. Don’t do this – you’ll cross link stuff and be pissed off later.
	\item Common NPGS errors:
	\begin{enumerate}
		\item PG(6) – caused by intersecting polygons. You can fix this in-situ by opening design cad lite, drawing over your offending polygon, and deleting your old one.
		\item PG(-1) – caused by the calculated alignment transformation throwing polygons outside your write field. Fix by decreasing your magnification to increase your write field size.
		\item If you put your alignment polygons and pattern polygons in the same .dc2 file, make sure you have four alignment marks. Even if the marks aren’t in the correct shape or in the correct place, NPGS will interpret the first four layers as alignment marks. If you need fewer than four, make a separate alignment .dc2 file.
		\item Invalid transformation matrix – caused by the alignment transformation throwing your alignment mark window outside the write field. Fix by reducing your magnification.
	\end{enumerate}
\end{enumerate}

\subsubsection{Alignment marks}
\begin{enumerate}
	\item Bake the heterostructure for 2 minutes at 180C to remove any water from the surface (may be superstition).
	\item Spin 950/A5 PMMA on the stack at 4K RPM.
	\item Bake at 180C for 20 minutes.
	\item Condition the column.(let it run at highest current at 30 kV). This may also be superstition, but I think charging the column can rearrange whatever dust or junk is on there and minimize the drift during actual writing.
	\item Set beam parameters to spot size 3, 10 kV.
	\item Using the large alignment marks (100 um separation) as alignment markers, write the beam dump and the alignment marks for the inner write field.
	\item Develop by dunking in 3:1 Water:Isopropyl Alcohol for 1 minute and blowing dry. Do this as gently as possible. Examine under microscope to make sure you have what you’re expecting.
	\item UV ozone for 90 seconds or so before putting in the evaporator (may be superstition).
	\item Evaporate 5 nm Ti/some amount of gold in the KJL or the Moore e-beam evaporator. I don’t think you have to go that thick with these. Probably 50 nm of gold is okay – you just need to be able to see what you put down underneath a film of PMMA in the SEM.
	\item Liftoff by submerging in acetone. This can be as short as 15 minutes or it can take up to an hour, but it shouldn’t take much longer than that.
\end{enumerate}

\subsubsection{Topgate}
\begin{enumerate}
	\item Bake the heterostructure for 2 minutes at 180C to remove any water from the surface (may be superstition).
	\item Spin 950/A5 PMMA on the stack at 4K RPM.
	\item Bake at 180C for 20 minutes.
	\item Condition the column (let it run at highest current at 30 kV). This may also be superstition, but I think charging the column can rearrange whatever dust or junk is on there and minimize the drift during actual writing.
	\item Set beam parameters to spot size 3, 10 kV.
	\item Using the large alignment marks (100 um separation) as alignment markers, write the beam dump and the alignment marks for the inner write field.
	\item Develop by dunking in 3:1 Water:Isopropyl Alcohol for 1 minute and blowing dry. Do this as gently as possible. Examine under microscope to make sure you have what you’re expecting.
	\item UV ozone for 90 seconds or so before putting in the evaporator (may be superstition).
	\item Evaporate Ti/Au in e-beam evaporator. 5 nm of Cr and an amount of Au equal to your tallest feature (or the entire heterostructure) plus 20 nm.
	\item Liftoff by submerging in acetone. This can be as short as 15 minutes or it can take up to an hour, but it shouldn’t take much longer than that.
\end{enumerate}

\subsubsection{Etch}
\begin{enumerate}
	\item Bake at 180C for 2 minutes.
	\item Spin 950/A4 PMMA on the stack at 4K RPM.
	\item Bake at 180C for 20 minutes.
	\item Set beam parameters to spot size 3, 30 kV.
	\item Write using NPGS.
	\item Develop by dunking in 3:1 Water:Isopropyl Alcohol for 1 minute and blowing dry. Examine under microscope.
	\item Write using the dgg hBN etch in the oxford plasma pro. Use cycles of two minutes of etching followed by one minute of cooldown. You eat about 20 nm of hBN in a minute. Typically 4 cycles of this etch (8 minutes total) is enough to eat through any reasonable heterostructure. I know the resist can take up to 12 minutes of etching. I don’t know much about going beyond this, but at 12 minutes the resist is pretty thin so I wouldn’t recommend doing it.
\end{enumerate}

\subsubsection{Ohmics}
\begin{enumerate}
	\item Bake at 180C for 2 minutes.
	\item Spin 950/A5 PMMA on the stack at 4K RPM
	\item Bake at 180C for 20 minutes.
	\item Set beam parameters to spot size 3, 10 kV.
	\item Write using NPGS.
	\item Develop by dunking in 3:1 Water:Isopropyl Alcohol for 1 minute and blowing dry. Examine under microscope.
	\item UV ozone for 30 seconds (may be superstition).
	\item Evaporate Cr/Au in e-beam evaporator. 5 nm of Cr and an amount of Au equal to your tallest feature (or the entire heterostructure) plus 20 nm. Pressure is critical here – do an overnight pumpdown and shoot off a bunch of Cr (10 nm) before opening the shutter to get the lowest pressure you can. Minimize the time between the Cr evaporation and the Au evaporation to minimize the amount that the Cr layer oxidizes. Shoot for the amount of time it takes the crucible to go dark plus two minutes. When heating the gold, let it outgas as it heats before opening the shutter. Evaporate Cr slowly (maybe 0.3 – 0.5 \AA /s) and Au more quickly (2 – 3 \AA /s).
\end{enumerate}

\section{Applying ionic liquid}
\begin{enumerate}
	\item Wearing gloves, dip the tip of a sterilized 250 mL pipet tip into the ionic liquid. There should be a drop of ionic liquid that clings to the outside of the pipet tip.
	\item Under a stereoscope, deposit the drop of ionic liquid onto a clean area of the substrate away from the active area of the device.
	\item Using a loop of gold wirebonding wire approximately 100 $\mu$m in diameter, pick up a small amount of the deposited drop.
	\item Press the ionic liquid in the wire loop onto the active area of the device. With practice, you can reliably deposit droplets of ionic liquid that are smaller than 100 $\mu$m in dimension.
\end{enumerate}

\section{Miscellaneous tips}
\begin{enumerate}
	\item Sometimes the anneal can spray contaminants all over your device, which sucks. To avoid this, anneal the tube and boat in open air at 1100 C for an hour. It gives off blackbody radiation, which is pretty cool looking.
	\item If you crosslink some PMMA, it’s not the end of the world. First, let it sit in acetone overnight. If that doesn’t work, you can try a high temperature oxygen anneal for a couple of hours. There is a spectrum of crosslinking, but I once crosslinked PMMA on a device and a high temp O2 anneal (500 C). for about 3 hours got rid of it.
\end{enumerate}

