\section*{How to read this dissertation}

Jump ahead to chapter XX to see Raman results, which are the new contribution

\chapter{Introduction}
Condensed matter physics is the rigorous study of what happens when a large number of cold atoms at high density are allowed to interact. It tells us why and at what temperature water freezes, why magnets attract some materials and not others, why glass is clear, why metal is shiny, and many other things. It is the branch of physics that reveals the richness of our physical world.

\section{Phases of matter}
Many materials that differ in their constituents and microscopic structure have similar bulk properties. For example, although water and mercury at ambient conditions have dramatically different densities and electrical conductivities, they are both nearly incompressible and deform continuously when a shear stress is applied. We capture these similarities by saying water and mercury are both in the liquid phase\footnote{Depending on the context, there may be a difference between a \textit{phase} of matter and and \textit{state} of matter. I will use phase in this dissertation as it seems to apply more generally.}. Phases of matter arise not just because of properties of the matter, but also how the matter is arranged.

A phase of matter has uniform equilibrium thermodynamic properties (density, magentization, etc.) and is defined by these properties being analytic functions of the thermodynamic parameters (e.g., temperature, pressure) \cite{Pathria2011}. Therefore, the properties of matter in static equilibrium in a given phase are the same for all space, and these properties are smooth functions of the parameters. For example, liquid water at a uniform temperature has the same density everywhere, and when it is heated by a small amount, its density decreases by a corresponding small amount. Phases of matter are separated by phase transitions, where the thermodynamic properties (or their derivatives with respect to a parameter) are no longer continuous\footnote{Infinite order phase transitions are a theoretical exception. See \cite{Costin1990}}. For example, when liquid water boils at ambient pressure, its temperature remains the same, but its density decreases discontinuously by a factor of $10^6$.

We can also use Landau theory \cite{Landau1937} to describe phases of matter by the symmetries of their Hamiltonian\footnote{A symmetry is an operation which leaves the Hamiltonian of the system invariant. For example, the Hamiltonian of a particle in free space $H = \Sigma_{i} \frac{p_{i}^{2}}{2m}$ is invariant under spatial translation $x \rightarrow x + a$}, and the phase transitions between them as the breaking or recovery of those symmetries. For example, when a liquid freezes into a solid crystal, the continuous translational symmetry of the liquid phase becomes a discrete translational symmetry as the molecules in the liquid assemble themselves into a liquid. Another example is a material transitioning from a non-magnetic to ferromagnetic phase. When the magnetic moments of the material align, it gains an overall macroscopic magnetization, breaking rotational symmetry. The following table lists some common phases and the symmetries they break \cite{Chaikin1995}.

\begin{center}
\resizebox{\textwidth}{!}{
	\begin{tabular}{l | l | l | l | l | l | l}
		\hline
		\hline
		\textbf{Phase} & Fluid & Nematic & Smectic-A & Crystal & Heisenberg Magnet & Superfluid \\ \hline
		\textbf{Broken Symmetry} & None & Rotational & 1D Translation & 3D Translation & Rotational & Phase\\ \hline \hline
	\end{tabular}
	}
	\captionof{table}{Selected phases and their associated broken symmetries}\label{tbl:nicetabelesstable}
\end{center}
		
However, in addition to the above examples, there are kinds of matter which maintain a single set of symmetries but nonetheless have phases separated by phase transitions. These kinds of matter possess topological order - a type of order that can define a phase of matter just like symmetry can \cite{Wen1990}.

\section{Topological Phases}

Topological order is a property of quantum systems that have both large ground state degeneracy and long-range entanglement. In these systems, there is no local order parameter like density or magnetization. Instead, there is a global topological invariant that changes discontinuously between phases \cite{Wen2017}. First, we set out to understand topology by considering the properties of a simple quantum topological system - Kitaev's toric code. Using the topological concepts we learn from the toric code, we will be able to understand the topological nature of some physical systems, including quantum hall states and the eventual subject of this dissertation: the spin liquid.

\subsection{The toric code}

The following explanation draws heavily from \cite{Kitaev2003} and \cite{topOrderEdX}.

Consider a system of spin-$\frac{1}{2}$ particles living on the edges of a square lattice with periodic boundary conditions. We first define two operators:

\begin{align*}
A_s&=\prod_{j \in star(s)} \sigma^{x}_{j} 		& B_{p}&=\prod_{j \in plaquette(s)} \sigma^{z}_{j}
\end{align*}

\begin{centering}
\includegraphics[width=0.5\textwidth]{C:/Users/dsbjr/Documents/GitHub/Dissertation/img/ToricCode-TopOrderEdX.jpg}
  \captionsetup{width=0.9\textwidth}
  \captionof{figure}[Toric code operators on the lattice]{A square lattice with a spin-$\frac{1}{2}$ particles on each edge. The toric code operators $A_{s}$ and $B_{p}$ act on the spins highlighted in blue and red, respectively. Diagram from \cite{topOrderEdX}.}
  \label{fig:toricCode1}
\end{centering}

The operator $A_{s}$ multiplies the spin in the x direction for all spins connected at vertex $s$ (for star). The operator $B_{p}$ multiplies the spin in the z direction for all the spins around a square $p$ (for plaquette). Operators $A_{s}$ and $B_{s}$ are both Hermitian with eigenvalues $\pm 1$. Perhaps surprisingly, they also commute. While it is clear the operators commute for distant vertices and plaquettes, we can also see that they commute for a vertex and plaquette with overlapping bonds because a vertex and plaquette will always share exactly two edges \cite{Kitaev2003}.

With an established understanding of $A_{s}$ and $B_{p}$, let us consider the ground state of the following Hamiltonian:

\begin{equation}
H_{\text{tor}} = -A \prod_{s} A_{s} - B \prod_{p} B_{p}
\end{equation}

Because $A_{s}$ and $B_{s}$ commute, the ground state of the Hamiltonian is simply the ground state that simultaneously minimizes the energy for both operators. Consider first the minimal energy state associated with $A_{s}$ in the $\sigma^{z}$ basis. Provided that exactly two bonds in a each plaquette have the same spin, the $A_{s}$ term in the Hamiltonian will be minimized. Having only two bonds in each plaquette with the same spin defines a ``loop gas" - any state that consists of closed loops of the same spin will be a satisfactory ground state. This large number of states that minimize the energy of the Hamiltonian satisfies the first requirement for topological order: massive ground state degeneracy.

\begin{centering}
\includegraphics[width=0.8\textwidth]{C:/Users/dsbjr/Documents/GitHub/Dissertation/img/ToricCode-Fractional-TopOrderEdX.jpg}
  \captionsetup{width=0.75\textwidth}
  \captionof{figure}[Fractional excitations in the toric code]{Loops and fractional excitations in the toric code. Each end is a topological defect that carries a fractional spin. Red and blue highlights relate to the $\sigma^{z}$ and $\sigma^{x}$ operators, respectively. Diagram from \cite{topOrderEdX}.}
  \label{fig:toricCode2}
\end{centering}

Now consider the minimal energy state associated with $B_{p}$. In the $\sigma^{x}$ basis, using the above argument we find that the minimal energy state consists of loops of the same spin drawn on the dual lattice.\footnote{The dual of a lattice $R$ is the set $\hat{R}$ of all vectors $\mathbf{x} \in$ span($\Lambda$) such that $\mathbf{x} \cdot \mathbf{y}$ is an integer for all $\mathbf{y} \in \Lambda$. In this case, the dual lattice is a square lattice with the same lattice constant having a vertex at the center of each plaquette in the original lattice.} But in the $\sigma^{z}$ basis, $\sigma^{x}$ is off-diagonal. Therefore the states that minimize $B_{p}$ in the $\sigma^{z}$ basis are necessarily superposition states - in this case an equal weight superposition of all the possible states that minimize $A_{s}$ \cite{Savary2017}.\footnote{This illustrates a beautiful symmetry of the toric code. In either (or any) basis, the ground state is highly entangled, as it must be.} This kind of superposition cannot be factored into a product state, and therefore satisfies the second requirement for topological order: long-range entanglement.

We have seen that the ground state of the toric code Hamiltonian satisfies the requirements for having topological order. But what can we observe about this system that shows it to have topological order? Further investigation of this system shows that it allows quasiparticle excitations with fractional quantum numbers.

Consider the ground state of the toric code Hamiltonian to be a vacuum state of closed spin loops. What if we were to flip a spin in the lattice? To do so, we would add some amount of energy (either $A$ or $B$ from the Hamiltonian) and break the loop into a string (see Figure~\ref{fig:toricCode2}). Any spin flip excitation must carry integer spin ($-\frac{1}{2}$ to $\frac{1}{2}$ or vice-versa is a change by an integer amount). However, when we break a loop into a string, we get \textit{two} quasiparticle excitation ``ends", each of which may diffuse around the lattice by smooth deformations\footnote{If we flip two and only two spins that share a vertex or a plaquette, we have not changed the energy of the system. Accordingly, such changes are allowed and can smoothly deform loops without breaking them} of the loops in the ground state. While flipping a spin changes the spin of the system by an integer, there are two physically separate quasiparticle excitations over which the integer spin is distributed. Each end of the string is therefore a quasiparticle excitation carrying fractional spin. Fractional excitations are the hallmark of topological order, and their presence in a system is proof that it is topologically nontrivial.

\begin{centering}
\includegraphics[width=0.8\textwidth]{C:/Users/dsbjr/Documents/GitHub/Dissertation/img/ToricCode-Torus-CC.jpg}
  \captionsetup{width=0.75\textwidth}
  \captionof{figure}[Topologically distinct loops in the toric code]{Red and blue lines show topologically distinct loops on the toric code lattice.}
  \label{fig:toricCode3}
\end{centering}

Another physical realization of the topology of the system is the four energetically degenerate but topologically distinct ground state configurations which are robust against local perturbations. Consider a loop that spans the boundaries of the lattice. This loop cannot be contracted to a point by any unitary locality-preserving operator, and is therefore topologically nontrivial. Further, a state with this loop is a ground state of the system, and therefore protected against local perturbations by the gap in the spectrum of $H_{\text{tor}}$ \cite{Bravyi2010}. There are four of these states, one for each way one can cross a periodic boundary (in the x and y directions, visualized in figure \ref{fig:toricCode3}), and one for each of the types of loops ($\sigma^{z}$ and $\sigma^{x}$). The number of boundary-spanning loops in a particular state serves at the topological invariant in this system and distinguishes topological states from one another. The topological invariant changes discontinuously as we move between states with different numbers of boundary-spanning loops. Rather than phases having different symmetries, we have phases with different topological invariants.

With the toric code, we have seen how massive ground state degeneracy and long-range entanglement give rise to fractional excitations and topological invariants that distinguish separate phases of matter. However, the toric code, while useful pedagogically, does not currently have a physical realization. To see physical evidence of topological order, we next turn to quantum Hall systems.

\subsection{Example: Quantum Hall Effects}

Here we briefly review the classical Hall effect, before describing the integer and fractional quantum Hall effects, which have topological order.

\subsubsection{The Classical Hall Effect}
A charged particle moving in a magnetic field experiences the Lorentz force, which is perpendicular to both the velocity of the particle and the magnetic field and proportional to the product of the charge, speed, and magnetic field strength \cite{Griffiths1999}:

\begin{equation}
\vec{\mathbf{F}} = q(\vec{\mathbf{v}} \times \vec{\mathbf{B}})
\end{equation}

During the diffusive transport of charge carriers (electrons or holes) in a conducting material exposed to a magnetic field, the Lorentz force gives rise to the classical Hall effect. Charge carriers accumulate along the edges of a current-carrying strip of material, creating an electric field transverse to the direction of the current and a corresponding Hall Voltage, given by:

\begin{equation}
V_{h} = \frac{I B R_{H}}{t}
\end{equation}

Where $I$,  $B$, $R_{H}$, and $t$ are the current, magnetic field strength, Hall coefficient (an intrinsic property of the material), and the thickness of the current-carrying sample, respectively. The Hall effect is often used to determine the charge and density of the charge carriers in a material \cite{Pierret2002}.

At low temperatures, low densities, and small length scales, the transport of charge carriers (assumed to be electrons for the remainder of this discussion) subject to a magnetic field is no longer dominated by scattering between electrons and impurities, nuclei, or other electrons. When electrons scatter so infrequently that they begin to display trajectories that no longer have the characteristics of a random walk (i.e., electrons move without collision long enough such that their trajectories bend due to the the Lorentz force), we say that the transport has transitioned from diffusive to ballistic. For our systems, this occurs when the average time between scattering events exceeds the cyclotron period. The delicate interaction between the kinetic and potential energy of ballistic electrons in a magnetic field gives rise to the rich physics of the quantum Hall effect \cite{Beenakker1991}.

\subsubsection{The Integer Quantum Hall Effect}

In 1980, von Klitzing, Dorda, and Pepper performed Hall measurements at high magnetic fields and low temperatures (up to 18 Tesla and as low as 1.5 Kelvin) on a two-dimensional electron gas (2DEG) at the inversion layer of a silicon-based metal oxide semi-conductor field effect transistor \cite{VonKlitzing1980}. The result of these measurements showed that the Hall voltage plateaued and the longitudinal voltage approached zero at certain discrete values of electron density (modulated by applying a gate voltage).

\begin{centering}
\includegraphics[width=0.5\textwidth]{C:/Users/dsbjr/Documents/GitHub/Dissertation/img/QHEMeasurement-vonK.png}
  \captionsetup{width=0.75\textwidth}
  \captionof{figure}[von Klitzing's measurement of the quantum Hall effect]{von Klitzing's measurement of the Hall voltage $U_{h}$ and longitudinal voltage $U_{pp}$ \cite{VonKlitzing1980}. Note that the minima in this figure at non-integer values of n result from spin and valley degeneracy.}
  \label{fig:IQHE1}
\end{centering}

We can understand the behavior of $U_{h}$ (the Hall voltage defined above as $V_{H}$) and $U_{pp}$ (the longitudinal voltage measured parallel to the direction of the current) by examining the wavefunction for electrons in a 2DEG subject to a magnetic field. Consider a system in which the magnetic field is oriented in the z-direction using the Landau gauge $\vec{\mathbf{A}} = (0,Bx,0)$. The Hamiltonian for this system\footnote{neglecting electron-electron interactions} is given by

\begin{equation}
\hat{H} = \sum_{i} \hat{H_{i}} = \frac{1}{2m_{e}} \left[ \hat{p}_{x,i}^{2} + \left(\hat{p}_{y,i} + \frac{eB}{c} \hat{x}_{i} \right)^{2} \right]
\end{equation}

Because the Hamiltonian does not explicit depend on the coordinates $y,i$, we can guess an ansatz solution of the form $\Psi(x,y) = \psi_{x}(x) e^{ik_{y}y}$ for a single electron. Solving the Schrodinger equation using this ansatz gives $\psi_{x}(x)$ as the solution to the quantum harmonic oscillator, and therefore the overall solution for a single electron becomes:

\begin{equation}
\Psi(x,y) = N e^{ik_{y}y}e^{\frac{(x-X)^{2}}{2(\hbar c/eB)}} H_{n} \left(\frac{x-X}{\sqrt{\hbar c/ eB}} \right)
\end{equation}

Where $X$ is a parameter related to the size of the cyclotron orbit, $N$ is a normalization constant, and $H_{n}$ is the nth Hermite polynomial. The ground state of the many electron wavefunction, the lowest Landau level can be written as \cite{Yoshioka2002}

\begin{equation} \label{psi-el}
\psi(z_{i},\overline{z}_{i}) = N \prod_{i > j} (z_{i} - z_{j}) e^{-\frac{eB}{4 \hbar c} \sum_{i} z_{i} \overline{z}_{i}}
\end{equation}

where $z_{i} = x_{i} + iy_{i}$ is the complex position of the \textit{i}th electron.

This wavefunction must be degenerate, as it is composed of single electron harmonic oscillator wavefunctions. In fact, the degeneracy is given by \cite{Chakraborty1995}:

\begin{equation}
\nu = \frac{n_{e}\Phi_{0}}{\Phi}
\end{equation}

where $n_{e}$ is the number of electrons and $\Phi$ and $\Phi_{0}$ are the magnetic flux and magnetic flux quantum, respectively\footnote{Here we assume that we work in a material without valley degeneracy and that the magnetic field is large enough such that the 2DEG is spin polarized (equivalently, that the Zeeman splitting is greater than $\hbar \omega_{c}$).}. Therefore, the ground state of this system is degenerate - a requirement for topological order. A discussion of entanglement in this system is beyond the scope of this dissertation.

We can see evidence of this topological order by considering the quantization of the Hall voltage in this system. The density of states is a superposition of harmonic oscillator states, so we write it\footnote{Note that this expression is exact for ideal materials at zero temperature only. Impurities and non-zero temperature will broaden the delta function peaks.} as:

\begin{equation}
D(E) \propto \sum_{n} \delta \left(E - \left(n + \frac{1}{2} \hbar \omega_{c} \right) \right)
\end{equation}

where $n$ is the index of the Landau level the electrons occupy, and $\omega_{c}$ is the cyclotron frequency given by $\omega_{c} = \frac{eB}{m_{e} c}$ \cite{Chakraborty1995}.

Consider the system at zero temperature. As the electron density is tuned, the Fermi energy in the sample passes through the delta functions in the expression for the density of states. Between the peaks in the density of states, the electrons are locked in cyclotron orbits and therefore the bulk of the system in insulating. However, the electrons on the edges of the system cannot complete a full cyclotron orbit without encountering the edge of the system. We can visualize these electrons as executing "skipping" orbits along the edges. These edge states carry current without dissipation and therefore $U_{pp}$ approaches zero and $U_{h}$ is fixed at a constant voltage. As the Fermi energy passes through each harmonic oscillator energy level, $\frac{U_{h}}{I}$ changes between different integer multiples of $R_{k} \approx 25812$ $\Omega$, the resistance quantum. We can see the hallmarks of topological order in this system by interpreting $n_{e}$ as the tunable parameter and $R_{h} = \frac{U_{h}}{I} = \nu R_{k}$\footnote{$\nu$ is also called the filling factor, because it can be interpreted as the number of Landau levels that are filled for a given electron density and magnetic flux. For filling factors having $\nu$ less than one, there is at least one flux quantum for every electron, so all the electrons are in the lowest Landau level. For $\nu$ greater than one, each flux quanta already is already associated with at least one electron, so by the Pauli exclusion principle the remaining electrons are pushed into higher Landau levels.} as the topological order parameter. As we tune the density, the topological order parameter changes discontinously between distinct quantum Hall phases.

A complete discussion entanglement and the topological properties of the IQHE requires an appeal to gauge symmetries and group theory beyond the scope of this dissertation (see section III of \cite{Wen2017} for further information). However, the topological order inherent in the IQHE manifests more easily in the fractional quantum Hall effect, where excitations are fractionalized and $\nu$ discussed above takes on non-integer values.

\subsubsection{The Fractional Quantum Hall Effect}

In 1982, Tsui, Stormer, and Gossard measured a quantized Hall voltage at $\nu = \frac{1}{3}$ \cite{Tsui1982}, which they called "striking evidence for a new electronic state..." (their results also suggested quantized Hall voltages for $\nu = \frac{2}{3}$ and $\nu = \frac{3}{2}$, but the mobility of electrons in the sample and the measurement temperature prevented confirming these states). We would not expect a gap in the density of states at filling factors corresponding to fractions. Accordingly, the analysis for the integer quantum Hall effect breaks down in the presence of Tsui's measurement.

In 1983, Laughlin, by a variational method, proposed the form of a wavefunction that accurately describes the Fractional Quantum Hall Effect (FQHE) identified by Tsui \cite{Laughlin1983}. The Laughlin wavefunction is given by

\begin{equation}
\psi = \prod_{i < j} \left( \frac{z_{i} - z_{j}}{l_{B}} \right)^{3} e^{-\frac{1}{4} \sum_{i} \frac{|z_{i}|^{2}}{l_{B}^{2}}}
\end{equation}

where $z_{i}$ is the complex position of the \textit{i}th electron as before and $l_{B}$ is the magnetic length given by $l_{B} = \sqrt{\frac{\hbar}{eB}}$.

Further measurements confirmed the presence of a series of fractional quantum Hall states occurring at filling factors of the form $\nu = \frac{1}{p}$, where where $p$ is an odd integer. The Laughlin wavefunction, generalized to the form

\begin{equation}
\psi = \prod_{i < j} \left( \frac{z_{i} - z_{j}}{l_{B}} \right)^{\frac{1}{\nu}} e^{-\frac{1}{4} \sum_{i} \frac{|z_{i}|^{2}}{l_{B}^{2}}}
\end{equation}

also describes these states. Note that $p$ must be odd such that the wavefunction is antisymmetric and describes fermions.

\begin{centering}
\includegraphics[width=0.5\textwidth]{C:/Users/dsbjr/Documents/GitHub/Dissertation/img/FQHEMeasurement-Willet.png}
  \captionsetup{width=0.75\textwidth}
  \captionof{figure}[Fractional Quantum Hall Effect Measurement]{Willet's measurement of the Hall resistivity as a function of magnetic field \cite{Willet1987}. Note that the large number of fractions.}
  \label{fig:FQHE1}
\end{centering}

The Laughlin wavefunction describes a system of electrons that avoid each other in an optimal way. For a state with $\nu = 1/p$, the wavefunction is proportional to the p$^{\text{th}}$ power of the separation between the electrons. This factor in the wavefunction yields two results. First, if an electron approaches within a magnetic length of another electron, the magnitude of the wavefunction drops dramatically, meaning electrons must stay far apart. Second, the wavefunction cannot be factored into single-particle states and therefore is massively entangled.

A fractional quantum Hall state meets the requirements for topological order and shows fractional excitations like those in the toric code. A direct measurement of the charge of quasiparticle excitations in the $\nu=\frac{1}{3}$ state shows that electrons carry a charge of $\frac{e}{3}$ \cite{Goldman1995}.

\subsubsection{Conclusion}

Through our examination of the quantum Hall effect, we have seen a physical realization of a topologically-ordered phase of matter.  For these systems, the tunable parameter is the electron density, the topological invariant is related to the filling factor, and the phases are separated by discontinuous changes in the topological invariant that we observe as changes in the quantized conductance. These phases demonstrate ground state degeneracy and entanglement, and show fractionalized excitations that are the hallmark of topological order.

\section{Spin Liquids}

Now that we have an understanding of topological order and have examined a physical realization of a topologically-ordered system, we turn our attention to the subject of this dissertation: the spin liquid.

\subsection{What is a Spin Liquid?}

Spin liquids are challenging to define. In fact, it is easier to say what they \textit{aren't} rather than what they are. As a working definition, we can define spin liquids as a systems with a ground state composed of a quantum superposition of well-formed, correlated, local magnetic moments that do not develop long-range order, even at T = 0 K \cite{Balents2010}. We will later see how this description requires the elements of topological order. But first, to get a physical understanding of spin liquids, we'll look at geometric frustration.

\subsubsection{Geometric Frustration}

Consider a system of Ising spins interacting on a square lattice and a triangular lattice according to the following Hamiltonian: 

\begin{equation}
H = \sum_{\text{N.N.}} -J s_{i} \cdot s_{j}
\end{equation}

On the square lattice, given either a ferromagnetic interaction ($J > 0$) or an antiferromagnetic interaction ($J < 0$), there is a single minimal energy configuration that is the ground state. The nearest neighbor spins are either parallel or anti-parallel, respectively. These are the ferromagnetic and antiferromagnetic (\textit{Ne{\'e}l}) states. There is also a unique minimal energy configuration for the ferromagnetic interaction on the triangular lattice. But what is the ground state for an antiferromagnetic interaction on the triangular lattice?

\begin{centering}
\includegraphics[width=0.5\textwidth]{C:/Users/dsbjr/Documents/GitHub/Dissertation/img/Frustration-Balents2010.png}
  \captionsetup{width=0.75\textwidth}
  \captionof{figure}[Geometric Frustration on the Triangular Lattice]{Degenerate lowest energy states for the antiferromagnetic nearest neighbor interaction on the triangular lattice. Red lines highlight the interaction between parallel spins which is energetically unfavorable. From \cite{Balents2010}.}
  \label{fig:Frustration1}
\end{centering}

On the triangular lattice, it is not possible to put all three spins into their lowest-energy configuration. For a given plaquette, the system achieves the lowest energy by having two spins parallel. There are three such states\footnote{There are actually six, but global spin flip symmetry makes three of these redundant.} which are energetically degenerate. We describe such a system as being ''frustrated" because there is no state where each interaction achieves its minimal energy. This frustration prevents antiferromagnetic Ising spins on a triangular lattice from developing long-range magnetic order at any temperature \cite{Wannier1950}.

While the antiferromagnetic interaction on a triangular lattice is a straightforward example of geometric frustration, other lattices (such as the Kagom{\'e} lattice in 2D and the pyrochlore lattice in 3D) may be geometrically frustrated when there is a nearest neighbor ferromagnetic or antiferromagnetic interaction. Despite their frustration, many of these systems have interactions (e.g., next-nearest neighbor) that cause them to develop magnetic order at low temperature. We can compare the degree of frustration in frustrated systems that order by calculating the frustration factor \cite{Ramirez1994}:

\begin{equation}
f = \frac{|\Theta_{cw}|}{T_{c}}
\end{equation}

Where $\Theta_{cw}$ is the Curie-Weiss temperature\footnote{Note that for an antiferromagnetic interaction $\Theta_{cw}$ is negative; therefore, we use the absolute value.} extracted from the high-temperature magnetic susceptibility and $T_{c}$ is the temperature at which the material magnetically orders. We can think of $f$ as telling us how strongly the order is suppressed by geometric frustration - the material would like to order at something like $\Theta_{cw}$, but given the geometric frustration, it orders at $T_{c}$ instead. A value of $f$ between 5 and 10 indicates strong geometric frustration. In the range $\Theta_{cw} < T < T_{c}$, we would expect frustration to play a factor: above $\Theta_{cw}$ thermal excitations keep the material from ordering, below $T_{c}$ the material is ordered.

\begin{center}
	\begin{tabular}{l | l | l | l | l}
		\hline
		\hline
		Material & VCl\textsubscript{2} & NaTiO\textsubscript{2} & ZnCr\textsubscript{2}O\textsubscript{4} & K\textsubscript{2}IrCl\textsubscript{6}\\ \hline
		Frustration Factor & 12 & $>$500 & 24 & 10\\ \hline \hline
	\end{tabular}
	\captionof{table}[Frustration factors for selected geometrically frustrated magnets]{Frustration factors for selected geometrically frustrated magnets \cite{Ramirez1994}}\label{tbl:frustratedmagens}
\end{center}

While most geometrically frustrated systems eventually develop magnetic order, there are some that do not (i.e., $f = \infty$). We will discuss two: spin ices and spin liquids.

\subsubsection{Frustration and Spin Ices}

A spin ice is a frustrated magnetic system with macroscopic ground state degeneracy\footnote{This system has \textit{effective} macroscopic ground state degeneracy. There is actually a single minimal energy ground state for spin ices, but this state is not physically accessible because there are many other disordered states of the only infinitesimally higher energy separated from the true ground state by infinite energy barriers \cite{Siddharthan2001}. Because the system freezes before any ordering can be observed, the ground state is effectively disordered \cite{Castelnovo2012}.} that freezes into a disordered configuration at sufficiently low temperature\footnote{These systems are called spin ices because the arrangements of the spins follow the so-called ``ice rules" for proton arrangement in water ice.}. As a spin ice is cooled, thermal fluctuations are no longer sufficiently energetic to change the spin configuration and the system is ``trapped" in a static, randomly chosen, disordered state \cite{Bramwell2001}. We can imagine that the system locally picks a spin configuration from the ensemble of degenerate minimal energy states and freezes there.

\begin{centering}
\includegraphics[width=0.3\textwidth]{C:/Users/dsbjr/Documents/GitHub/Dissertation/img/PyrochloreLattice-Balents2010.png}
  \captionsetup{width=0.75\textwidth}
  \captionof{figure}[Frustration on a pyrochlore lattice]{Frustration on a pyrochlore lattice. Spins occupy the corner of each tetrahedron and must either point directly toward or away from the center of the tetrahedron. Diagram from \cite{Balents2010}.}
  \label{fig:pyrochlorelattice}
\end{centering}

Ho\textsubscript{2}Ti\textsubscript{2}O\textsubscript{2} was the first confirmed spin ice. This material has a pyrochlore lattice, with Ho\textsuperscript{+3} serving as the magnetic ion at the corner of each tetrahedron. Spins in this system obey a ``two-in, two-out" rule, meaning that the net spin in a tetrahedron is zero. Neutron scattering has shown that Ho\textsubscript{2}Ti\textsubscript{2}Os\textsubscript{2} does not develop magnetic order down to 50 mK \cite{Harris1997}. However, the field dependence of neutron scattering shows that applying a magnetic field before cooling can freeze the system in a particular disordered state. This change can be removed by heating and subsequently cooling the sample in zero field. Finally, entropy measurements near 0 K show that the entropy remaining in the ground state indicates a disorder similar to that of water ice, confirming Ho\textsubscript{2}Ti\textsubscript{2}O\textsubscript{2} as a spin ice \cite{Ramirez1999}.

We have seen that spin ices satisfy the criteria for topological order: massive ground state degeneracy from the frustration of the system, and entanglement of the spins from the ice rules. Accordingly, we could expect spin ices to host fractional excitations, which they do. Consider a tetrahedron in the pryochlore lattice with spins localized on each vertex obeying the ``two-in, two-out" ice rules. If we pay the energetic cost to flip a single spin, we produce two tetrahedra that no longer satisfy the ice rules; one tetrahedron has excess spin, the other tetrahedron is spin deficient.

<insert image from rikita dusad's paper>

Subsequent spin flips can allow these two tetrahedra to diffuse through the lattice without violating the ice rules. Because spin flips without ice rule violations do not cost energy, these spinful defects (hereafter referred to as charges) are not confined. We can see that these charges are fractionalized excitations in two ways. First, there are two defects generated from a single spin flip, meaning the elementary excitation of the system has been split in two. Second, we can see these charges as magnetic monopoles of an emergent gauge field.

<insert image from spin ice, fractionalization, etc>

Each tetrahedron in the spin ice has net zero spin. If we superimpose hypothetical field lines connecting and directed along each spin, the result is a divergenceless field. The charges are then sources and sinks of these field lines. If we imagine the divergenceless field as the \textbf{B} field to which it is mathematically identical, the charges are analogous to magnetic monopoles. Both neutron scattering <cite neutron scattering paper> and direct measurement using a SQUID <cite rikita dusad's paper> confirm the existence of these excitations.

We've already seen that spin ices satisfy the requirements for topological order and have fractional exciaitons. But describing the spin ice as a material having monopoles as fractional excitations gives us an example of another way to identify topological systems: the presence of an emergent gauge field. Emergent gauge fields are associated with topological order \cite{Sachdev2019}, so if we identify a system with an emergent gauge field, we can be reasonably sure it is topological. Armed with a description of the topological nature of spin ices and emergent gauge fields, we can now discuss spin liquids.

\subsubsection{Frustration and Spin Liquids}

Rigorously defined, a spin liquid is a ground state composed of a quantum superposition of well-formed, correlated, local magnetic moments that do not develop long-range order, even at T = 0 K. But a more physical way to understand a spin liquid is to compare it to a spin ice. A spin liquid is, in some ways, the quantum analog of a spin ice that \textit{never} freezes. To understand this analogy, let's return to the example of antiferromagnetic Ising spins on a triangular lattice.

Like in the spin ice, each spin on the triangular lattice is frustrated by the antiferromagnetic interaction, so there are many states that are degenerate. Also, like in the spin ice, the antiferromagnetic interaction forces the spins to be sensitive to the orientation of their nearest neighbors.

The minimal energy state of this system will have as many nearest neighbor spins as possible oriented anti-parallel to their neighbors. Of course, there are many such states, none of which is preferable to the others. Accordingly, at T = 0 K, the ground state of this system should be a superposition of the degenerate states that minimize the energy.

<insert picture of RVB state here>

This ground state, being composed of only those degenerate states with certain nearest neighbor energy states, is no longer a product state. Instead, the ground state is highly entangled. 


<Need some text to go here, dealing with topological order in spin ices>
\subsection{Why are Spin Liquids Interesting?}