\chapter{Introduction}
Condensed matter physics is the rigorous study of what happens when a large number of cold atoms at high density are allowed to interact. It tells us why and at what temperature water freezes, why magnets attract some materials and not others, why glass is clear, why metal is shiny, and many other things. It is the branch of physics that reveals the richness of our physical world.

\section{Phases of matter}
Many materials that differ in their constituents and microscopic structure have similar bulk properties. For example, although water and mercury at ambient conditions have dramatically different densities and electrical conductivities, they are both nearly incompressible and deform continuously when a shear stress is applied. We capture these similarities by saying water and mercury are both in the liquid phase\footnote{Depending on the context, there may be a difference between a \textit{phase} of matter and and \textit{state} of matter. I will use phase in this dissertation as it seems to apply more generally.}. Phases of matter arise not just because of constituent particles, but also by the way those constituents are arranged.

A phase of matter has uniform equilibrium thermodynamic properties (density, magentization, etc.) and is defined by these properties being analytic functions of the thermodynamic parameters (e.g., temperature, pressure) \cite{Pathria2011}. Therefore, the properties of matter in static equilibrium in a given phase are the same for all space, and these properties are smooth functions of the parameters. For example, liquid water at a uniform temperature has the same density everywhere, and when it is heated by a small amount, its density decreases by a corresponding small amount. Phases of matter are separated by phase transitions, where the thermodynamic properties (or their derivatives with respect to a parameter) are no longer continuous\footnote{Infinite order phase transitions are a theoretical exception. See \cite{Costin1990}}. For example, when liquid water boils at ambient pressure, its temperature remains the same, but its density decreases discontinuously by a factor of $10^6$.

We can also use Landau theory \cite{Landau1937} to describe phases of matter by the symmetries of their Hamiltonian\footnote{A symmetry is an operation which leaves the Hamiltonian of the system invariant. For example, the Hamiltonian of a particle in free space $H = \Sigma_{i} \frac{p_{i}^{2}}{2m}$ is invariant under spatial translation $x \rightarrow x + a$}, and the phase transitions between them as the breaking or recovery of those symmetries. For example, when a liquid freezes into a solid crystal, the continuous translational symmetry of the liquid phase becomes a discrete translational symmetry as the molecules in the liquid assemble themselves into a liquid. Another example is a material transitioning from a non-magnetic to ferromagnetic phase. When the magnetic moments of the material align, it gains an overall macroscopic magnetization, breaking rotational symmetry. The following table lists some common phases and the symmetries they break \cite{Chaikin1995}.

\begin{center}
\resizebox{\textwidth}{!}{
	\begin{tabular}{l | l | l | l | l | l | l}
		\hline
		\hline
		\textbf{Phase} & Fluid & Nematic & Smectic-A & Crystal & Heisenberg Magnet & Superfluid \\ \hline
		\textbf{Broken Symmetry} & None & Rotational & 1D Translation & 3D Translation & Rotational & Phase\\ \hline \hline
	\end{tabular}
	}
	\captionof{table}{Selected phases and their associated broken symmetries}\label{tbl:nicetabelesstable}
\end{center}
		
However, in addition to the above examples, there are kinds of matter which maintain a single set of symmetries but nonetheless have phases separated by phase transitions. These kinds of matter possess topological order - a type of order that can define a phase of matter just like symmetry can \cite{Wen1990}.

\section{Topological Phases}

Topological order is a property of quantum systems that have both long-range entanglement and large ground state degeneracy. In these systems, there is no local order parameter like density or magnetization. Instead, there is a global topological invariant that changes discontinuously between phases \cite{Wen2017}. First, we set out to understand topology by considering the properties of a simple quantum topological system - Kitaev's toric code. Using the topological concepts we learn from the toric code, we will be able to understand the topological nature of some physical systems, including quantum hall states and the eventual subject of this dissertation: the spin liquid.

\subsection{The toric code}

The following explanation draws heavily from \cite{Kitaev2003} and \cite{topOrderEdX}.

Consider a system of spin-$\frac{1}{2}$ electrons living on the edges of a square lattice. We first define two operators:

\begin{align*}
A_s&=\prod_{j \in star(s)} \sigma^{x}_{j} 		& B_{p}&=\prod_{j \in plaquette(s)} \sigma^{z}_{j}
\end{align*}


 

Fuck all this - teach topology with the Toric code.

Interacting system spins on a 2D lattice with a special hamiltonian. The ground state of this Hamiltonian is a loop gas - as long as there are no free ends, the energy of the system is the same. Therefore, any state with loops is a good state. The vertex operator changes between degenerate states. Basically, it changes the loop configuration.

If you put this lattice on a torus, then you get loops that cannot be deformed by the vertex operator smoothly into other loops. These are the loops that go across the periodic boundary. 

Things to remember: Quantum hall states have topological order - consider the conductivity an order parameter - as you change the field this parameter changes discontinuously without a change in symmetry. The chern number you can derive from the band structure (?) and is a topological order parameter that changes. Maybe I can start with the quantum hall effect?

Start with IQHE - explain it and then show how the conductivity is an order parameter. Then show how the FQHE extends the IQHE. Then show the topological excitations in the FQHE are evidence of topological order and their presence requires the ground state to have topological order. Then show that spin liquids are a state that has topological excitations, and therefore topological order - even in the absence of symmetry breaking.

Somehow topological order is connected to fractional excitations in two dimensions. Fractional quantum hall liquids always have edge excitations.

Topological order is connected to the ground state degeneracy of a system. The ground state degeneracy is somehow related to the topological order parameter.

Consider using the Toric code.

What is condensed matter, what's a spin liquid, why gate it, etc... This is going okay so far.