\chapter{Introduction}
Condensed matter physics is the rigorous study of what happens when a large number of cold atoms at high density are allowed to interact. It tells us why and at what temperature water freezes, why magnets attract some materials and not others, why glass is clear, why metal is shiny, and many other things. It is the branch of physics that reveals the richness of our physical world.

\section{Phases of matter}
Many materials that differ in their constituents and microscopic structure have similar bulk properties. For example, although water and mercury at ambient conditions have dramatically different densities and electrical conductivities, they are both nearly incompressible and deform continuously when a shear stress is applied. We capture these similarities by saying water and mercury are both in the liquid phase\footnote{Depending on the context, there may be a difference between a \textit{phase} of matter and and \textit{state} of matter. I will use phase in this dissertation as it seems to apply more generally.}.

A phase of matter has uniform equilibrium thermodynamic properties (density, magentization, etc.) and is defined by these properties being analytic functions of the thermodynamic parameters (e.g., temperature, pressure) \cite{Pathria2011}. Essentially, for small changes of parameters in a a particular phase, the thermodynamic properties are smooth functions of the parameters. Phases of matter are separated by phase transitions, where the thermodynamic properties or their derivatives are no longer continuous. For example, when liquid water boils at ambient pressure, its density decreases discontinuously by a factor of $10^6$, even though its temperature remains the same.

In many cases, we can also describe phases of matter by their different symmetries, and the phase transitions between them as the breaking or recovery of those symmetries\footnote{Not all phase transitions break symmetries, but this a useful tool in understanding phases}. For example, when a liquid freezes into a solid crystal, the translational symmetry of the liquid phase is broken as the molecules in the liquid assemble themselves into a liquid. Another example is a material transitioning from a non-magnetic to ferromagnetic phase. When the magnetic moments of the material align, it gains an overall macroscopic magnetization, breaking rotational symmetry. The following table lists some common phases and the symmetries they break \cite{Chaikin1995}.

\begin{center}
\resizebox{\textwidth}{!}{
	\begin{tabular}{l | l | l | l | l | l | l}
		\hline
		\hline
		\textbf{Phase} & Fluid & Nematic & Smectic-A & Crystal & Heisenberg Magnet & Superfluid \\ \hline
		\textbf{Broken Symmetry} & None & Rotational & 1D Translation & 3D Translation & Rotational & Phase\\ \hline \hline
	\end{tabular}
	}
	\captionof{table}{Selected phases and their associated broken symmetries}\label{tbl:nicetabelesstable}
\end{center}
		

What is condensed matter, what's a spin liquid, why gate it, etc...