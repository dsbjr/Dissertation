\chapter{Conclusion}

This dissertation investigated \rucl doped by electrolyte biasing and found that, contrary to predictions, \rucl remained highly insulating. Further, Raman spectra for electrolyte-biased \rucl identified a hysteretic transition between two visually distinct states driven by electrolyte bias. Finally, x-ray diffraction measurements showed these distinct states do not result from the cations in the electrolyte physically interacting with the \ruclnospace . In this concluding section, I review the results of these measurements and identify possible frameworks that could explain the data. I then suggest future measurements that could further clarify the nature of electrolyte-biased \ruclnospace .

\section{Summary of Results and Possible Explanations}



\section{Future Work}
\begin{itemize}
\item Better x-rays to examine new structure
\item Pulse echo measurements of gated sample to find evidence of phase change
\end{itemize}

Possible references:
%Kim, H.-S., & Kee, H.-Y. (2015). Crystal structure and magnetism in alpha-RuCl3: an ab-initio study, 155143, 1–10. https://doi.org/10.1103/PhysRevB.93.155143
Discusses different crystal structures