\chapter{Conclusion}

This dissertation investigated \rucl doped by electrolyte biasing and found that, contrary to predictions, \rucl remained highly insulating. Further, Raman spectra for electrolyte-biased \rucl identified a hysteretic transition between two visually distinct states driven by electrolyte bias voltage. Finally, x-ray diffraction measurements showed these distinct states do not result from the cations in the electrolyte physically interacting with the \ruclnospace . In this concluding section, I review the results of these measurements and identify possible frameworks that could explain the data. I then suggest measurements that could further clarify the nature of electrolyte-biased \ruclnospace .

\section{Hypotheses}

\subsection{Unlikely Explanations}
\begin{itemize}
\item \textbf{Electrochemistry:} In an electrolyte gating experiment, the first confounding effect that must be addressed is electrochemistry. However, there are several observations that rule out electrochemistry as the primary explanation of the behavior of electrolyte biased \ruclnospace . Measurements are made only within the electrochemical stability window of the electrolyte. Gate current remains low during measurements and, after an increase in gate voltage, there is a transient gate current that decays to an equilibrium gate current. The transition between the two states with distinct Raman spectra is repeatable, ruling out an irreversible chemical interaction. Finally, x-ray diffraction measurements show that the electrolyte does not penetrate the \rucl lattice, indicating that only the top and sides of the \rucl are available for chemical interaction. However, the presence of substrate features in the Raman spectrum indicates that the excitation characterizes the entire flake, which is dominated by the chemically-unaffected bulk.

\item \textbf{Intercalation:} Only cations intercalate in \ruclnospace , given the planes of chlorine atoms present at each interface. However, it seems unreasonable to expect a cation as large as DEME+ to fit between the layers. Additionally, x-ray diffraction measurements show the interlayer separation remains a constant 5.7 \AA{} and does not change as a function of gate voltage. The layers do not move and there is no room for DEME+ to fit. Intercalation and subsequent exfoliation of a single top layer is unlikely (there is no reason to suspect only the top layer intercalates when previous measurements show uniform intercalation) and cannot explain the behavior of the bulk, which dominates the Raman signal. However, intercalation of only protons, which may not be observable by x-ray diffraction, cannot be ruled out.

\item \textbf{Stress caused by electrostatics:} \rucl in these measurements is a thin, charged flake immersed in a biased electrolyte. Accordingly, the electric fields may subject the flake to stress that causes changes in its Raman spectrum. Further, this stress could be consistent with the mottled appearance of the flakes. However, if the stress is electrostatic, then it should appear at both positive and negative electrolyte bias voltage. Instead, the response of the material is asymmetric with respect to gate voltage.

\item \textbf{Charge localization:} The electrolyte bias adds charge to \ruclnospace , but the material could remain insulating because the charge is localized by lattice deformations and cannot participate in transport. However, charge localization should be continuously dependent on the amount of charge added, and there are only sharp changes in the Raman spectrum as a function of electrolyte bias. Additionally, lattice deformations associated with charge defects, like those in color centers in alkali halides, show the presence of new, broad peaks associated with the lattice deformations. Such features are not present in either Raman spectrum for electrolyte-biased \ruclnospace .
\end{itemize}

\subsection{The case for a phase transition}

The above discussion shows that many of the experimental artifacts or less interesting explanations for the behavior of electrolyte-biased \rucl can be ruled out. What remains is the possibility of structural or electronic phase transition. The transition between states having different Raman spectra is hysteretic and has an interface, showing the transition is first order. Further, there are several different \rucl structures that are nearly degenerate in energy that gating could drive transitions between. Additionally, there is evidence that charge doping in \rucl creates charge order. If the charge-ordering couples to the lattice, then an electronic transition could explain the behavior. Future work should explore these possibilities.

\section{Future Work}

Future investigation of electrolyte-biased \rucl should focus on x-ray diffraction measurements. Brighter x-rays from a synchrotron light source would be able to resolve scattering planes with a component parallel to the c axis, and therefore better characterize the structure of \rucl in either state. Further, diffuse x-ray scattering, which measures correlations in electron density across unit cells, could characterize the role of stress in the transition.

If additional x-ray measurements do not show a change in the structure of \ruclnospace , then the possibility of an electronic transition should be considered. Charge-ordered materials have different transport properties depending on the orientation of the stripes, so if the transport properties of electrolyte-biased \rucl are anisotropic, then the transition may be structural instead of electronic.

Possible references:
%Kim, H.-S., & Kee, H.-Y. (2015). Crystal structure and magnetism in alpha-RuCl3: an ab-initio study, 155143, 1–10. https://doi.org/10.1103/PhysRevB.93.155143
Discusses different crystal structures