\chapter{Properties of \texorpdfstring{$\alpha$-Ruthenium (III) Chloride}{alpha-RuCl3}}
This chapter covers the structural, electronic, and magnetic properties of \rucl and serves as a reference for further discussion of the material. This chapter also contains a discussion of the Kitaev model and how \rucl realizes this model in the lab. Finally, I discuss published measurements of \rucl and how they confirm it as a proximate Kitaev spin liquid.

\section{Crystal Structure and Synthesis}

Generally, ruthenium (III) chloride (\ruclnospace) is a brown or black solid that is a precursor for 	ruthenium chemistry. It is commonly found as a hydrate, where water molecules are incorporated into its structure. High-purity hydrated ruthenium (III) chloride is commercially available from chemical vendors like Sigma Aldrich. We will consider only anhydrous ruthenium (III) chloride (i.e., without water) in this dissertation.

Ruthenium (III) chloride has two polymorphs, called $\alpha$ and $\beta$. Both $\alpha$ and $\beta$-ruthenium (III) chloride are composed of ruthenium atoms octahedrally coordinated by chlorine. However, in the $\alpha$ polymorph the octahedra are edge sharing, while in the $\beta$ polymorph, the octahedra are face-sharing. The edge-sharing octahedra give the $\alpha$ polymorph its interesting magnetic qualities, and therefore, unless explicitly stated otherwise, we will confine our discussion to $\alpha$ ruthenium (III) chloride and refer to it as \ruclnospace .

At room temperature and ambient pressure, the structure of \rucl consists of edge-sharing octahedra of chlorine atoms that coordinate ruthenium. These edge-sharing octahedra form a hexagonal lattice that defines the a-b plane of the material. While interactions between atoms in the a-b plane are ionic bonds\footnote{I describe in-plane bonding as ionic because it concerns bonds between metals and non-metals. However, the modest difference in Pauling electronegativity between ruthenium and chlorine (0.96) suggests the bond has substantial covalent character.}, the only interactions along the c-axis are van der Waals bonds. Accordingly, \rucl is referred to as a quasi-2D, two-dimensional, or van der Waals material. The layers have an ABC stacking order, such that the unit cell contains three repeats of the a-b plane. The layer separation is approximately 5.7 \AA, while the separation between ruthenium and chlorine atoms in the a-b plane is approximately 0.6 \AA. This structure is monoclinic and belongs to space group C2/m. As show in Figure \ref{fig:RuCl3CrystStruct-1}, the angles between Ru-Cl bonds are very nearly 90$^{\circ}$. The nearly ideal coordinated of ruthenium atoms will be important to understanding \rucl as a proximate Kitaev spin liquid.

\begin{centering}
\includegraphics[width=0.5\textwidth]{C:/Users/dsbjr/Documents/GitHub/Dissertation/img/RuCl3CrystalStructure-Plumb.png}
  \captionsetup{width=0.75\textwidth}
  \captionof{figure}[Crystal structure of \ruclnospace]{Crystal structure of \rucl (Figure from \cite{Plumb2014}). (a) View of the b-c plane, showing ABC stacking of the layers. (b) View of the a-b plane showing RuCl$_{6}$ octahedra form a hexagonal lattice. (c) \rucl octahedra showing bond angles very close to 90$^{\circ}$.} 
  \label{fig:RuCl3CrystStruct-1}
\end{centering}

\rucl exhibits a structural phase transition from monoclinic (space group C2/m) to rhombohedral (space group R$\overline{3}$) near 150K \cite{Glamazda2017}. This phase transition shows a large degree of thermal hysteresis and has been identified in both magnetic susceptibility and x-ray diffraction measurements \cite{Park2016}. Electronic transport measurements also claim to show evidence of this transition \cite{Mashhadi2018}.

\rucl may be synthesized by several methods, including by direct reaction of Ru and Cl at elevated temperatures in the presence of carbon monoxide \cite{Binotto1971}, chemical vapor growth \cite{Gronke2018}, and by vapor transport techniques using purified commercial $\beta$-\ruclnospace \cite{Cao2016}. The \rucl crystals used for this work were grown by Stephen Nagler and colleagues at Oak Ridge National Laboratory using the latter method.

\section{Electronic Structure}

\rucl is a terrible conductor of electricity. But this highly insulating nature comes from a rich electronic structure that will be relevant when interpreting electronic transport data. Therefore in this section I will briefly review band theory and Mott insulators before showing how these concepts explain the peculiarities of \ruclnospace 's electronic structure.

\subsection{Review: Band Theory}

Matter in the solid state is composed of individual atoms separated by distances so small (typically \AA s or less) that the atomic orbitals interact and combine to form a broader set of molecular orbitals extending over the entire material. These nature of these broader orbitals defines the range of energies an electron in the solid may have, called bands, and the range of energies an electron may not have, called band gaps. Band theory calculates the bands and band gaps of a material from the properties of allowed non-interacting, single electron electron wavefunctions in a periodic potential. Although based on assumptions that are not always applicable, band theory successfully explains many properties of solid state systems, including electrical resistivity and optical absorption. Before examining the electronic structure of \ruclnospace , we will briefly review band theory to understand some general aspects of band structure.

In a crystalline material, each unit cell consists of an identical set of atomic potentials and orbitals that repeat to produce larger orbitals. In the thermodynamic limit, the wavefunction of an electron in the material will experience periodic boundary conditions across the unit cell. Effectively, the wavefunction must be periodic in space with a period that matches the size of the unit cell. This is \textit{Bloch's Theorem} and it can be expressed mathematically as:

\begin{equation}
\psi_{k}(x + a) = e^{ika} \psi_{k}(x)
\end{equation}

where $a$ is the period of the lattice, $k$ is a wavenumber related to the crystal momentum\footnote{The crystal momentum $\hbar k$ is an effective momentum for a particle traveling in a lattice.}, and $\psi_{k}$ is the wavefunction associated with crystal momentum $k$ \cite{Davies1997}.

For reasons that will become clear, not all values of $k$ are permitted. These prohibited values of $k$ are the previously discussed band gaps, and are a natural consequence of electrons in periodic potentials. To understand how band gaps emerge, we will go through a quick analysis of the Kronig-Penney model, loosely following \cite{Davies1997}.

The Kronig-Penney model describes the propagation of electrons through a series of potential barriers, like the one shown in Figure \ref{fig:RuCl3ElecStruct-1}.

\begin{centering}
\includegraphics[width=0.5\textwidth]{C:/Users/dsbjr/Documents/GitHub/Dissertation/img/RuCl3ElectronicStructure-KronigPenney.png}
  \captionsetup{width=0.75\textwidth}
  \captionof{figure}[Kronig-Penney Potential]{Square wave potential of height $U_{0}$ with period $a$ and square wave length $b$ (adapted from \cite{Erez2014}).} 
  \label{fig:RuCl3ElecStruct-1}
\end{centering}

We begin by analyzing a simpler system, the propagation of a delocalized electron of energy $E$ encountering a semi-infinite potential barrier of height $U_{0}$ at $x = 0$. If we define two regions, $x <0$ and $x>0$, we can write an ansatz for our time-independent wavefunction

\begin{equation}
\psi(x) =
	\begin{cases}
	A e^{i k_{1} x} + B e^{-i k_{1} x}, & x < 0 \\
	C e^{i k_{2} x} + D e^{-i k_{2} x}, & x > 0
	\end{cases}
\end{equation}
	
Where $A,B$ ($C,D$) are the amplitudes of the forward and reverse propagating waves for $x<0$ ($x>0$), respectively, and $k_{1}$ and $k_{2}$ are the wavevectors in those regions. The wave function must be continuous everywhere, so the value of the wavefunction and its slope must match at $x = 0$, requiring

\begin{equation}
	\begin{array}{cc}
		A + B = C + D, & k_{1}(A - B) = k_{2}(C - D) \\
	\end{array}
\end{equation}

Our ansatz and the boundary conditions define a system of equations that can be expressed using a transfer matrix $\mathbf{T_{0}}$, relating the amplitudes of the incident waves to amplitudes on the other side of the step.

\begin{equation}
	\begin{pmatrix}
		C \\
		D \\
	\end{pmatrix} =
	\mathbf{T_{0}}
	\begin{pmatrix}
		A \\
		B \\
	\end{pmatrix}
\end{equation}

\begin{equation}
\mathbf{T_{0}} = \frac{1}{2 k_{2}}
	\begin{pmatrix}
		k_{2} + k_{1} & k_{2} - k_{1} \\
		k_{2} - k_{1} & k_{2} + k_{1} \\
	\end{pmatrix}
\end{equation}


While this expression for the transfer matrix holds true only for a step at $x=0$, we can easily translate it in $x$ by changing the phase as follows

\begin{equation}
\mathbf{T}(x = d) = 
	\begin{pmatrix}
		e^{-i k_{2} d} & 0 \\
		0 & e^{i k_{2} d}  \\
	\end{pmatrix}
	\mathbf{T_{0}}
	\begin{pmatrix}
		e^{i k_{1} d} & 0 \\
		0 & e^{-i k_{1} d}  \\
	\end{pmatrix}
\end{equation}

Armed with Bloch's Theorem and the transfer matrix formalism, we can make short work of the Kronig-Penney potential. For example, the transfer matrix for the barrier to the immediate right of $x = 0$ is given by

\begin{equation}
\mathbf{T}(x = d) = 
	\begin{pmatrix}
		e^{-i k_{1} a} & 0 \\
		0 & e^{i k_{1} a}  \\
	\end{pmatrix}
	\mathbf{T_{0}}
	\begin{pmatrix}
		e^{i k_{1} a} & 0 \\
		0 & e^{-i k_{1} a}  \\
	\end{pmatrix}
\end{equation}

The overall transfer matrix is

\begin{equation}
\mathbf{T} = ...(\mathbf{A}^{-2}\mathbf{T_{0}} \mathbf{A}^{2})(\mathbf{A}^{-1}\mathbf{T_{0}} \mathbf{A})(\mathbf{T_{0}})(\mathbf{A}^{-1}\mathbf{T_{0}} \mathbf{A})(\mathbf{A}^{-2}\mathbf{T_{0}} \mathbf{A}^{2})...
\end{equation}

\begin{equation*}
= ... \mathbf{A}\mathbf{T_{0}}\mathbf{A}\mathbf{T_{0}}\mathbf{A}\mathbf{T_{0}}...
\end{equation*}

Each unit cell is represented by the product $\mathbf{A}\mathbf{T_{0}}$. Using Bloch's theorem, we can write

\begin{equation}
	\begin{pmatrix}
		a_{n+1} \\
		b_{n+1} \\
	\end{pmatrix} =
	\mathbf{AT_{0}}
	\begin{pmatrix}
		a_{n} \\
		b_{n} \\
	\end{pmatrix} = 
	e^{i k a}
	\begin{pmatrix}
		a_{n} \\
		b_{n} \\
	\end{pmatrix}
\end{equation}

From this equation, we know that $e^{i k a}$ must be an eigenvalue of $\mathbf{AT_{0}}$. For the Kronig-Penney model, we then find

\begin{equation}
\cos{k a} = \cos{k_{1} w} \cos{k_{2} b} - \frac{k_{1}^{2} + k_{2}^{2}}{2 k_{1} k_{2}} \sin{k_{1} w} \sin{k_{2} b}
\end{equation}

where $k_{1}$ is the wavevector between barriers, $k_{2}$ is the wavevector in the barriers, and $k$ is the Bloch wavevumber describing the overall propagation of the electron in the lattice.

Note that the range of cosine is only $[-1, 1]$, so for some values of $k_{1}$ and $k_{2}$ there are no (propagating) plane wave solutions. The regions of k-space with propagating solutions are called bands, and the regions without propagating solutions are called band gaps.

If instead of barriers we have a series of $\delta$-function potentials, the equation for the wavevector becomes simpler

\begin{equation}
\cos{k a} = \cos{k_{1} a} + \left( \frac{m a S}{\hbar^{2}} \right) \frac{\sin{k_{1} a}}{k_{1} a}
\end{equation}

where $m$ is the effective mass of the electron and $S$ is the strength of the $\delta$ function. In Figure \ref{fig:RuCl3ElecStruct-2}, the energy as a function of wavevector is plotted in the extended zone scheme.

\begin{centering}
\includegraphics[width=0.5\textwidth]{C:/Users/dsbjr/Documents/GitHub/Dissertation/img/KronigPenneyExtendedZone-Davies.png}
  \captionsetup{width=0.75\textwidth}
  \captionof{figure}[Dispersion Relation for Kronig-Penney Delta Function Model]{Energy vs. Wavevector for Kronig-Penney model of $\delta$-function potentials separated by distance $a$. Note that band gaps appear at integer multiples of $\frac{\pi}{a}$. From \cite{Davies1997}.} 
  \label{fig:RuCl3ElecStruct-2}
\end{centering}

The band gaps in Figure \ref{fig:RuCl3ElecStruct-2} appear every $k = \frac{n\pi}{a}$. This wavevectors correspond to reciprocal lattice vectors, and any periodic system will necessarily have band gaps at these value of $k$. While these band gaps emerge naturally from the Kronig-Penney model, we can also interpret them in the context of a diffraction. If a wave with wavelength $\lambda$ scatters from a series of periodic planes separated by distance $a$, the reflected waves destructively interfere with the incident wave when $k = \frac{2\pi}{\lambda} = \frac{n \pi}{a}$, preventing the wave from propagating.

While band gaps always appear at zone boundaries, the nature of the periodic potential dictates the properties of the band structure everywhere else. Consider bringing together a large group of carbon atoms into a diamond lattice. Bands arising from the overlap of 2p and 2s carbon orbitals are shown in Figure \ref{fig:RuCl3ElecStruct-3}.  At large interatomic distance the orbitals do not overlap, but as the carbon atoms approach the overlap in their orbitals gives rise to delocalized orbitals so densely packed in energy as to effectively be a continuum. These delocalized orbitals are the bands, which are separated by a forbidden region called the band gap.

\begin{centering}
\includegraphics[width=0.75\textwidth]{C:/Users/dsbjr/Documents/GitHub/Dissertation/img/DiamondBandStructure-Chetvorno.png}
  \captionsetup{width=0.75\textwidth}
  \captionof{figure}[Diamond Band Structure and Interatomic Distance]{Diamond band structure as a function of interatomic separation. At large separation bandwidth corresponds to only the allowed energy of the 2s and 2p orbitals. As the separation is reduced, the overlapping orbitals give rise to bands and band gaps. Image adapted from Chetvorno.} 
  \label{fig:RuCl3ElecStruct-3}
\end{centering}

Bands and band gaps are the electronic structure of a material and determine the material's electronic properties. For example, a material with a band gap between a band where all states are occupied and one with no states occupied is an insulator because energy is required to add or excite an electron into a propagating state. If the band gap in such a material is small, the material is a semiconductor. And if partially filled bands are present, the material is a metal.

Band structure calculations for \rucl predict that it should be a metal, which is clearly incorrect. Band theory fails spectacularly in this case because electrons in \rucl are strongly correlated, and electron-electron interactions cannot be neglected. Accordingly, we will study another model to see how electron-electron interactions can cause materials to be insulating that would otherwise predict to be metallic.

\subsection{Review: The Hubbard Model and Mott Insulators}

To appreciate the effect of electron-electron interactions, we will qualitatively explore a simplified version of the venerable Hubbard model in two dimensions.

\begin{centering}
\includegraphics[width=0.3\textwidth]{C:/Users/dsbjr/Documents/GitHub/Dissertation/img/HubbardModel-Yamada.png}
  \captionsetup{width=0.5\textwidth}
  \captionof{figure}[Hubbard Model in Two Dimensions]{The Hubbard model in two dimensions. Zero, one, or two electrons can occupy each lattice site, with double occupancy costing an energy U. Electrons hop between sites with hopping integral $t$. From \cite{Yamada2018}.}
  \label{fig:RuCl3ElecStruct-5}
\end{centering}

In the Hubbard model, electrons occupy discrete lattice sites (see Figure \ref{fig:RuCl3ElecStruct-5}), rather than propagate in free space. Each lattice site can hold zero, one, or two electrons (which must have opposite spin). The propagation of electrons and the associated kinetic energy is captured by a hopping term in the Hamiltonian, which moves electrons between lattice sites. Interactions are modeled by an on-site repulsion term, which increases the overall energy of the system if two electrons occupy a single site.

The Hamiltonian is given by \cite{Atland2010}

\begin{equation}
\hat{H} = -t \sum_{<ij>} a^{\dagger}_{i \sigma} a_{j \sigma} + U \sum_{i} \hat{n}_{i \uparrow} \hat{n}_{i \downarrow} + \text{H.c.}
\end{equation}

where $t$ is the hopping integral, $a_{i \sigma}$ are the creation and annihilation operators for an electron of spin $\sigma$ at lattice site $i$, $U$ (called the Hubbard U) is the on site repulsion, $\hat{n}_{i\uparrow(\downarrow)}$ counts the electrons with spin up (down) on lattice site $i$, and $<ij>$ indicates the sum should go over nearest neighbors.

Considering the limits of the Hubbard model is an instructive exercise. In the dilute limit (i.e., $\braket{n_{i}} << 1$), electron-electron interactions are weak because there aren't enough electrons around to interact with each other, and the model reproduces band theory \cite{Fazekas1990}:

\begin{equation}
\epsilon(\mathbf{k}) = -2t \sum_{j = x,y} \cos{k_{j} a}
\end{equation}

In the band limit, $U \rightarrow 0$, causing electron-electron interactions to become negligible; again, we recover band theory. In the atomic limit, $t \rightarrow 0$, the electrons can no longer move, and we have the energy spectrum of a single lattice site. Already we can see the Hubbard model produces results that are more in line with reality. As the lattice spacing increases, $t$ decreases due to decreasing overlap of atomic orbitals, until motion between lattice sites is completely suppressed. In band theory, the lattice spacing could approach infinity and there would still be propagating electron modes (check the Kronig-Penney model to see this).

So far, the limiting cases have predicted what we would expect. But what happens at half-filling ($\braket{n_{i}} \approx 1$)\footnote{Theoretically, any deviation from half-filling should result in a metallic state because the empty site or doubly-occupied site should be able to propagate. However, in a real system defects and impurities may prevent propagation and widen the range of filling values that can support insulating states.} when $U$ and $t$ are of similar magnitude? In this case, the model describes competition between the hopping term, which delocalizes the electrons, and the Hubbard U, which localizes them. For small values of $U/t$, the model predicts a metallic system. But for large values of $U/t$, the energetic penalty for double occupancy forces the electrons to stay localized on a particular lattice site and the system is insulating. At a critical value $U^{*} \approx z t$, where $z$ is the coordination number of the lattice, the system undergoes a Mott transition (a continuous phase transition) from an insulating to a metallic state \cite{Fazekas}. We can visualize this transition as the eventual meeting and overlap of occupied and unoccupied Hubbard bands (corresponding to single and double occupancy of the lattice sites) at a the critical value $U^{*}$, as shown in Figure \ref{fig:RuCl3ElecStruct-4}.

\begin{centering}
\includegraphics[width=0.75\textwidth]{C:/Users/dsbjr/Documents/GitHub/Dissertation/img/MottTransition-Fazekas.png}
  \captionsetup{width=0.75\textwidth}
  \captionof{figure}[Mott Transition]{As the on-site repulsion parameter U is reduced below the critical value, the upper and lower Hubbard bands meet and overlap. From \cite{Fazekas1990}.} 
  \label{fig:RuCl3ElecStruct-4}
\end{centering}

As we have seen, electron-electron correlations can dramatically change the qualitative behavior of electronic systems in a ways that band theory cannot capture. Now that we have a qualitative understanding of both band theory and Mott insulators, we can proceed to study the electronic structure of \ruclnospace .

\subsection{Measured Electronic Properties of \texorpdfstring{\rucl}{RuCl3}}

\subsubsection{Transport measurements}

Early transport measurements of single-crystal \rucl (approximately 50 $\mu$m thick) show it to be an insulator with room-temperature in-plane resistivity near $2x10^{3}$ $\Omega$cm and out of plane resistivity approximately three orders of manitude larger. Conduction is thermally activated both in-plane and out-of-plane with an activation energy of approximately 0.1 eV \cite{Binotto1971}. Subsequent Hall measurements using specially-designed high-resistance electronics found an activation energy of approximately 0.15 eV and showed the conduction to be band transport rather than hopping (mobility increases with decreasing temperature). The majority carriers were shown to be electrons \cite{Rojas1983}.

Recently, transport measurements performed on exfoliated, high-purity \rucl crystals (tens of nanometers in thickness) have found somewhat different results. Measurements at California State University, Long Beach measure 0.08 eV for the thermal activation of conduction \cite{Kim2017}. Measurements at the Max Planck Institute in Germany found dramatically lower room-temperature resistivity and variable range hopping instead of band conduction \cite{Mashhadi2018}. 

Taken together, these measurements suggest \rucl is a simple band insulator with a modest band gap of around 100 meV (allowing for some experimental variation). However, other measurements refine and challenge this view.

\subsubsection{Optical measurements}
Optical measurements find an optical band gap of about 1 eV, well in excess of the measured thermal activation energy \cite{Sandilands2016}. Optical absorption at higher energies are identified with transitions between Ru and Cl inner orbitals.

While there are extensive Raman spectroscopy measurements, these measurements relate more to magnetic exicitations or phonons and are covered elsewhere in this dissertation.

\subsubsection{Other measurements and calculations}
Other techniques used to investigate the electronic structure of \rucl include scanning tunneling microscopy and spectroscopy (STM and STS), ange-resolved photoemission spectroscopy (ARPES), and x-ray spectroscopy. STM and STS measurements find a band gap of approximately 0.25 eV and find symmetry breaking of charge distribution on the surface layer of \rucl \cite{Ziatdinov2016}.

ARPES measurements find large gaps (1.2 eV \cite{Zhou2016} and 1.9 eV \cite{Sinn2016}) which are somewhat consistent with optical measurements. However, in order to reproduce to the observed band structure and gaps, electronic structure calculations require both electron-electron (non-zero Hubbard U) and a spin-orbit coupling. Accordingly, these measurements suggest \rucl is a spin-assisted Mott insulator - a Mott insulator that requires spin-orbit coupling to open a gap.


\subsection{A unified electronic picture of \texorpdfstring{\rucl}{RuCl3}}
While there appear to be conflicting measurements in the published literature, many of the salient features of \rucl can be understood in a single framework. First, \rucl is a spin-assisted Mott insulator, in that band structure predicts it to be metallic, but its insulating behaviour is the result of both repulsive electron-electron interactions and spin orbit coupling. Second, differing measurements of the gap are actually measuring different transitions. The 0.1 eV thermal activation for conduction observed in transport measurements comes from a small band approximately 0.1 eV above the Fermi level, which was not seen in ARPES and was hard to see in optical measurements. Other, larger gaps correspond to transitions between bands with larger energy differences as shown in Figure \ref{fig:RuCl3ElecStruct-6} \cite{Plumb2014}.

\begin{centering}
\includegraphics[width=0.9\textwidth]{C:/Users/dsbjr/Documents/GitHub/Dissertation/img/AbsorptionBandStructure-Plumb.png}
  \captionsetup{width=0.75\textwidth}
  \captionof{figure}[Optical Absorption and Calculated Band Structure of \ruclnospace]{Optical absorption and calculated band structure. The features in optical absorption are correlated with transitions between bands. Note the relatively small amplitude of the $\alpha$ feature, which may explain why it was previously unobserved. From \cite{Plumb2014}}. 
  \label{fig:RuCl3ElecStruct-6}
\end{centering}

\section{Magnetic Properties}

The magnetic properties of a material are determined by the pairings and spin interactions of its electrons. Accordingly, we will examine these for \rucl, and see that the magnetic properties are dominated by effective spin-\textonehalf electrons localized on the ruthenium atoms. We will then explore the interactions between these localized electrons and how they lead to a realization of the Kitaev model. Finally, we will discuss recent measurements that confirm the electrons to be localized and the behavior of \rucl as a spin liquid.

\subsection{Effective spin-\textonehalf{} moments in \texorpdfstring{\rucl}{RuCl3}}

Ruthenium has electronic configuration [Kr]4d\textsuperscript{7}5s\textsuperscript{1}. In \ruclnospace , the $s$ electron and one 2 $d$ electrons are paired into bonding orbitals and are therefore we neglect their contribution to magnetism. Under this assumption, the magnetism of the material is determined by the remaining 5 electrons in the 4$d$ orbitals.

Because the ruthenium exists in a solid, the energy of the d orbitals is affected by the surrounding crystal. This interaction, called the crystal field effect, actually splits the 4d orbitals into doubly-degenerate e\textsubscript{g} and triply-degenerate t\textsubscript{2g} orbitals, as shown in Figure \ref{fig:RuCl3MagProp-1}.

\begin{centering}
\includegraphics[width=0.9\textwidth]{C:/Users/dsbjr/Documents/GitHub/Dissertation/img/CrystalFieldSplitting-Stamokostas.png}
  \captionsetup{width=0.75\textwidth}
  \captionof{figure}[Octahedral crystal field splitting of d orbitals]{The presence of octahedral coordination breaks the SO(3) rotational symmetry of the d orbitals and reduces it to the discrete octahedral symmetry $O_{h}$. d orbitals that point directly at ligands become higher in energy than d orbitals that do not by the crystal field splitting energy $\Delta$. From \cite{Stamokostas2018}.} 
  \label{fig:RuCl3MagProp-1}
\end{centering}

Spin-orbit coupling further splits the e\textsubscript{g} and t\textsubscript{2g} orbitals\footnote{This is not \textit{exactly} correct. The spin-orbit coupling in \rucl is sufficiently small that it mixes the e\textsubscript{g} and t\textsubscript{2g} orbitals. However, introduction of the Hubbard U term in the Hamiltonian enhances the spin orbit coupling and returns the $J_eff = \frac{1}{2}$ character of a single t\textsubscript{2g} orbital, making this picture qualitatively correct \cite{Kim2015}.} by $\frac{3\zeta}{2}$ (where $\zeta$ is the spin orbit coupling), as shown in Figure \ref{fig:RuCl3MagProp-2}.

\begin{centering}
\includegraphics[width=0.75\textwidth]{C:/Users/dsbjr/Documents/GitHub/Dissertation/img/SpinOrbitSplitting-Stamokostas.png}
  \captionsetup{width=0.75\textwidth}
  \captionof{figure}[Spin-orbit splitting of t2g orbitals]{Spin orbit coupling further splits the t\textsubscript{2g} orbitals into effective J = \textonehalf{} and J = $\frac{3}{2}$ states. Adapted from \cite{Stamokostas2018}.} 
  \label{fig:RuCl3MagProp-2}
\end{centering}

Four of the remaining $d$ electrons are paired in the  lower-energy $J_{eff} = \frac{3}{2}$ states, while the remaining electron is unpaired and occupies the $J_{eff} = \frac{1}{2}$ state. This electron is responsible for the magnetism of \ruclnospace, and is a localized spin-\textonehalf{} moment. DFT calculations, which accurately reproduce ARPES, x-ray photoemission, and electron energy loss spectroscopy, justifies this $J_{eff} = \frac{1}{2}$ description \cite{Koitzsch2016}.

The problem of interacting spin-\textonehalf{} moments on the vertices of a hexagonal lattice is closely related to the Kitaev model of a quantum spin liquid, which has already been solved. We will now turn our attention to this model.

\subsection{The Kitaev QSL and its Solution}

In 2006, Kitaev developed an exactly solvable model of a QSL \cite{Kitaev2006}. He imagined a system of spins on a hexagonal lattice being described by the following Hamiltonian

\begin{equation}
H = -J_{x} \sum_{\text{x links}} \sigma_{j}^{x} \sigma_{k}^{x} -J_{y} \sum_{\text{y links}} \sigma_{j}^{y} \sigma_{k}^{y} -J_{z} \sum_{\text{z links}} \sigma_{j}^{z} \sigma_{k}^{z} = \sum_{<ij>,\gamma}K_{\gamma}\sigma_{i}^{\gamma}\sigma_{j}^{\gamma}
\end{equation}

where the x, y, and z links are defined in Figure \ref{fig:RuCl3MagProp-3}.

\begin{centering}
\includegraphics[width=0.75\textwidth]{C:/Users/dsbjr/Documents/GitHub/Dissertation/img/KitaevLattice-Kitaev.png}
  \captionsetup{width=0.75\textwidth}
  \captionof{figure}[Lattice for the Kitaev Model]{Definitions of the three types of links on a hexagonal lattice used in the Kitaev Model. From \cite{Kitaev2006}.} 
  \label{fig:RuCl3MagProp-3}
\end{centering}

While the anisotropic interaction may seem unusual, but in hexagonal systems with octahedral coordination, small deviations in bond angle lead to inequivalent bonds that could give rise to anisotropy (see Figure \ref{fig:RuCl3CrystStruct-1}.

Surprisingly, the Hamiltonian commutes with the so-called ``plaquette'' operator

\begin{equation}
W_{p} = \sigma_{1}^{x}\sigma_{2}^{y}\sigma_{3}^{z}\sigma_{4}^{x}\sigma_{5}^{y}\sigma_{6}^{z}
\end{equation}

which is the product around a hexagonal loop of six spin operators with an index matching that of the outgoing bond. The conserved quantity associated with the plaquette operator is a $Z_{2}$ gauge flux\footnote{$Z_{2}$ refers to the operator taking values $\pm 1$. Gauge flux refers to taking the product around a loop being similar to finding the flux through a surface by integrating a gauge field along the edge.} or vison. These gauge fluxes split the Hilbert space into sectors with given values of $W_{p}$ for each plaquette, making the Hamiltonian block diagonal and easier to solve.

Any fermionic mode can be written as the sum or difference of Majorana operators

\begin{equation}
\begin{array}{cc}
	c_{2k-1} = a_{k} + a_{k}^{\dagger}, & c_{2k} = \frac{1}{i} (a_{k} - a_{k}^{\dagger}) \\
\end{array}
\end{equation}

Which have the following algebra

\begin{equation}
\begin{cases}
	c_{j}^{2} = 1 \\
	c_{j} c_{l} = - c_{l} c_{j}, & j \neq l \\
\end{cases}
\end{equation}

Let us represent the creation and annihilation operators for the spins in the Kitaev Hamiltonian as four majorana operators $b^{x}, b^{y}, b^{z}, c$, and expand the two-dimensional Hilbert subspace into a four-dimensional extended Fock space using the following identifications

\begin{equation}
\widetilde{\sigma^{\gamma}} = i b^{\gamma} c
\end{equation}

Writing our Hamiltonian in the extended Fock space we get

\begin{equation}
\widetilde{H} = \sum_{<ij>,\gamma} K_{\gamma} \widetilde{\sigma^{\gamma}_{i}}\widetilde{\sigma^{\gamma}_{j}} = \sum_{<ij>,\gamma} (i b^{\gamma}_{i}c_{i}) (i b^{\gamma}_{i}c_{j}) = \sum_{<ij>,\gamma} -i(ib^{\gamma}_{i}b^{\gamma}_{j})c_{i}c_{j}
\end{equation}

Define the term in parentheses as an operator along link $i,j$ where $\gamma \in x,y,z$ refers to the character of the link

\begin{equation}
\hat{u}_{i,j} = i b_{i}^{\gamma} b_{j}^{\gamma}
\end{equation}

Then we can write the Hamiltonian in a suggestive form

\begin{equation}
\begin{array}{cc}
	\tilde{H} = \frac{i}{4} \sum_{<ij>} \hat{A}_{ij} c_{i} c_{j}, & \hat{A}_{ij} = 2 J^{\gamma}_{ij}\hat{u}_{i,j} \\
\end{array}
\end{equation}

Taking the product of $\hat{u}_{i,j}$ around the boundary of a plaquette is actually equivalent to our plaquette operator $W_{p}$, and it turns out that the ground state of this Hamiltonian has all $W_{p} = 1$ \cite{Lieb1994}. Accordingly, for the ground state, we can replace $\hat{u}_{i,j}$ with +1 to get a final Hamiltonian

\begin{equation}
\tilde{H} = \frac{i}{4} \sum_{<ij>} \hat{A}_{ij} c_{i} c_{j}
\end{equation}

This is the Hamiltonian for free Majorana fermions. These fermions have a Dirac-like dispersion relation

\begin{equation}
E_{k} = \pm \sqrt{\epsilon_{k}^{2} + \delta_{k}^{2}}
\end{equation}

where $\delta_{k}$ is zero for weakly anisotropic $J^{\gamma}$ and non-zero when the $J^{\gamma}$ violate one of the triangle inequalities. Figure \ref{fig:RuCl3MagProp-4} is a visualization of the phase diagram.

\begin{centering}
\includegraphics[width=0.5\textwidth]{C:/Users/dsbjr/Documents/GitHub/Dissertation/img/KitaevPhaseDiagram-Kitaev.png}
  \captionsetup{width=0.75\textwidth}
  \captionof{figure}[Phase Diagram of the Kitaev Model]{Phase Diagram of the Kitaev model, showing a gapless excitation spectrum for weak anisotropy in values of $J_{\gamma}$. From \cite{Kitaev2006}.} 
  \label{fig:RuCl3MagProp-4}
\end{centering}

Majorana fermions are inherently fractional quasiparticles. Accordingly, the low-energy excitations of the Kitaev Hamiltonian reveal the topological nature of this system. Observing (1) the absence of long-range magnetic order down to 0 K and (2) a broad, gapped or gapless, spectrum of magnetic excitations in \rucl would be powerful evidence of its spin liquid character.


\section{RuCl3 as a spin liquid}

Discuss magnetic transitions - paramagnet to antiferromagnet. Stacking faults causing additional magnetic transitions.

Finally, the conduction electrons identified as charge carriers in transport measurement are actually a $J_{\text{eff}} = \frac{1}{2}$ system localized on ruthenium atoms \cite{Plumb2014}.


%review this: Kim, H.-S., & Kee, H.-Y. (2015). Crystal structure and magnetism in alpha-RuCl3: an ab-initio study, 155143, 1–10. https://doi.org/10.1103/PhysRevB.93.155143

%Sears, J. A., Songvilay, M., Plumb, K. W., Clancy, J. P., Qiu, Y., Zhao, Y., … Kim, Y. J. (2015). Magnetic order in α - RuCl3: A honeycomb-lattice quantum magnet with strong spin-orbit coupling. Physical Review B - Condensed Matter and Materials Physics, 91(14). https://doi.org/10.1103/PhysRevB.91.144420


\subsection{RuCl3 as a spin liquid}

Discuss and solve the Kitaev model.

\subsection{Spin Liquid Behavior in \texorpdfstring{\rucl}{RuCl3}}

Measurements that show \rucl is a proximate Kitaev spin liquid: neutron scattering, field-induced disordered states, half-quantized thermal hall conductance, magnetic impurity doping suppressing AF interaction.