\chapter{Properties of \texorpdfstring{$\alpha$-Ruthenium (III) Chloride}{alpha-RuCl3}}
This chapter covers the structural, electronic, and magnetic properties of \rucl and serves as a reference for further discussion of the material. This chapter also contains a discussion of the Kitaev model and how \rucl realizes this model in the lab. Finally, I discuss published measurements of \rucl and how they confirm it as a proximate Kitaev spin liquid.

\section{Crystal Structure and Synthesis}

Generally, ruthenium (III) chloride (\ruclnospace) is a brown or black solid that is a precursor for 	ruthenium chemistry. It is commonly found as a hydrate, where water molecules are incorporated into its structure. High-purity hydrated ruthenium (III) chloride is commercially available from chemical vendors like Sigma Aldrich. We will consider only anhydrous ruthenium (III) chloride (i.e., without water) in this dissertation.

Ruthenium (III) chloride has two polymorphs, called $\alpha$ and $\beta$. Both $\alpha$ and $\beta$-ruthenium (III) chloride are composed of ruthenium atoms octahedrally coordinated by chlorine. However, in the $\alpha$ polymorph the octahedra are edge sharing, while in the $\beta$ polymorph, the octahedra are face-sharing. The edge-sharing octahedra give the $\alpha$ polymorph its interesting magnetic qualities, and therefore, unless explicitly stated otherwise, we will confine our discussion to $\alpha$ ruthenium (III) chloride and refer to it as \ruclnospace .

At room temperature and ambient pressure, the structure of \rucl consists of edge-sharing octahedra of chlorine atoms that coordinate ruthenium. These edge-sharing octahedra form a hexagonal lattice that defines the a-b plane of the material. While interactions between atoms in the a-b plane are ionic bonds\footnote{I describe in-plane bonding as ionic because it concerns bonds between metals and non-metals. However, the modest difference in Pauling electronegativity between ruthenium and chlorine (0.96) suggests the bond has substantial covalent character.}, the only interactions along the c-axis are van der Waals bonds. Accordingly, \rucl is referred to as a quasi-2D, two-dimensional, or van der Waals material. The layers have an ABC stacking order, such that the unit cell contains three repeats of the a-b plane. The layer separation is approximately 5.7 \AA, while the separation between ruthenium and chlorine atoms in the a-b plane is approximately 0.6 \AA. This structure is monoclinic and belongs to space group C2/m. As show in Figure \ref{fig:RuCl3CrystStruct-1}, the angles between Ru-Cl bonds are very nearly 90$^{\circ}$. The nearly ideal coordinated of ruthenium atoms will be important to understanding \rucl as a proximate Kitaev spin liquid.

\begin{centering}
\includegraphics[width=0.5\textwidth]{C:/Users/dsbjr/Documents/GitHub/Dissertation/img/RuCl3CrystalStructure-Plumb.png}
  \captionsetup{width=0.75\textwidth}
  \captionof{figure}[Crystal structure of \ruclnospace]{Crystal structure of \rucl (Figure from \cite{Plumb2014}). (a) View of the b-c plane, showing ABC stacking of the layers. (b) View of the a-b plane showing RuCl$_{6}$ octahedra form a hexagonal lattice. (c) \rucl octahedra showing bond angles very close to 90$^{\circ}$.} 
  \label{fig:RuCl3CrystStruct-1}
\end{centering}

\rucl exhibits a structural phase transition from monoclinic (space group C2/m) to rhombohedral (space group R$\overline{3}$) near 150K \cite{Glamazda2017}. This phase transition shows a large degree of thermal hysteresis and has been identified in both magnetic susceptibility and x-ray diffraction measurements \cite{Park2016}. Electronic transport measurements also claim to show evidence of this transition \cite{Mashhadi2018}.

\rucl may be synthesized by several methods, including by direct reaction of Ru and Cl at elevated temperatures in the presence of carbon monoxide \cite{Binotto1971}, chemical vapor growth \cite{Gronke2018}, and by vapor transport techniques using purified commercial $\beta$-\ruclnospace \cite{Cao2016}. The \rucl crystals used for this work were grown by Stephen Nagler and colleagues at Oak Ridge National Laboratory using the latter method.

\section{Electronic Structure}

\rucl is a terrible conductor of electricity. But this highly insulating nature comes from a rich electronic structure that will be relevant when interpreting electronic transport data. Therefore in this section I will briefly review band theory, Mott insulators, and spin-orbit coupling before showing how these concepts explain the peculiarities of \rucl 's electronic structure.

\subsection{A Review of Band Theory and Correlated Materials}

Matter in the solid state is composed of individual atoms separated by distances so small (typically \AA s or less) that the atomic orbitals interact and combine to form a broader set of molecular orbitals extending over the entire material. In a crystalline material, each unit cell consists of an identical set of atomic potentials and orbitals that repeat to produce larger orbitals. In the limit of an infinitely large material\footnote{In this case, anything macroscopic is effectively infinitely large relative because macroscopic length scales are so much larger than the size of a unit cell in a crystalline material.}, the wavefunction of an electron in the material will experience periodic boundary conditions across the unit cell. Effectively, the wavefunction must be periodic in space with a period that matches the size of the unit cell. This is \textit{Bloch's Theorem} and it can be expressed mathematically as:

\begin{equation}
\psi_{k}(x + a) = e^{ika} \psi_{k}(x)
\end{equation}

where $a$ is the period of the lattice, $k$ is a wavenumber related to the crystal momentum\footnote{The crystal momentum $\hbar k$ is an effective momentum for a particle traveling in a lattice.}, and $\psi_{k}$ is the wavefunction associated with crystal momentum $k$ \cite{Davies1997}.

For reasons that will become clear, not all values of $k$ are permitted. These prohibited values of $k$ lead to gaps in the allowed energy for electrons called band gaps. Band gaps are a natural consequence of electrons in periodic potentials. To understand how band gaps emerge, we will go through a quick analysis of the Kronig-Penney model.

The Kronig-Penney model describes the propagation of electrons through a square wave potential in space, like the one shown in Figure \ref{fig:RuCl3CrystStruct-2}.

\begin{centering}
\includegraphics[width=0.5\textwidth]{C:/Users/dsbjr/Documents/GitHub/Dissertation/img/RuCl3ElectronicStructure-KronigPenney.png}
  \captionsetup{width=0.75\textwidth}
  \captionof{figure}[Kronig-Penney Potential]{Square wave potential of height $U_{0}$ with period $a$ and square wave length $b$ (adapted from \cite{Erez2014}).} 
  \label{fig:RuCl3CrystStruct-2}
\end{centering}





\section{Magnetic Properties}

Discuss magnetic transitions - paramagnet to antiferromagnet. Stacking faults causing additional magnetic transitions.

\section{\texorpdfstring{\rucl}{RuCl3} as a Kitaev Spin Liquid}

\subsection{The Kitaev Model}

Discuss and solve the Kitaev model.

\subsection{Spin Liquid Behavior in \texorpdfstring{\rucl}{RuCl3}}

Measurements that show \rucl is a proximate Kitaev spin liquid: neutron scattering, field-induced disordered states, half-quantized thermal hall conductance, magnetic impurity doping suppressing AF interaction.