\chapter{Properties of \texorpdfstring{$\alpha$-Ruthenium (III) Chloride}{alpha-RuCl3}}
This chapter covers the structural, electronic, and magnetic properties of \rucl and serves as a reference for further discussion of the material. This chapter also contains a discussion of the Kitaev model and how \rucl realizes this model in the lab. Finally, I discuss published measurements of \rucl and how they confirm it as a proximate Kitaev spin liquid.

\section{Crystal Structure and Synthesis}

Generally, ruthenium (III) chloride (\ruclnospace) is a brown or black solid that is a precursor for 	ruthenium chemistry. It is commonly found as a hydrate, where water molecules are incorporated into its structure. High-purity hydrated ruthenium (III) chloride is commercially available from chemical vendors like Sigma Aldrich. We will consider only anhydrous ruthenium (III) chloride (i.e., without water) in this dissertation.

Ruthenium (III) chloride has two polymorphs, called $\alpha$ and $\beta$. Both $\alpha$ and $\beta$-ruthenium (III) chloride are composed of ruthenium atoms octahedrally coordinated by chlorine. However, in the $\alpha$ polymorph the octahedra are edge sharing, while in the $\beta$ polymorph, the octahedra are face-sharing. The edge-sharing octahedra give the $\alpha$ polymorph its interesting magnetic qualities, and therefore, unless explicitly stated otherwise, we will confine our discussion to $\alpha$ ruthenium (III) chloride and refer to it as \ruclnospace .

At room temperature and ambient pressure, the structure of \rucl consists of edge-sharing octahedra of chlorine atoms that coordinate ruthenium. These edge-sharing octahedra form a hexagonal lattice that defines the a-b plane of the material. While interactions between atoms in the a-b plane are ionic bonds\footnote{I describe in-plane bonding as ionic because it concerns bonds between metals and non-metals. However, the modest difference in Pauling electronegativity between ruthenium and chlorine (0.96) suggests the bond has substantial covalent character.}, the only interactions along the c-axis are van der Waals bonds. Accordingly, \rucl is referred to as a quasi-2D, two-dimensional, or van der Waals material. The layers have an ABC stacking order, such that the unit cell contains three repeats of the a-b plane. The layer separation is approximately 5.7 \AA, while the separation between ruthenium and chlorine atoms in the a-b plane is approximately 0.6 \AA. This structure is monoclinic and belongs to space group C2/m. As show in Figure \ref{fig:RuCl3CrystStruct-1}, the angles between Ru-Cl bonds are very nearly 90$^{\circ}$. The nearly ideal coordinated of ruthenium atoms will be important to understanding \rucl as a proximate Kitaev spin liquid.

\begin{centering}
\includegraphics[width=0.5\textwidth]{C:/Users/dsbjr/Documents/GitHub/Dissertation/img/RuCl3CrystalStructure-Plumb.png}
  \captionsetup{width=0.75\textwidth}
  \captionof{figure}[Crystal structure of \ruclnospace]{Crystal structure of \rucl (Figure from \cite{Plumb2014}). (a) View of the b-c plane, showing ABC stacking of the layers. (b) View of the a-b plane showing RuCl$_{6}$ octahedra form a hexagonal lattice. (c) \rucl octahedra showing bond angles very close to 90$^{\circ}$.} 
  \label{fig:RuCl3CrystStruct-1}
\end{centering}

\rucl exhibits a structural phase transition from monoclinic (space group C2/m) to rhombohedral (space group R$\overline{3}$) near 150K \cite{Glamazda2017}. This phase transition shows a large degree of thermal hysteresis and has been identified in both magnetic susceptibility and x-ray diffraction measurements \cite{Park2016}. Electronic transport measurements also claim to show evidence of this transition \cite{Mashhadi2018}.

\rucl may be synthesized by several methods, including by direct reaction of Ru and Cl at elevated temperatures in the presence of carbon monoxide \cite{Binotto1971}, chemical vapor growth \cite{Gronke2018}, and by vapor transport techniques using purified commercial $\beta$-\ruclnospace \cite{Cao2016}. The \rucl crystals used for this work were grown by Stephen Nagler and colleagues at Oak Ridge National Laboratory using the latter method.

\section{Electronic Structure}

\rucl is a terrible conductor of electricity. But this highly insulating nature comes from a rich electronic structure that will be relevant when interpreting electronic transport data. Therefore in this section I will briefly review band theory, Mott insulators, and spin-orbit coupling before showing how these concepts explain the peculiarities of \ruclnospace 's electronic structure.

\subsection{Review: Band Theory}

Matter in the solid state is composed of individual atoms separated by distances so small (typically \AA s or less) that the atomic orbitals interact and combine to form a broader set of molecular orbitals extending over the entire material. These nature of these broader orbitals defines the range of energies an electron in the solid may have, called bands, and the range of energies an electron may not have, called band gaps. Band theory calculates the bands and band gaps of a material from the properties of allowed non-interacting, single electron electron wavefunctions in a periodic potential. Although based on assumptions that are not always applicable, band theory successfully explains many properties of solid state systems, including electrical resistivity and optical absorption. Before examining the electronic structure of \ruclnospace , we will briefly review band theory to understand some general aspects of band structure.

In a crystalline material, each unit cell consists of an identical set of atomic potentials and orbitals that repeat to produce larger orbitals. In the thermodynamic limit, the wavefunction of an electron in the material will experience periodic boundary conditions across the unit cell. Effectively, the wavefunction must be periodic in space with a period that matches the size of the unit cell. This is \textit{Bloch's Theorem} and it can be expressed mathematically as:

\begin{equation}
\psi_{k}(x + a) = e^{ika} \psi_{k}(x)
\end{equation}

where $a$ is the period of the lattice, $k$ is a wavenumber related to the crystal momentum\footnote{The crystal momentum $\hbar k$ is an effective momentum for a particle traveling in a lattice.}, and $\psi_{k}$ is the wavefunction associated with crystal momentum $k$ \cite{Davies1997}.

For reasons that will become clear, not all values of $k$ are permitted. These prohibited values of $k$ are the previously discussed band gaps, and are a natural consequence of electrons in periodic potentials. To understand how band gaps emerge, we will go through a quick analysis of the Kronig-Penney model, loosely following \cite{Davies1997}.

The Kronig-Penney model describes the propagation of electrons through a series of potential barriers, like the one shown in Figure \ref{fig:RuCl3ElecStruct-1}.

\begin{centering}
\includegraphics[width=0.5\textwidth]{C:/Users/dsbjr/Documents/GitHub/Dissertation/img/RuCl3ElectronicStructure-KronigPenney.png}
  \captionsetup{width=0.75\textwidth}
  \captionof{figure}[Kronig-Penney Potential]{Square wave potential of height $U_{0}$ with period $a$ and square wave length $b$ (adapted from \cite{Erez2014}).} 
  \label{fig:RuCl3ElecStruct-1}
\end{centering}

We begin by analyzing a simpler system, the propagation of a delocalized electron of energy $E$ encountering a semi-infinite potential barrier of height $U_{0}$ at $x = 0$. If we define two regions, $x <0$ and $x>0$, we can write an ansatz for our time-independent wavefunction

\begin{equation}
\psi(x) =
	\begin{cases}
	A e^{i k_{1} x} + B e^{-i k_{1} x}, & x < 0 \\
	C e^{i k_{2} x} + D e^{-i k_{2} x}, & x > 0
	\end{cases}
\end{equation}
	
Where $A,B$ ($C,D$) are the amplitudes of the forward and reverse propagating waves for $x<0$ ($x>0$), respectively, and $k_{1}$ and $k_{2}$ are the wavevectors in those regions. The wave function must be continuous everywhere, so the value of the wavefunction and its slope must match at $x = 0$, requiring

\begin{equation}
	\begin{array}{cc}
		A + B = C + D, & k_{1}(A - B) = k_{2}(C - D) \\
	\end{array}
\end{equation}

Our ansatz and the boundary conditions define a system of equations that can be expressed using a transfer matrix $\mathbf{T_{0}}$, relating the amplitudes of the incident waves to amplitudes on the other side of the step.

\begin{equation}
	\begin{pmatrix}
		C \\
		D \\
	\end{pmatrix} =
	\mathbf{T_{0}}
	\begin{pmatrix}
		A \\
		B \\
	\end{pmatrix}
\end{equation}

\begin{equation}
\mathbf{T_{0}} = \frac{1}{2 k_{2}}
	\begin{pmatrix}
		k_{2} + k_{1} & k_{2} - k_{1} \\
		k_{2} - k_{1} & k_{2} + k_{1} \\
	\end{pmatrix}
\end{equation}


While this expression for the transfer matrix holds true only for a step at $x=0$, we can easily translate it in $x$ by changing the phase as follows

\begin{equation}
\mathbf{T}(x = d) = 
	\begin{pmatrix}
		e^{-i k_{2} d} & 0 \\
		0 & e^{i k_{2} d}  \\
	\end{pmatrix}
	\mathbf{T_{0}}
	\begin{pmatrix}
		e^{i k_{1} d} & 0 \\
		0 & e^{-i k_{1} d}  \\
	\end{pmatrix}
\end{equation}

Armed with Bloch's Theorem and the transfer matrix formalism, we can make short work of the Kronig-Penney potential. For example, the transfer matrix for the barrier to the immediate right of $x = 0$ is given by

\begin{equation}
\mathbf{T}(x = d) = 
	\begin{pmatrix}
		e^{-i k_{1} a} & 0 \\
		0 & e^{i k_{1} a}  \\
	\end{pmatrix}
	\mathbf{T_{0}}
	\begin{pmatrix}
		e^{i k_{1} a} & 0 \\
		0 & e^{-i k_{1} a}  \\
	\end{pmatrix}
\end{equation}

The overall transfer matrix is

\begin{equation}
\mathbf{T} = ...(\mathbf{A}^{-2}\mathbf{T_{0}} \mathbf{A}^{2})(\mathbf{A}^{-1}\mathbf{T_{0}} \mathbf{A})(\mathbf{T_{0}})(\mathbf{A}^{-1}\mathbf{T_{0}} \mathbf{A})(\mathbf{A}^{-2}\mathbf{T_{0}} \mathbf{A}^{2})...
\end{equation}

\begin{equation*}
= ... \mathbf{A}\mathbf{T_{0}}\mathbf{A}\mathbf{T_{0}}\mathbf{A}\mathbf{T_{0}}...
\end{equation*}

Each unit cell is represented by the product $\mathbf{A}\mathbf{T_{0}}$. Using Bloch's theorem, we can write

\begin{equation}
	\begin{pmatrix}
		a_{n+1} \\
		b_{n+1} \\
	\end{pmatrix} =
	\mathbf{AT_{0}}
	\begin{pmatrix}
		a_{n} \\
		b_{n} \\
	\end{pmatrix} = 
	e^{i k a}
	\begin{pmatrix}
		a_{n} \\
		b_{n} \\
	\end{pmatrix}
\end{equation}

From this equation, we know that $e^{i k a}$ must be an eigenvalue of $\mathbf{AT_{0}}$. For the Kronig-Penney model, we then find

\begin{equation}
\cos{k a} = \cos{k_{1} w} \cos{k_{2} b} - \frac{k_{1}^{2} + k_{2}^{2}}{2 k_{1} k_{2}} \sin{k_{1} w} \sin{k_{2} b}
\end{equation}

where $k_{1}$ is the wavevector between barriers, $k_{2}$ is the wavevector in the barriers, and $k$ is the Bloch wavevumber describing the overall propagation of the electron in the lattice.

Note that the range of cosine is only $[-1, 1]$, so for some values of $k_{1}$ and $k_{2}$ there are no (propagating) plane wave solutions. The regions of k-space with propagating solutions are called bands, and the regions without propagating solutions are called band gaps.

If instead of barriers we have a series of $\delta$-function potentials, the equation for the wavevector becomes simpler

\begin{equation}
\cos{k a} = \cos{k_{1} a} + \left( \frac{m a S}{\hbar^{2}} \right) \frac{\sin{k_{1} a}}{k_{1} a}
\end{equation}

where $m$ is the effective mass of the electron and $S$ is the strength of the $\delta$ function. In Figure \ref{fig:RuCl3ElecStruct-2}, the energy as a function of wavevector is plotted in the extended zone scheme.

\begin{centering}
\includegraphics[width=0.5\textwidth]{C:/Users/dsbjr/Documents/GitHub/Dissertation/img/KronigPenneyExtendedZone-Davies.png}
  \captionsetup{width=0.75\textwidth}
  \captionof{figure}[Dispersion Relation for Kronig-Penney Delta Function Model]{Energy vs. Wavevector for Kronig-Penney model of $\delta$-function potentials separated by distance $a$. Note that band gaps appear at integer multiples of $\frac{\pi}{a}$. From \cite{Davies1997}.} 
  \label{fig:RuCl3ElecStruct-2}
\end{centering}

The band gaps in Figure \ref{fig:RuCl3ElecStruct-2} appear every $k = \frac{n\pi}{a}$. This wavevectors correspond to reciprocal lattice vectors, and any periodic system will necessarily have band gaps at these value of $k$. While these band gaps emerge naturally from the Kronig-Penney model, we can also interpret them in the context of a diffraction. If a wave with wavelength $\lambda$ scatters from a series of periodic planes separated by distance $a$, the reflected waves destructively interfere with the incident wave when $k = \frac{2\pi}{\lambda} = \frac{n \pi}{a}$, preventing the wave from propagating.

While band gaps always appear at zone boundaries, the nature of the periodic potential dictates the properties of the band structure everywhere else. Consider bringing together a large group of carbon atoms into a diamond lattice. Bands arising from the overlap of 2p and 2s carbon orbitals are shown in figure XX.  At large interatomic distance the orbitals do not overlap, but as the carbon atoms approach the overlap in their orbitals gives rise to delocalized orbitals so densely packed in energy as to effectively be a continuum. These delocalized orbitals are the bands, which are separated by a forbidden region called the band gap.

\begin{centering}
\includegraphics[width=0.75\textwidth]{C:/Users/dsbjr/Documents/GitHub/Dissertation/img/DiamondBandStructure-Chetvorno.png}
  \captionsetup{width=0.75\textwidth}
  \captionof{figure}[Diamond Band Structure and Interatomic Distance]{Diamond band structure as a function of interatomic separation. At large separation bandwidth corresponds to only the allowed energy of the 2s and 2p orbitals. As the separation is reduced, the overlapping orbitals give rise to bands and band gaps. Image adapted from Chetvorno.} 
  \label{fig:RuCl3ElecStruct-3}
\end{centering}

Bands and band gaps are the electronic structure of a material and determine the material's electronic properties. For example, a material with a band gap between a band where all states are occupied and one with no states occupied is an insulator because energy is required to add or excite an electron into a propagating state. If the band gap in such a material is small, the material is a semiconductor. And if partially filled bands are present, the material is a metal.

Band structure calculations for \rucl predict that it should be a metal, which is clearly incorrect. Band theory fails spectacularly in this case because electrons in \rucl are strongly correlated, and electron-electron interactions cannot be neglected. Accordingly, we will study another model to see how electron-electron interactions can cause materials to be insulating that would otherwise predict to be metallic.

\subsection{Review: Correlated Materials and Mott Insulators}

\subsection{\texorpdfstring{\rucl}{RuCl3}as a spin-assisted Mott Insulator}


\section{Magnetic Properties}

Discuss magnetic transitions - paramagnet to antiferromagnet. Stacking faults causing additional magnetic transitions.

\section{\texorpdfstring{\rucl}{RuCl3}as a Kitaev Spin Liquid}

\subsection{The Kitaev Model}

Discuss and solve the Kitaev model.

\subsection{Spin Liquid Behavior in \texorpdfstring{\rucl}{RuCl3}}

Measurements that show \rucl is a proximate Kitaev spin liquid: neutron scattering, field-induced disordered states, half-quantized thermal hall conductance, magnetic impurity doping suppressing AF interaction.