\chapter{Other work}

This section is a summary of other projects I was involved in during my time at Stanford.

\section{Scanning tunneling microscopy at NIST}

As part of my National Physical Sciences Consortium fellowship, I spent summer 2015 at the National Institute of Standards and Technology under Dr. Joseph Stroscio, working on a project to image fractional quantum hall edge states in graphene. I was unsuccessful in making this measurement. As of the writing of this dissertation, Dr. Stroscio and his colleagues are working on measuring a device produced by Cory Dean's group at Columbia.

\section{Graphene quantum hall in pulsed fields}

\subsubsection{Introduction}
Graphene, an atomically thin layer of carbon atoms, has a variety of properties that make it useful for fundamental research and electronics applications, including massless charge carriers and high charge carrier mobility \cite{CastroNeto2009}. In low magnetic fields and at low charge carrier density, quasiparticles of the graphene electronic ground state are abelian composite fermions \cite{Jain1989}. However, at high magnetic fields and carrier densities, quasiparticles of the electronic ground state are predicted to obey non-abelian statistics \cite{Papic2011}, and may be of interest for topological quantum computation \cite{Nayak2008}. Because currently available DC magnetic fields cannot provide the required magnetic field strength to investigate non-abelian ground states in graphene, pulsed magnetic fields must be used to explore this phenomenon. 

\subsubsection{Experimental}
We fabricated Hall bars from high-mobility graphene-boron nitride heterostructures on 300 nm SiO2/n++ Si substrates in accordance with \cite{Lee2016}. We performed AC and DC transport measurements on these devices in a 3He cryostat in a 65T multi-shot pulsed magnet at the National High Magnetic Field Laboratory, Los Alamos National Laboratory. 

\subsubsection{Results and Discussion}
Normally, when performing Hall measurements, one would use an AC lock-in measurement to achieve lower noise. However, due to the short pulse duration, the required frequencies would be such that capacitive coupling between lines becomes significant and obscures any quantum Hall plateaus. Accordingly, we developed a measurement scheme using DC drive that combines four separate magnet shots to produce a single set of Hall traces; Hall plateaus that did not appear in the AC measurement or a single DC shot measurement are clearly visible in this improved DC measurement. Figure \ref{fig:GraphenePulsed-1} below presents DC Hall measurements with Hall plateaus identified by filling factor.

\begin{centering}
\includegraphics[width=0.5\textwidth]{./img/GraphenePulsedField-original.png}
  \captionsetup{width=0.75\textwidth}
  \captionof{figure}[Graphene quantum Hall in pulsed field]{DC Hall measurement of encapsulated graphene heterostructure made at NHMFL-LANL.} 
  \label{fig:GraphenePulsed-1}
\end{centering}
	
\subsubsection{Conclusions}
 The DC measurement scheme and measurement process developed during this run provide the capability to perform clean quantum Hall measurements in encapsulated graphene heterostructures in large pulsed magnetic fields that may be potentially useful in future experiments. Further measurements may allow for the observation of fractional quantum Hall states at high field or non-abelian ground states.

\section{Phonon drag in AlGaN/GaN two-dimensional electron gases for high thermoelectric performance}
I provided Hall measurements to the Senesky group at Stanford to support their measurements of phonon drag in galium nitride heterostructures, discussed here \cite{Yalamarthy2019}.

\section{Terahertz spectroscopy of \rucl }
I provided transport measurements of bulk \rucl to support terahertz spectroscopy measurements in the Orenstein group at Berkeley, discussed here \cite{Liang2018}.

\end{appendices}