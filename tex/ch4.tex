\chapter{Electronic Transport}
This chapter describes original measurements of the resistivity of bulk and exfoliated single-crystal \rucl as a function of temperature and electrolyte gate voltage. Bulk measurements are found to correspond with published literature, while measurements of the exfoliated flakes suggest an anomalously high energy density of states at the Fermi level and show substantial hysteresis in resistance when sweeping the electrolyte gate voltage. I conclude by discussing some explanations for the unusual behavior of the exfoliated flakes.

\section{Introduction to Electronic Transport Measurements}

Electronic transport measurements characterize the movement of charge through a material by looking for changes in resistivity as a function of some tuned parameter, like temperature, carrier density, or magnetic field. Because the resistivity is a empirical consequence of both the electronic structure of and scattering processes in a material, changes in resistivity as a function of the tuned parameter provide valuable information about these aspects.

We can learn about the electronic structure of a material by measuring the thermal activation energy of conduction. Consider a material with its electrochemical potential in the band gap. Because there are no partially occupied bands, the material is an insulator. However, insulators do conduct electricity - in order to do so, a carrier must be thermally excited to the bottom of the conduction band\footnote{Put plainly, a phonon scatters from an electron, increasing its energy so that it now occupies a state in the conduction band.}. From Drude theory \cite{Ashcroft1976}, we know that $\sigma \propto n$, where $n$ is the density of thermally excited electrons. We can write the probability of exciting an electron as a Boltzmann factor:

\begin{equation}
\mathbb{P}[\text{exc.}] \propto e^{-\frac{\Delta}{kT}}
\end{equation}

where $k$ is the Boltzmann constant and $\Delta = E_{c} - \mu$. Using the proportionality,

\begin{equation}
\sigma = A e^{-\frac{\Delta}{kT}}
\end{equation}

which is the Arrhenius relation. The value of $\Delta$ extracted from a resistance-temperature curve is the separation in energy space between the chemical potential and the bottom of the conduction band. Changes in $\Delta$ as a function of carrier density contain information about the electronic density of states.


\section{\texorpdfstring{\rucl}{RuCl3} Bulk Crystals}

Initial measurements of Bulk \ruclnospace . Also Weary Heart.

\section{Exfoliated \texorpdfstring{\rucl}{RuCl3} Crystals}

Use data from Rookle02, Rookle03, Rookle09 (I think the contacts were good for these samples) and then RuCl3TransportSample006