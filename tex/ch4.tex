\chapter{Electronic Transport}
This chapter describes original measurements of the resistivity of bulk and exfoliated single-crystal \rucl as a function of temperature and electrolyte gate voltage. Bulk measurements are found to correspond with published literature, while measurements of the exfoliated flakes suggest an anomalously high energy density of states at the Fermi level and show substantial hysteresis in resistance when sweeping the electrolyte gate voltage. I conclude by discussing some explanations for the unusual behavior of the exfoliated flakes.

\section{Introduction to Electronic Transport Measurements}

Electronic transport measurements characterize the movement of charge through a material by looking for changes in resistivity as a function of some tuned parameter, like temperature, carrier density, or magnetic field. Because the resistivity is a empirical consequence of both the electronic structure of and scattering processes in a material, changes in resistivity as a function of the tuned parameter provide valuable information about these aspects.

We can learn about the electronic structure of a material by measuring the thermal activation energy of conduction. Consider a material with its electrochemical potential in the band gap. Because there are no partially occupied bands, the material is an insulator. However, insulators do conduct electricity - in order to do so, a carrier must be thermally excited to the bottom of the conduction band\footnote{Put plainly, a phonon scatters from an electron, increasing its energy so that it now occupies a state in the conduction band.}. From Drude theory \cite{Ashcroft1976}, we know that $\sigma \propto n$, where $n$ is the density of thermally excited electrons. We can write the probability of exciting an electron as a Boltzmann factor:

\begin{equation}
\mathbb{P}[\text{exc.}] \propto e^{-\frac{\Delta}{kT}}
\end{equation}

where $k$ is the Boltzmann constant and $\Delta = E_{c} - \mu$. Using the proportionality,

\begin{equation}
\sigma = A e^{-\frac{\Delta}{kT}}
\end{equation}

which is the Arrhenius relation. The value of $\Delta$ extracted from a resistance-temperature curve is the separation in energy space between the chemical potential and the bottom of the conduction band. Changes in $\Delta$ as a function of carrier density contain information about the electronic density of states.


\section{Measurements of \texorpdfstring{\rucl}{RuCl3}bulk crystals}

\subsubsection{Methods}
Single crystals of bulk \rucl were separated into thin layers approximately 3 mm x 3 mm x 100 $\mu$m using sharp tweezers. Individual layers were mounted on a 5 mm x 5 mm x 500 $\mu$m chip of 300 nm SiO\textsubscript{2} on n\textsuperscript{++} Si using a small amount of H20E Epotek conductive epoxy at each of four corners. Electrical contact to the chip carrier was made using gold wirebonding wire and the conductive epoxy mounts. The epoxy was cured at 100\degree C for 30 minutes. The sample was measured in vacuum in a custom cryostat using a DC current bias from a Keithley 2400 source measurement unit. Temperature was measured using an on-chip thermometer. Sheet resistance, and subsequently resistivity, were caj	lculated using 8 different van der Pauw current and voltage measurements to remove any geometric effects.

Hall measurements were not reproducible between magnetic field sweeps because of the small Hall angle $\theta = \tan^{-1} \left( \frac{\sigma_{xy}}{\sigma_{xx}} \right)$ and are therefore not reported.

\subsubsection{Summary of results}

\begin{centering}
\includegraphics[width=0.75\textwidth]{./img/RuCl3BulkArrheniusPlot-Original.png}
  \captionsetup{width=0.75\textwidth}
  \captionof{figure}[Bulk \rucl Arrhenius Plot]{a. Arrhenius plot for two separate samples of single-crystal bulk \ruclnospace . The thermal activation energy is different between samples, but roughly consistent with published literature b. Picture of a bulk \rucl transport sample.} 
  \label{fig:ElecTransBulk-01}
\end{centering}

Figure \ref{fig:ElecTransBulk-01} presents resistance-temperature curves for two bulk \rucl samples. The thermal activation energy, while different between the samples, remains roughly consistent with published literature. The values of resistivity near room temperature ($\rho \approx 10^{3}$ $\Omega$-cm) are also consistent with published literature.

While Hall measurements are not available for these samples, the consistency with published results for resistivity and thermal activation energy suggests these samples are similar to previously measured samples. Most likely, these samples of \rucl show band conduction with majority electron carriers.

\section{Measurements of exfoliated \texorpdfstring{\rucl}{RuCl3}crystals}

Use data from Rookle02, Rookle03, Rookle09 (I think the contacts were good for these samples) and then RuCl3TransportSample006