\chapter{X-Ray Diffraction}

This chapter covers x-ray diffraction measurements of pristine and electrolyte gated \ruclnospace . I provide an introduction to the technique and cover previous x-ray diffraction measurements of \ruclnospace , then discuss the changes in x-ray diffraction patterns caused by electrolyte gating. I find that the separation between monolayers of \rucl does not change as a function of electrolyte gate voltage, ruling out ion intercalation and limiting electrochemistry to the outer layer of an exfoliated flake.

\section{Introduction}

X-ray diffraction uses coherent scattering of x-rays from a crystal\footnote{A crystal is a material whose atoms occupy points generated by a set of discrete translations operations given by a closed set of lattice vectors. More simply, a crystal is a solid made of a regular array of atoms.} to determine its structure. Because x-rays have wavelengths that approximately match the separation between atoms in solids, a crystal can serve as a diffraction grating for x-ray radiation. The resulting diffraction pattern can be used to calculate the positions of atoms in the crystal.

Consider a two-dimensional square lattice of points having a separation $d$ of order \AA . X-rays incident on this lattice with wavevector $\mathbf{k_{i}}$ will scatter with wavevector $\mathbf{k_{f}}$. Considering only elastic scattering (the vast majority of scattering events), $|\mathbf{k_{i}}| = |\mathbf{k_{f}}|$. The intensity of the scattered radiation will be highest when the difference in path length for each wave is an integer multiple of the wavelength, or when the scattered waves all have the same phase. The intensity is maximum when the incident angle ($\theta$), wavelength ($\lambda$), and separation $d$ satisfy the Bragg condition:

\begin{centering}
\includegraphics[width=0.5\textwidth]{./img/BraggCondition-Wiki.png}
  \captionsetup{width=0.75\textwidth}
  \captionof{figure}[X-ray scattering from a lattice]{X-ray scattering from a two-dimensional lattice. The Bragg condition is satisfied when twice the bolded distance ($d \sin \theta$) is an integer multiple of the wavelength.} 
  \label{fig:XrayIntro-1}
\end{centering}

\begin{equation}
2d \sin \theta = n \lambda
\end{equation}

The example in Figure

\section{Measurements of \rucl}

\subsubsection{Methods}

\subsection{Pristine \rucl}

\subsection{Electrolyte-gated \rucl}

\subsection{Discussion}


Possible references:
%Kim, H.-S., & Kee, H.-Y. (2015). Crystal structure and magnetism in alpha-RuCl3: an ab-initio study, 155143, 1–10. https://doi.org/10.1103/PhysRevB.93.155143
Discusses different crystal structures

%Johnson, R. D., Williams, S. C., Haghighirad, A. A., Singleton, J., Zapf, V., Manuel, P., … Coldea, R. (2015). Monoclinic crystal structure of α-RuCl3 and the zigzag antiferromagnetic ground state. Physical Review B - Condensed Matter and Materials Physics, 92(23). https://doi.org/10.1103/PhysRevB.92.235119
Talks about stacking faults