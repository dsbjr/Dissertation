\chapter{X-Ray Diffraction}

This chapter covers x-ray diffraction measurements of pristine and electrolyte gated \ruclnospace . I provide an introduction to the technique and cover previous x-ray diffraction measurements of \ruclnospace , then discuss the changes in x-ray diffraction patterns caused by electrolyte gating. I find that the separation between monolayers of \rucl does not change as a function of electrolyte gate voltage, ruling out ion intercalation and limiting electrochemistry to the outer layer of an exfoliated flake.

\section{Introduction}

X-ray diffraction uses coherent scattering of x-rays from a crystal\footnote{A crystal is a material whose atoms occupy points generated by a set of discrete translations operations given by a closed set of lattice vectors. More simply, a crystal is a solid made of a regular array of atoms.} to determine its structure. Because x-rays have wavelengths that approximately match the separation between atoms in solids, a crystal can serve as a diffraction grating for x-ray radiation. The resulting diffraction pattern can be used to calculate the positions of atoms in the crystal.

Consider a two-dimensional square lattice of points having a separation $d$ of order \AA . X-rays incident on this lattice with wavevector $\mathbf{k_{i}}$ will scatter with wavevector $\mathbf{k_{f}}$. Considering only elastic scattering (the vast majority of scattering events), $|\mathbf{k_{i}}| = |\mathbf{k_{f}}|$. The intensity of the scattered radiation will be highest when the difference in path length for each wave is an integer multiple of the wavelength, or when the scattered waves all have the same phase. The intensity is maximum when the incident angle ($\theta$), wavelength ($\lambda$), and separation $d$ satisfy the Bragg condition:

\begin{centering}
\includegraphics[width=0.5\textwidth]{./img/BraggCondition-Wiki.png}
  \captionsetup{width=0.75\textwidth}
  \captionof{figure}[X-ray scattering from a lattice]{X-ray scattering from a two-dimensional lattice. The Bragg condition is satisfied when twice the bolded distance ($d \sin \theta$) is an integer multiple of the wavelength.} 
  \label{fig:XrayIntro-1}
\end{centering}

\begin{equation}
2d \sin \theta = n \lambda
\end{equation}

The equation above can be used to calculate $d$ given incident x-rays of a known wavelength and the scattering angle $\theta$. The diffraction pattern for the lattice in Figure \ref{fig:XrayIntro-1} will show diffraction peaks not just at $\theta = \sin^{-1} \left( \frac{n \lambda}{2d} \right)$, but also at $\theta = \sin^{-1} \left( \frac{n \lambda}{2d\sqrt{2}} \right)$ and others. These ``extra'' peaks appear because the $d$ in the Bragg condition does not necessarily correspond to the distance between atoms, but rather to the distance between scattering planes.

\begin{centering}
\includegraphics[width=0.5\textwidth]{./img/BraggDiffraction-Cullity.png}
  \captionsetup{width=0.75\textwidth}
  \captionof{figure}[2D scattering planes labeled with Miller indices]{Bragg diffraction planes in a 2D square lattice. Each set of planes corresponds to a different spacing and will result in a different family of peaks. Scattering planes are labeled by their Miller indices. From \cite{Cullity2014}.} 
  \label{fig:XrayIntro-2}
\end{centering}

Scattering planes are labeled by Miller indices, which can be defined in various ways. Given a set of lattice vectors $\mathbf{a_{1}},\mathbf{a_{2}},\mathbf{a_{3}}$, define reciprocal lattice vectors $\mathbf{b_{i}}$ such that $\mathbf{a_{i}} \cdot \mathbf{b_{j}} = \delta_{ij}$. Then the plane corresponding to Miller indices $(hkl)$ is the plane perpendicular to the vector

\begin{equation}
\mathbf{g_{hkl}} = h \mathbf{b_{1}} + k \mathbf{b_{2}} + l \mathbf{b_{3}}
\end{equation}

and the separation between planes $d_{hkl} = \frac{1}{|\mathbf{g_{hkl}}|}$. Alternatively, we can think of Miller indices as the inverse of fractional intercepts in the unit cell, as in Figure \ref{fig:XrayIntro-3}.

\begin{centering}
\includegraphics[width=0.5\textwidth]{./img/MillerIndices-Cullity.png}
  \captionsetup{width=0.75\textwidth}
  \captionof{figure}[Unit cell with scattering planes labeled with Miller indices]{Unit cell with scattering planes labeled by Miller indices. Each plane intersects the x, y, and z axes at 1/$n$th of the unit cell dimension, where $n$ is the Miller index for the appropriate axis. From \cite{Cullity2014}.} 
  \label{fig:XrayIntro-3}
\end{centering}

To understand this interpretation of scattering planes, consider the (200) plane is Figure \ref{fig:XrayIntro-3}. Because the index $h = 2$, this plane intersects the $\mathbf{a}$ axis at $\frac{1}{2}$ of its extent along the $\mathbf{a}$ direction. $k,l = 0$, so there are no intercepts along the $\mathbf{b}$ and $\mathbf{c}$ axes. The same logic applies for the remaining scattering planes, with the caveat that a bar over an index (like in $(\bar{1} 1 0)$) means that the index is negative. The scattering intensity from planes with low Miller indices is typically higher than from those planes with high Miller indices. This difference results from low index planes having more atoms and therefore a higher electron density with which to scatter x-rays.

To this point, we have used an intuitive and physical approach to understand how x-rays and crystals interact. But underlying our this discussion is a mathematically rigorous theory of x-ray scattering. If we define a scattering wavevector

\begin{equation}
\mathbf{q} = \mathbf{k_{f} - k_{i}}
\end{equation}

then we can write the amplitude of the scattered x-rays as the sum of all the scattered waves weighted by their phase

\begin{equation}
F(\mathbf{q}) = \int \rho_{e}(\mathbf{r}) e^{i \mathbf{q} \cdot \mathbf{r}} d\mathbf{r}
\end{equation}

where $\rho_{e}(\mathbf{r})$ is the electron density. $F(\mathbf{q})$ is exactly the electron density in reciprocal space. 

We can also arrive at the scattered amplitude by considering the scattering from each atom in the unit cell. X-ray scattering from individual atoms $f(\mathbf{q})$, called atomic form factors, are tabulated \cite{Henke1993}. The scattering from an individual unit cell can be calculated by summing the form factor and weighting it by its position.

\begin{equation}
F(\mathbf{q}) = \sum_{\text{unit cell}} f(\mathbf{q}) e^{i \mathbf{q} \cdot \mathbf{r}}
\end{equation}

The scattering intensity is given by

\begin{equation}
I(\mathbf{q}) = |F(\mathbf{q})|^{2}
\end{equation}

which is measured in a scattering experiment. From the measured intensity, we can make inferences about $\rho_{e}(\mathbf{r})$ that determine the crystal structure.

\begin{centering}
\includegraphics[width=0.75\textwidth]{./img/FourRingDiffractometer.png}
  \captionsetup{width=0.75\textwidth}
  \captionof{figure}[Diagram of x-ray scattering angles]{Four-ring diffractometer with scattering angles labeled. From \cite{Clark2007}.} 
  \label{fig:XrayIntro-4}
\end{centering}

In the laboratory, the scattering vector $\mathbf{q}$ is given not by  $x,y,z$ coordinates, but instead by four angles: $\theta$, $\phi$, $\Omega$, and $\chi$. The definitions of these angles are shown in Figure \ref{fig:XrayIntro-4}. Loosely speaking, the angles $\theta$ and $\phi$ specify the orientation of the crystal, and the angles $\Omega$ and $\chi$ describe the direction of the beam.

\section{Previous measurements of \rucl}

As discussed in Chapter 2, x-ray diffraction studies of bulk single-crystal \rucl show it to be a two-dimensional material (interlayer spacing 5.7 \AA) of space group C2/m having monoclinic symmetry \cite{Johnson2015}. However, electron diffraction measurements of \rucl exfoliated by lithium intercalation are consistent with a P3\textsubscript{1}12 space group, which has trigonal symmetry. These different symmetries may be due to stacking faults introduced by the exfoliation process that cause the stacking order to change from \textit{ABC} to \textit{AB} \cite{Gronke2018}.

X-ray measurements also show that \rucl immersed in a solution of ions and subjected to an electric field will rapidly and reversibly incorporate cations into its structure. \rucl favors intercalation of cations because each \rucl monolayer is bounded by electronegative chlorine planes. X-ray diffraction measurements show that the interlayer spacing of intercalated \rucl is approximately equal to the ionic radius of the intercalated cation \cite{Schollhorn1983,Steffen1986}. The change in interlayer spacing for \rucl intercalated by selected cations is summarized in Figure \ref{fig:XrayIntro-5}.

\begin{centering}
\includegraphics[width=0.75\textwidth]{./img/RuCl3IntercalatedSpacing-original.png}
  \captionsetup{width=0.75\textwidth}
  \captionof{figure}[Intercalated \rucl interlayer spacing]{Interlayer spacing of \rucl intercalated by selected ionic species. The interlayer spacing for intercalated \rucl is almost double that of pristine \ruclnospace .} 
  \label{fig:XrayIntro-5}
\end{centering}

These results suggest that x-ray diffraction can distinguish between intercalated and unintercalated \rucl by determining the interlayer spacing.

\section{X-ray measurements of \rucl at Stanford} 

\subsubsection{Methods}
X-ray scattering measurements were made at Stanford in the Stanford Nano Shared Facilities X-ray Diffraction Laboratory on the Bruker D8 Venture single crystal x-ray diffractometer using Cu-generated 1.54 \AA{} x-rays. Single crystal samples of \rucl approximately 100 $\mu$m thick were affixed to a glass substrate with thin pieces of scotch tape before measurement. Pristine \rucl exfoliated onto a 300 nm Si/SiO\textsubscript{2} substrate were measured similarly.

Electrolyte gating samples were fabricated using techniques similar to those described in the methods sections of chapters 4 and 5. The samples were electrically contacted using silver epoxy and 8 mil insulated copper wire.

For electrolyte gating measurements, samples were mounted using tape and a goniometer stage. I selected angles $\theta$, $\phi$, $\Omega$, and $\chi$ that maximized the intensity of the interlayer scattering peak before applying ionic liquid. After finding the optimal angles and setting the diffractometer appropriately, I removed the sample and applied the ionic liquid. I subsequently reinstalled and electrically connected the sample. Potential bias was provided by a Keithley 2400 source measurement unit.

\subsection{Pristine \rucl}

Measurements of bulk single-crystal \rucl are consistent with literature, as shown in Figure \ref{fig:XrayMeas-1}. However, I find the crystal structure consistent with the trigonal symmetry found in powder diffraction measurements  \cite{Fletcher1967} and in exfoliated flakes \cite{Gronke2018}, rather than monoclinic C2/m symmetry. As suggested in the literature, stacking faults may be the cause of differing symmetries. The presence of these stacking faults is unsurprising - the bulk sample I measured is a cutting from a larger bulk crystal which was roughly handled.

\begin{centering}
\includegraphics[width=0.75\textwidth]{./img/BulkRuCl3XrayDiffraction.png}
  \captionsetup{width=0.75\textwidth}
  \captionof{figure}[Bulk \rucl diffraction pattern]{Bulk \rucl diffraction pattern as measured at Stanford. Peaks are labeled with their Miller indices.} 
  \label{fig:XrayMeas-1}
\end{centering}

In exfoliated \ruclnospace , the (003) interlayer peak remains visible, but the intensity of the remaining peaks is too small to resolve (see Figure \ref{fig:XrayMeas-2}). However, the (003) peak remains at the same value of $2\theta$ independent of exfoliations, showing that the interlayer separation remains a constant 5.7 \AA .

\begin{centering}
\includegraphics[width=0.75\textwidth]{./img/ExfoliatedRuCl3Xray-original.png}
  \captionsetup{width=0.75\textwidth}
  \captionof{figure}[Exfoliated \rucl diffraction pattern]{Exfoliated \rucl diffraction pattern as measured at Stanford. At this particular value of $\phi$, substrate peaks are not visible. However, they are present at other values of $\phi$.} 
  \label{fig:XrayMeas-2}
\end{centering}

Despite being too small to resolve on a log scale, there does appear to be at least one other x-ray peak present for exfoliated \rucl at particular values of $\phi$ with 120\degree{} symmetry. This feature is less than a factor of two above background and does not map exactly to a known scattering plane in \rucl (though being less than 2\degree{} from the (104) plane). This feature is highlighted in Figure \ref{fig:XrayMeas-3}.

\begin{centering}
\includegraphics[width=0.5\textwidth]{./img/ExfoliatedRuCl3Diffractogram-original.png}
  \captionsetup{width=0.75\textwidth}
  \captionof{figure}[Exfoliated \rucl diffractogram]{Exfoliated \rucl diffractogram (contrast enhanced) as measured at Stanford. Notable features include the \rucl interlayer separation at $\chi = 0$ and $\theta = 15.7$\degree, the ring-like features associated with diffraction from evaporated Au contacts, and the bright feature at large $\theta$ and $\chi < 0$ from the Si substrate. The feature highlighted in green I attribute to exfoliated \ruclnospace .} 
  \label{fig:XrayMeas-3}
\end{centering}

\subsection{Electrolyte-gated \rucl}

The diffraction pattern of exfoliated \rucl does not seem to change as a function of electrolyte bias. Figure \ref{fig:XrayMeas-4} shows the diffraction pattern at different electrolyte voltages - note that the interlayer separation feature remains at the same value of $2\theta$, indicating the interlayer separation remains constant.

\begin{centering}
\includegraphics[width=0.75\textwidth]{./img/ExfoliatedRuCl3InterlayerSeparationGateVoltage-original.png}
  \captionsetup{width=0.75\textwidth}
  \captionof{figure}[Electrolyte-biased exfoliated \rucl diffraction pattern]{Electrolyte-biased and exfoliated \rucl diffraction pattern at different electrolyte voltages, offset for clarity. Note that the feature corresponding to \rucl interlayer separation maintains the same value of $2\theta$ and has the same width despite the changing electrolyte bias.} 
  \label{fig:XrayMeas-4}
\end{centering}

The feature near the value of $2\theta$ for the (104) scattering plane is further attenuated after the application of the electrolyte and becomes hard to resolve. With my current data, it is hard to tell if this peak is hard to resolve because its corresponding scattering plane becomes disordered, or if it's just too faint at the available x-ray intensity.

\section{Discussion}

X-ray diffraction measurements of exfoliated \rucl show that the interlayer separation does not change as a function of electrolyte bias. Accordingly, given the large size of the ions comprising the electrolyte, we can be confident that the ions do not migrate into and out of the lattice. Therefore, the hysteretic behavior observed in electronic transport and Raman spectroscopy cannot be explained by the intercalation of DEME+.

Additionally, because the interlayer separation does not change, the chemical interaction between the electrolyte and the exfoliated \rucl is limited to the top and the sides of the material. The measured flakes of \rucl have thicknesses from between 50 nm to 100 nm, corresponding to 100 to 200 monolayers of \ruclnospace . Accordingly, any electrochemical intercation can account only for 1\% of the observed signal - a negligible amount. Therefore, these x-ray measurements both rule out intercalation by DEME+ and show that electrochemistry is negligible.

Possible references:
%Kim, H.-S., & Kee, H.-Y. (2015). Crystal structure and magnetism in alpha-RuCl3: an ab-initio study, 155143, 1–10. https://doi.org/10.1103/PhysRevB.93.155143
Discusses different crystal structures

%Johnson, R. D., Williams, S. C., Haghighirad, A. A., Singleton, J., Zapf, V., Manuel, P., … Coldea, R. (2015). Monoclinic crystal structure of α-RuCl3 and the zigzag antiferromagnetic ground state. Physical Review B - Condensed Matter and Materials Physics, 92(23). https://doi.org/10.1103/PhysRevB.92.235119
Talks about stacking faults