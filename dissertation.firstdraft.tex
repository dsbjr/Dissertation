%\documentclass[12pt,twoside]{report}
\documentclass[12pt]{report}

%% Last Modification by Derrick Boone 5 February 2019. Template from Emma Pease

% note that the document can be single or double sided.  
% note that the Registrar's office now allows 10pt, 11pt, or 12pt

\usepackage{suthesis-2e}
\usepackage{caption}
\usepackage{graphicx}
\usepackage{amsmath}
\usepackage{amsthm}
\usepackage{braket}
\usepackage{url}
\usepackage{gensymb}
\usepackage{textcomp}
\usepackage{braket}
\usepackage{amsfonts}
%\usepackage{geometry}
\usepackage[toc,page]{appendix}
\usepackage[pdfencoding=auto]{hyperref}
% default is now online June/2016 version
%\usepackage[online]{suthesis-2e}
%\usepackage[hardcopy]{suthesis-2e}
% the following is for doing engineering theses. 
% I am definitely not sure of the wording on the signature page so check
%\usepackage[engineer]{suthesis-2e}

% one can change the default font to Times Roman but note that most
% ways of creating pdf files from latex automatically embed (which btw
% is a good idea even with the standard fonts)

    \title{Electrolyte Biasing of the Proximate Kitaev Spin Liquid $\alpha$-Ruthenium (III) Chloride}
    \author{Derrick Sherrod Boone, Jr.}
    \dept{Applied Physics}
    \principaladviser{David Goldhaber-Gordon}
    \firstreader{Ian Fisher}
    \secondreader{Marc Kastner}
%% one can also have a \thirdreader and \fourthreader

%% note that certain departments and types of theses have other requirements
%% For instance theses in the departments of 
%% Asian Languages
%% French and Italian
%% Spanish and Portuguese
%% need to define the \dept, \dualthesis, and the actual language
%\dualthesis
%\languagemajor{Chinese}
%% 
%% Those for Graduate Program in Humanities need to define 
%\humanitiesthesis
%\jointprogram{Arts and Crafts}
%% 
%% For submission to a committee or program (no department)
% \committeethesis
% \programthesis
%%
%% For School of Education or Business or Law
% \educationthesis
% \businessthesis
% \lawthesis  (law actually isn't listed in the official documents, 2013/1014)

%DEFINE NEW COMMANDS
\newcommand{\rucl}{RuCl\textsubscript{3} }
\newcommand{\ruclnospace}{RuCl\textsubscript{3}}
\newtheorem{assumption}{Assumption}
\newcommand{\percmsq}{e\textsuperscript{-}/cm\textsuperscript{2}}
\newcommand{\pplus}{P\textsubscript{+}}
\newcommand{\pminus}{P\textsubscript{-}}

\bibliographystyle{unsrt}

\begin{document}

% for a variety of reasons this is an all in one document; however,
% when actually doing the thesis it is strongly recommended that each
% chapter be in a separate file and use \include to include in the
% main file.

%% the \beforepreface command produces the title page
%% in the online version it skips the copyright (page 2) and signature (page 3) pages 
%% in the non-online version these would be included
    \beforepreface


%% Abstract can be any number of pages
    \prefacesection{Abstract}
A quantum spin liquid (QSL) is an exotic phase of matter identified by interacting spins that do not develop long-range magnetic order down to T = 0 K. QSLs have topological order and are therefore relevant to storing and processing quantum information. Additionally, doped QSLs may be related to high temperature-superconductivity, making the study of this phase relevant to a broad section of condensed matter physics.

Recently, the two-dimensional material and spin-assisted Mott insulator $\alpha$-ruthenium (III) chloride (\ruclnospace) has attracted research interest because of its relationship to spin liquids. Localized electrons in \rucl have anisotropic Ising interactions that are described by the Kitaev model – a theoretical model of a spin liquid that is exactly solvable. The low-temperature antiferromagnetic order in \rucl can be suppressed by an in-plane magnetic field to create a field-induced QSL, identifying \rucl as a proximate Kitaev spin liquid.

Because \rucl is both a proximate Kitaev spin liquid and a Mott insulator, charge doping may create interesting electronic phases. Previous attempts to chemically dope \rucl have been unsuccessful. However, previously-used doping techniques either disordered the lattice or added an unmeasured amount of charge, making this negative result hard to interpret. Charge doping by electrolyte biasing would eliminate these problems, and improve on previous measurements.

I use electronic transport measurements and Raman spectroscopy under electrolyte bias to show that, upon doping, \rucl undergoes a first order phase transition with only small changes in conductivity. X-ray diffraction of electrolyte-biased \rucl rules out intercalation being the cause of this phase transition, and confirms the assumption that electrochemical reactions are negligible. I suggest that this phase transition prevents \rucl from being used to study doped spin liquid physics and present further measurements to characterize the newly identified phase.

%% one can also have a prefacesection that is a Preface instead of
%% Acknowledgements.   The thematic purpose is the same (thanks).
    \prefacesection{Acknowledgements}
        PhDs do not happen in a vacuum, and mine is no exception. I am indebted to too many people to name; please indulge the following attempt.
        
        I would first like to honor my committee. During my time at Stanford I have had the pleasure and privilege of being advised by David Goldhaber-Gordon. I have yet to meet a kinder or more intelligent man, and the way he does science remains an inspiration to me. David's advisor (and my grand-advisor) Marc Kastner was the mastermind behind this research, and if you look closely at this work you will see his influence everywhere. Marc's incredible knowledge and guidance were invaluable - I can truthfully say I would not have finished without him. Ian Fisher, my academic advisor, has been an incredible teacher and thought partner through this process. I am grateful for the time you spent with me, despite me not being a member of your group. My defense chair, Evan Reed, was the first person who explained band theory to me in a way I understood. Thank you, Evan, for making sure I really got it. Finally, Benjamin Feldman is an experimentalist whose work I've admired even before he became a Stanford faculty member. I appreciate you lending your time to make sure I was successful.

        My mentors in the Goldhaber-Gordon group have also been scientists beyond compare. John Bartel, who taught me what a low-pass filter was; Menyoung Lee, who built the group's first graphene transfer station and showed me e-beam lithography; and Patrick Gallagher, who taught me a  list of things to long to recount and whose AFM tips I used until my very last day: thank you. My immediate seniors, Eli Fox and Lucas Peeters, provided sorely needed discussions that helped sharpen my understanding of this work and our shared discipline. My peers - Aaron Sharpe, Winston Pouse, Connie Hsueh, and Linsey Rodenbach - you have been more than just colleagues. You are treasured friends. I will cherish my bond with you for the rest of my life. And to all the members of the group, your support, encouragement, and empathy are what allowed me to persevere. Thank you.

        I had the honor of visiting the National Institute of Standards and Technology during summer 2015. I thank the electron physics group for hosting me, Dr. Joseph Stroscio for advising me during my time there, and the National Physical Sciences Consortium for funding this trip and partially funding four years of my PhD.

        My family has been a consistent source of comfort and motivation over the past six years. I thank my father, Dr. Derrick S. Boone, Sr., for exposing me to math and science at an early age and for staying with me during the last month leading up to my defense. I thank my mother, Patricia Boone, for loving me and keeping me sane during a challenging childhood. And I thank my sister, Madeleine Boone, and brother, Jacob Boone, for their consistent support.

        My wife, Andrea Leigh Schwartz Boone, has been an incredible partner to me since our meeting in 2010, reunion in 2013, and marriage in 2016. There is no one I love, respect, or admire more. We have been through so much in our short time together, and through it all you have given me the gift of a home. To you I owe a debt I cannot repay.

        Finally, to my son and first child, Everett James Boone: being your father is the realization of my greatest dream. I have loved you from the moment I felt your first kick in your mother's womb, and I am in awe of you each day. Know that as long as I draw breath, I will be your biggest cheerleader and greatest champion. I dedicate this dissertation to you.

%% afterpreface produces a table of contents and any other tables
%% wanted. At the end pagenumbering changes from roman to arabic and
%% is restarted
    \afterpreface
 
 
\chapter{Introduction}
Condensed matter physics is the rigorous study of what happens when a large number of cold atoms at high density are allowed to interact. It tells us why and at what temperature water freezes, why magnets attract some materials and not others, why glass is clear, why metal is shiny, and many other things. It is the branch of physics that reveals the richness of our physical world.

\section{Phases of matter}
Many materials that differ in their constituents and microscopic structure have similar bulk properties. For example, although water and mercury at ambient conditions have dramatically different densities and electrical conductivities, they are both nearly incompressible and deform continuously when a shear stress is applied. We capture these similarities by saying water and mercury are both in the liquid phase\footnote{Depending on the context, there may be a difference between a \textit{phase} of matter and and \textit{state} of matter. I will use phase in this dissertation as it seems to apply more generally.}. Phases of matter arise not just because of constituent particles, but also by the way those constituents are arranged.

A phase of matter has uniform equilibrium thermodynamic properties (density, magentization, etc.) and is defined by these properties being analytic functions of the thermodynamic parameters (e.g., temperature, pressure) \cite{Pathria2011}. Therefore, the properties of matter in static equilibrium in a given phase are the same for all space, and these properties are smooth functions of the parameters. For example, liquid water at a uniform temperature has the same density everywhere, and when it is heated by a small amount, its density decreases by a corresponding small amount. Phases of matter are separated by phase transitions, where the thermodynamic properties (or their derivatives with respect to a parameter) are no longer continuous\footnote{Infinite order phase transitions are a theoretical exception. See \cite{Costin1990}}. For example, when liquid water boils at ambient pressure, its temperature remains the same, but its density decreases discontinuously by a factor of $10^6$.

We can also use Landau theory \cite{Landau1937} to describe phases of matter by the symmetries of their Hamiltonian\footnote{A symmetry is an operation which leaves the Hamiltonian of the system invariant. For example, the Hamiltonian of a particle in free space $H = \Sigma_{i} \frac{p_{i}^{2}}{2m}$ is invariant under spatial translation $x \rightarrow x + a$}, and the phase transitions between them as the breaking or recovery of those symmetries. For example, when a liquid freezes into a solid crystal, the continuous translational symmetry of the liquid phase becomes a discrete translational symmetry as the molecules in the liquid assemble themselves into a liquid. Another example is a material transitioning from a non-magnetic to ferromagnetic phase. When the magnetic moments of the material align, it gains an overall macroscopic magnetization, breaking rotational symmetry. The following table lists some common phases and the symmetries they break \cite{Chaikin1995}.

\begin{center}
\resizebox{\textwidth}{!}{
	\begin{tabular}{l | l | l | l | l | l | l}
		\hline
		\hline
		\textbf{Phase} & Fluid & Nematic & Smectic-A & Crystal & Heisenberg Magnet & Superfluid \\ \hline
		\textbf{Broken Symmetry} & None & Rotational & 1D Translation & 3D Translation & Rotational & Phase\\ \hline \hline
	\end{tabular}
	}
	\captionof{table}{Selected phases and their associated broken symmetries}\label{tbl:nicetabelesstable}
\end{center}
		
However, in addition to the above examples, there are kinds of matter which maintain a single set of symmetries but nonetheless have phases separated by phase transitions. These kinds of matter possess topological order - a type of order that can define a phase of matter just like symmetry can \cite{Wen1990}.

\section{Topological Phases}

Topological order is a property of quantum systems that have both long-range entanglement and large ground state degeneracy. In these systems, there is no local order parameter like density or magnetization. Instead, there is a global topological invariant that changes discontinuously between phases \cite{Wen2017}. First, we set out to understand topology by considering the properties of a simple quantum topological system - Kitaev's toric code. Using the topological concepts we learn from the toric code, we will be able to understand the topological nature of some physical systems, including quantum hall states and the eventual subject of this dissertation: the spin liquid.

\subsection{The toric code}

The following explanation draws heavily from \cite{Kitaev2003} and \cite{topOrderEdX}.

Consider a system of spin-$\frac{1}{2}$ electrons living on the edges of a square lattice. We first define two operators:

\begin{align*}
A_s&=\prod_{j \in star(s)} \sigma^{x}_{j} 		& B_{p}&=\prod_{j \in plaquette(s)} \sigma^{z}_{j}
\end{align*}


 

Fuck all this - teach topology with the Toric code.

Interacting system spins on a 2D lattice with a special hamiltonian. The ground state of this Hamiltonian is a loop gas - as long as there are no free ends, the energy of the system is the same. Therefore, any state with loops is a good state. The vertex operator changes between degenerate states. Basically, it changes the loop configuration.

If you put this lattice on a torus, then you get loops that cannot be deformed by the vertex operator smoothly into other loops. These are the loops that go across the periodic boundary. 

Things to remember: Quantum hall states have topological order - consider the conductivity an order parameter - as you change the field this parameter changes discontinuously without a change in symmetry. The chern number you can derive from the band structure (?) and is a topological order parameter that changes. Maybe I can start with the quantum hall effect?

Start with IQHE - explain it and then show how the conductivity is an order parameter. Then show how the FQHE extends the IQHE. Then show the topological excitations in the FQHE are evidence of topological order and their presence requires the ground state to have topological order. Then show that spin liquids are a state that has topological excitations, and therefore topological order - even in the absence of symmetry breaking.

Somehow topological order is connected to fractional excitations in two dimensions. Fractional quantum hall liquids always have edge excitations.

Topological order is connected to the ground state degeneracy of a system. The ground state degeneracy is somehow related to the topological order parameter.

Consider using the Toric code.

What is condensed matter, what's a spin liquid, why gate it, etc... This is going okay so far.

\chapter{Properties of \texorpdfstring{$\alpha$-Ruthenium Trichloride}{alpha-RuCl3}}
This chapter covers the structural, electronic, and magnetic properties of \rucl and serves as a reference for further discussion of the material. This chapter also contains a discussion of the Kitaev model and how \rucl realizes this model in the lab. Finally, I discuss published measurements of \rucl and how they confirm it as a proximate Kitaev spin liquid.

\section{Synthesis and Crystal Structure}

Generally, ruthenium (III) chloride (\rucl) is a brown or black solid that is a precursor for 	ruthenium chemistry. It is commonly found as a hydrate, where water molecules are incorporated into its structure. High-purity hydrated ruthenium (III) chloride is commercially available from chemical vendors like Sigma Aldrich. We will consider only anhydrous ruthenium (III) chloride (i.e., without water) in this dissertation.

Ruthenium (III) chloride has two polymorphs, called $\alpha$ and $\beta$. Both $\alpha$ and $\beta$-ruthenium (III) chloride are composed of ruthenium atoms octahedrally coordinated by chlorine. However, in the $\alpha$ polymorph the octahedra are edge sharing, while in the $\beta$ polymorph, the octahedra are face-sharing. The edge-sharing octahedra give the $\alpha$ polymorph its interesting magnetic qualities, and therefore, unless explicitly stated otherwise, \rucl will refer to the $\alpha$ polymorph.

As stated previously, the structure of \rucl 



\section{Electronic Structure}

Discuss evidence that RuCl3 is a Mott insulator. Show plots of calculated electronic structure.

\section{Magnetic Properties}

Discuss magnetic transitions - paramagnet to antiferromagnet. Stacking faults causing additional magnetic transitions.

\section{\texorpdfstring{\rucl}{RuCl3} as a Kitaev Spin Liquid}

\subsection{The Kitaev Model}

Discuss and solve the Kitaev model.

\subsection{Spin Liquid Behavior in \texorpdfstring{\rucl}{RuCl3}}

Measurements that show \rucl is a proximate Kitaev spin liquid: neutron scattering, field-induced disordered states, half-quantized thermal hall conductance, magnetic impurity doping suppressing AF interaction.

\chapter{Electrolyte Gating}

This chapter provides an introduction to electrolyte gating. We discuss the mechanism underlying gating methods, the advantages and disadvantages of gating with an electrolyte, how to choose an electrolyte, and how to design a device for gating. We conclude with a discussion of some important gating measurements.

\section{Mechanism of field effect gating}

Condensed matter physicists and electrical engineers often add charge carriers (either electrons or holes) to a material with the hope of changing its properties - increasing or decreasing conductivity, changing an insulator into a metal, or even changing a material's physical structure. The most common and least intrusive method to add carriers uses the field effect to electrostatically gate a material. The field effect adds carriers using an electric field rather than chemical substitution or intercalation. 

We can define an electrochemical potential $\mu$ that represents the amount of energy required to add an electron to a system. Using this definition

\begin{equation}
\mu = E_{F} + qV
\end{equation}

where $E_{F}$ is the Fermi energy\footnote{This is only correct at T = 0 K. At finite temperatures, $E_{F}$ should be replaced with an electrochemical potential at zero field, which reflects the broadening of the occupation function. However, for most materials, the Fermi temperature is so high that the difference is negligible.}, $q$ is the fundamental charge, and $V$ the applied electric potential. When two systems with different electrochemical potentials are brought into contact, heat and charge will flow across the interface until the electrochemical potentials are equal. We can exploit this effect to add carriers to a material. If we apply a field to a material that decreases $\mu$, then electrically connect that material to a reservoir (i.e., electrical ground), charge carriers will flow into the material until the chemical potentials are equal. We have used the field effect to increase the carrier density in our material. What is probably the most produced device in the world, the metal-oxide-semiconductor field effect transistor (MOSFET), uses a metal ``gate'' and an insulating oxide to introduce carriers in a semiconductor, controllably changing the conductivity of the semiconductor channel in the device.

\begin{centering}
\includegraphics[width=0.5\textwidth]{C:/Users/dsbjr/Documents/GitHub/Dissertation/img/MOSFET-original.png}
  \captionsetup{width=0.75\textwidth}
  \captionof{figure}[MOSFET Diagram]{Cartoon of a metal-oxide-semiconductor field effect transistor (MOSFET). Voltage $V_{g}$ applied to the metal gate electrode creates a field which polarizes the oxide dielectric and induces charge carriers in the channel. When the channel is conducting, $V_{sd}$ applied between the source and drain electrodes drives a large current.} 
  \label{fig:ElecGate-1}
\end{centering}

We can calculate the charge added to the channel by modeling the MOSFET as a parallel plate capacitor

\begin{equation}
Q = CV_{g} = \epsilon \frac{A}{d} V_{g}
\end{equation}

Where $A$ is the area of the electrode and $d$ is the thickness of the oxide, we can divide by area to find the induced charge density $n$

\begin{equation}
n = \frac{Q}{A} = \frac{\epsilon V_{g}}{d}
\end{equation}

The strongest SiO\textsubscript{2} can withstand a field strength of approximately 1 V/nm before breakdown\footnote{Breakdown occurs when being subjected to an electric field causes a dielectric to lose its insulating properties. Breakdown is caused by a combination of how long the dielectric is subjected to the field and how long the dielectric is subjected to the field.} \cite{Palumbo2019}. Using our parallel plate capacitor model, the maximum amount of charge that can be induced this way is approximately 2 x $10^{12}$ \percmsq . For specially grown high-K dielectrics, this maximum induced charge increases to about 5 x $10^{13}$ \percmsq \cite{Robertson2004}.

While these density changes are impressive, there are use cases for field effect gating that require greater changes in density, including reaching the van Hove singularity in graphene or removing all the carriers from a metal thin film. To achieve larger changes in density, we can appeal to the technique of electrolyte gating.

\begin{centering}
\includegraphics[width=0.5\textwidth]{C:/Users/dsbjr/Documents/GitHub/Dissertation/img/ElectrolyteGating-original.png}
  \captionsetup{width=0.75\textwidth}
  \captionof{figure}[Mechanism of Electrolyte Gating]{Electrolyte gating uses an electrically polarized molten salt to accumulate ions at the surface of a material of interest. This layer of ions creates an electic double layer capacitor, inducing charge into the material.} 
  \label{fig:ElecGate-2}
\end{centering}

Electrolyte gating (\ref{fig:ElecGate-2}) replaces an oxide dielectric with an electrolyte (a molten salt) consisting only of positively and negatively charged species. When a voltage is applied to the gate electrode, the applied electric field causes the charged species to migrate: negative charges accumulate at the gate and positive charges accumulate at the channel (or vice-versa, depending on the sign of the gate voltage). This polarization of charge forms an electrical double layer capacitor at the interface of the channel. To get a sense of the possible induced charge, we can use our previous equation for density, replacing $d$ by the ionic radius of the species (typically ~1 nm), $\epsilon$ by the appropriate dielectric constant, and $V_{g}$ by the maximum voltage that can be applied to the liquid. In this case we find an induced density of 5 x $10^{14}$ \percmsq , an order of magnitude higher than that possible with the best oxide dielectrics.

Changing carrier density by an additional order of magnitude beyond oxide gating is a powerful tool, but electrolyte gating also has strengths and weaknesses that must be considered when designing and executing an experiment. These are discussed in the following section.

\section{Considerations for electrolyte gating}

A successful electrolyte gating experiment involves choosing the most appropriate electrolyte and thoughtfully designing the device. We'll examine both.

\subsection{Choosing the right electrolyte}

Electrolytes are typically composed of monovalent, complex molecules. A wide variety of cations and anions are available for use. A selection of these are presented in Figure \ref{fig:ElecGate-3}.

\begin{centering}
\includegraphics[width=0.75\textwidth]{C:/Users/dsbjr/Documents/GitHub/Dissertation/img/IonicLiquids-Petach.png}
  \captionsetup{width=0.75\textwidth}
  \captionof{figure}[Structure of Electrolyte Cations]{Selected cations and anions used in anhydrous electrolytes. a. Diethylmethyl(2-methoxyethyl)ammonium, DEME\textsuperscript{+} b. 1-butyl-methylpyrrolidinium, BMPY\textsuperscript{+} c. 1-ethyl-3-methyl-imidazolium, EMI\textsuperscript{+} d. Tetrafluoroborate, BF$_{4}^{-}$ e. Tris(pentafluoroethyl)trifluorophosphate, FAP\textsuperscript{-} f. bis(trisfluoromethylsulfonyl)imide, TFSI\textsuperscript{-}. From \cite{Petach2017}.} 
  \label{fig:ElecGate-3}
\end{centering}

The most important concern when choosing an electrolyte is chemistry. Unlike metal-oxide gating, which uses a chemically inert oxide, electrolyte gating involves intimate contact between the material of interest and the (potentially) chemically active electrolyte. Chemical reactions between the material and the electrolyte will change the nature of the material and disturb the double-layer capacitance, making the gating ineffective at best and destructive at worst. 

Applying a potential across the electrolyte creates the possiblity of not just chemical, but electrochemical (i.e., reduction-oxidation) reactions that can have similar effects to those listed above. A method to characterize the electrochemistry of an electrolyte is to measure its electrochemical stability window - the range of voltages beyond which the ions in the electrolyte will begin to react at the electrodes. The electrochemical stability sets the maximum potential that can be applied across the electrolyte, and therefore the maximum induced carrier density.

The electrochemical stability window is typically measured with cyclic voltammetry - sweeping the potential across the electrolyte at a constant rate and measuring the current. This current should be constant, and deviations suggest chemistry is occurring. A cyclic voltammetry measurement for DEME-TFSI is presented in Figure \ref{fig:ElecGate-4}. As seen in the figure, the behavior of the gate current becomes substantially nonlinear around $\pm 2$ volts. Accordingly, I have assumed the electrochemical stability window for this electrolyte and  measurement to be -2 to +2 volts. The observed electrochemical stability window for DEME-BF4 is similar.

\begin{centering}
\includegraphics[width=0.5\textwidth]{C:/Users/dsbjr/Documents/GitHub/Dissertation/img/DEME-TFSIStabilityWindow-Original.jpg}
  \captionsetup{width=0.75\textwidth}
  \captionof{figure}[Electrochemical stability window for DEME-TFSI]{Cyclic voltammetry measurement of DEME-TFSI (293 K, 1 x $10^{-6}$ Torr). The sweep rate is 13 mV/second. Current density is defined at the working electrode.} 
  \label{fig:ElecGate-4}
\end{centering}

Another concern for electrolyte gating is contamination of the electrolyte. Electrolytes are polar and hygroscopic; therefore, they will collect moisture and other contaminates from the atmosphere. When solvated in the ionic, these contaminants can create undesirable electrochemical interactions. Fortunately, the high vapor pressure of electrolytes means they can withstand vacuum, so some water can be removed by baking at elevated temperature in high vacuum. However, water remains the biggest source of contamination and exposure to air will necessarily reduce the electrochemical stability window. When possible, in my measurements I have attempted to reduce the exposure of the electrolyte to atmosphere.

Finally, electrolytes freeze below room temperature, which offers some challenges and advantages. Ions in an electrolyte must be free to move for the electrolyte to be polarized, so the electric field must be applied above the freezing temperature. The lower the freezing temperature, the lower the temperature at which the field can be applied. Because electrochemical reactions are suppressed at low temperatures, applying the field just above the freezing temperature minimizes There are often changes in the electric double layer near the freezing temperature, so it is important to avoid electrolytes with transition temperatures in a range of interest for the gated material.

\subsection{Thoughtful device design}

In a typical electrolyte gating experiment, a droplet of the electrolyte simultaneously covers a large metal coplanar gate and the material of interest. In addition to the concerns discussed previously, this configuration presents additional problems that can be minimized through device design.

Decreases in electric potential (i.e., potential drops) can occur anywhere between the biased coplanar gate and ground. For example, the dominant potential drop could occur at the gate-electrolyte interface, at the electrolyte-channel interface, or even between the coplanar gate and a substrate. Further, the electrolyte may form a passivating layer that can alter these potential drops as a function of applied gate voltage.

An effective electrolyte gating measurement has the dominant potential drop at the electrolyte-channel interface. Using a larger coplanar gate increases the likelihood that the potential will fall predominantly at the electrolyte -channel interface. Because impedance through the electrolyte goes as the square of inverse area, having a relatively large coplanar gate reduces the 


\section{Notable Electrolyte Gating Measurements}

Ambipolar gating of WS2. MoS2 metal-insulator transition. Trevor Petach's measurements.

\chapter{Electronic Transport}
This chapter covers all the transport measurements

\chapter{Raman Spectroscopy}

Raman spectroscopy uses inelastic light scattering to measure certain vibrational modes in materials. For single molecules, these modes correspond to Raman-active vibrations that change the polarizability of the molecule. For crystals, these modes are the collective motion of the lattice - the phonon modes.

I have used Raman spectroscopy to investigate the phonon modes of electrolyte-gated \ruclnospace. Because phonon modes are sensitive to bond lengths and atomic interactions, defects or distortion in the lattice should be visible as shifting or broadening of peaks in the Raman spectrum. Therefore Raman spectroscopy can identify changes in the lattice that may explain the absence of an expected electronic phase transition in electrolyte-gated \ruclnospace.

\section{Fundamentals of Raman Spectroscopy}
When a material is illuminated by monochromatic light of frequency $\omega$, the incident light is either absorbed, scattered, reflected, or transmitted. While the spectrum of the transmitted and reflected light contains only light at frequency $\omega$, the spectrum of the scattered light includes both radiation at $\omega$ and additional pairs of frequencies $\omega \pm \omega_{i}$. The frequencies $\omega_{i}$ are typically associated with transitions between rotational, vibrational, and electronic states of the constituent molecules of the material \cite{Long2002}. The scattered light typically has a randomized phase and polarization relative to the incident light.

The highest intensity radiation in the spectrum of scattered light occurs at frequency $\omega$ and is called Rayleigh scattered\footnote{Rayleigh scattering is responsible for diffuse sky radiation, or in otherwords, why the sky is blue.}. Radiation at other frequencies is Raman scattered. Raman scattering with frequency $\omega_{i} < \omega$ is called Stokes Raman scattering; Raman scattering with frequency $\omega_{i} > \omega$ is called anti-Stokes Raman scattering. Necessarily, Rayleigh scattering is an elastic scattering process ($E = \hbar \omega = \hbar \omega_{i}$), while Raman scattering is inelastic ($E_{\text{initial}} = \hbar \omega \neq E_{\text{final}} = \hbar \omega_{i}$).

A single molecule Rayleigh scatters incident light by absorbing a photon, moving from the ground state to an unstable virtual state, then decaying back to the ground state by emitting a photon with energy equal to the incident photon. In contrast, a single molecule Raman scatters incident light by absorbing a photon and moving to unstable virtual state, then decaying to an intermediate state above the ground state by emitting a lower-energy photon \footnote{This is Stokes Raman scattering. If a photon is incident on a molecule in an already excited state, and then the molecule decays back to the ground state, the emitted photon has \textit{more} energy than the incident photon. This type of scattering is anti-Stokes scattering and its magnitude can be used to determine the temperature of a material.}. The intermediate state then decays to the ground state by non-photonic processes (i.e., cooling by collision with other molecules).

Scattering from single crystals follows a similar pattern: an incident photon excites a molecule into an unstable virutal state. The Rayleigh scattering case is identical to that for a single molecule. But for Raman scattering, the intermediate state decays to the ground state by emitting a phonon - a quantized lattice vibration that carries energy away from the excited molecule. By conservation of energy, the emitted photon necessarily has less energy than the incident photon, and that difference in energy\footnote{This difference in energy is referred to as the Raman shift, and is typically expressed in wavenumber $\nu$ having units of inverse length (typically cm$^{-1}$).}. must be equal to the energy of the phonon. Accordingly, the spectrum of Raman-scattered light is rich with information about lattice vibrations. Figure \ref{fig:FundRamanSpect1} provides a visual aid for understanding these scattering processes.

\begin{centering}
\includegraphics[width=0.9\textwidth]{C:/Users/dsbjr/Documents/GitHub/Dissertation/img/RamanRayleighScatter-nanophoton.png}
  \captionsetup{width=0.75\textwidth}
  \captionof{figure}[Elastic and inelastic light scattering processes]{An energy level diagram of Rayleigh, Stokes Raman, and anti-Stokes Raman scattering. Image by Nanophoton Corporation, retrieved from \url{https://www.nanophoton.net/raman/raman-spectroscopy.html}}
  \label{fig:FundRamanSpect1}
\end{centering}

While Raman spectroscopy is an invaluable tool, it does have limitations. Only certain vibrational modes can participate in Raman scattering. To understand which modes are amenable to investigation by Raman spectroscopy, and the underpinnings of Rayleigh and Raman scattering, we appeal to a classical analysis of the theory of light scattering from molecules. The following discussion is an abbreviated version done by Long \cite{Long2002}.

A molecule exposed to radiation having an incident frequency $\omega_{1}$ will radiate with intensity $I$, given by

\begin{equation}
I = k'_{\omega} \omega_{s}^{4} p_{0}^{2} \sin^{2} \theta
\end{equation}

with

\begin{equation}
k'_{\omega} = \frac{1}{32 \pi^{2} \epsilon_{0} c^{3}}
\end{equation}

where $p_{0}$ is the magnitude of the electric dipole induced at frequency $\omega_{s}$ which is generally but not necessarily different from $\omega_{1}$. The coresponding wavenumber equations are

\begin{equation}
I = k'_{\nu} \nu_{s}^{4} p_{0}^{2} \sin^{2} \theta
\end{equation}

\begin{equation}
k'_{\nu} = \frac{\pi^{2} c}{2 \epsilon_{0}}
\end{equation}

We can write the induced electric dipole as a multipole expansion:

\begin{equation}
\mathbf{p} = \mathbf{p}^{(1)} + \mathbf{p}^{(2)} + \mathbf{p}^{(3)} + ...
\end{equation}

where

\begin{equation}
\begin{aligned}
	\mathbf{p}^{(1)} &= \boldsymbol{\alpha} \cdot \mathbf{E} \\
	\mathbf{p}^{(2)} &= \frac{1}{2} \boldsymbol{\beta} \cdot \mathbf{E} \mathbf{E} \\
	\mathbf{p}^{(3)} &= \frac{1}{6} \boldsymbol{\gamma} \cdot \mathbf{E} \mathbf{E} \mathbf{E}
\end{aligned}
\end{equation}
	
where $\boldsymbol{\alpha}$, $\boldsymbol{\beta}$, and $\boldsymbol{\gamma}$ are the polarizability, hyperpolarizability, and second order hyperpolarizability tensors of second, third, and fourth rank, respectively. An analysis including $\boldsymbol{\beta}$ and $\boldsymbol{\gamma}$ leads to hyper Raman scattering related to higher harmonics of the incident radiation that are beyond the scope of this dissertation. Accordingly, we retain only the dipole term in the multipole expansion.

A priori, there is no reason to assume the polarizability of the molecule will be constant as the constituent atoms vibrate when the molecule is excited. Accordingly, we can expand the polarizability tensor $\boldsymbol{\alpha}$ in a Taylor series of displacements from equilibrium:

\begin{equation}
\alpha_{\rho \sigma} = (\alpha_{\rho \sigma})_{0} + \sum_{k} \left( \frac{\partial \alpha_{\rho \sigma}}{\partial Q_{k}} \right)_{0} Q_{k} + \frac{1}{2} \sum_{k,l} \left( \frac{\partial^{2} \alpha_{\rho \sigma}}{\partial Q_{k} \partial Q_{l}} \right)_{0} Q_{k} Q_{l} ...
\end{equation}

where $\alpha_{\rho \sigma}$ are the components of the polarizability tensor (with $\rho$ and $\sigma$ spanning values x, y, and z), $Q_{i}$ are the normal coordinates of vibrations associated with molecular vibrational frequencies $\omega{i}$, and a subscript 0 indicates a value at equilibrium. Taking the electrical harmonicity approximation\footnote{Electrical harmonicity means that the variation of the polarizability in a vibration is proportional to the first power of $Q$, in analogy to mechanical harmonicity where a restoring force is proportional to displacement from equilibrium.}, we retain only the first power of $Q$ and can write the previous equation in the following way:

\begin{equation}
\begin{aligned}
	(\alpha_{\rho \sigma})_{k} &= (\alpha_{\rho \sigma})_{0} + (\alpha'_{\rho \sigma})_{k} Q_{k} \\
	(\alpha'_{\rho \sigma})_{k} &= \left( \frac{\partial \alpha_{\rho \sigma}}{\partial Q_{k}} \right)_{0}
\end{aligned}
\end{equation}

The components $(\alpha'_{\rho \sigma})_{k}$ are a well-formed tensor $\boldsymbol{\alpha}'_{k}$, so we can write the above equation as:

\begin{equation}
\boldsymbol{\alpha}_{k} = \boldsymbol{\alpha}_{0} + \boldsymbol{\alpha}'_{k} Q_{k}
\end{equation}


\section{Raman Spectrum of RuCl$\textsubscript{3}$}
Raman modes in RuCl
Raman spectra of RuCl

\section{Raman Spectrum of Electrolytes}
Results of tests to select DEME-BF4

\section{Raman Spectroscopy of RuCl$\textsubscript{3}$ with in-situ Electrolyte Gating}
Intercalated and unintercalated RuCl
Conclusion that in plane lattice is unperturbed

\chapter{X-Ray Diffraction}

Possible references:
%Kim, H.-S., & Kee, H.-Y. (2015). Crystal structure and magnetism in alpha-RuCl3: an ab-initio study, 155143, 1–10. https://doi.org/10.1103/PhysRevB.93.155143
Discusses different crystal structures

%Johnson, R. D., Williams, S. C., Haghighirad, A. A., Singleton, J., Zapf, V., Manuel, P., … Coldea, R. (2015). Monoclinic crystal structure of α-RuCl3 and the zigzag antiferromagnetic ground state. Physical Review B - Condensed Matter and Materials Physics, 92(23). https://doi.org/10.1103/PhysRevB.92.235119
Talks about stacking faults

\chapter{Conclusion}

This dissertation investigated \rucl doped by electrolyte biasing and found that, contrary to predictions, \rucl remained highly insulating. Further, Raman spectra for electrolyte-biased \rucl identified a hysteretic transition between two visually distinct states driven by electrolyte bias voltage. Finally, x-ray diffraction measurements showed these distinct states do not result from the cations in the electrolyte physically interacting with the \ruclnospace . In this concluding section, I review the results of these measurements and identify possible frameworks that could explain the data. I then suggest measurements that could further clarify the nature of electrolyte-biased \ruclnospace .

\section{Hypotheses}

\subsection{Unlikely explanations}
\begin{itemize}
\item \textbf{Electrochemistry:} In an electrolyte gating experiment, the first confounding effect that must be addressed is electrochemistry. However, there are several observations that rule out electrochemistry as the primary explanation of the behavior of electrolyte biased \ruclnospace . Measurements are made only within the electrochemical stability window of the electrolyte. Gate current remains low during measurements and, after an increase in gate voltage, there is a transient gate current that decays to an equilibrium gate current. The transition between the two states with distinct Raman spectra is repeatable, ruling out an irreversible chemical interaction. Finally, x-ray diffraction measurements show that the electrolyte does not penetrate the \rucl lattice, indicating that only the top and sides of the \rucl are available for chemical interaction. However, the presence of substrate features in the Raman spectrum indicates that the excitation characterizes the entire flake, which is dominated by the chemically-isolated bulk.

\item \textbf{Intercalation:} Only cations intercalate in \ruclnospace , given the planes of chlorine atoms present at each interface. However, it seems unreasonable to expect a cation as large as DEME+ to fit between the layers. Additionally, x-ray diffraction measurements show the interlayer separation remains a constant 5.7 \AA{} and does not change as a function of gate voltage. The layers do not move and there is no room for DEME+ to fit. Intercalation and subsequent exfoliation of a single top layer is unlikely (there is no reason to suspect only the top layer intercalates when previous measurements show uniform intercalation) and cannot explain the behavior of the bulk, which dominates the Raman signal. However, intercalation of only protons, which may not be observable by x-ray diffraction, cannot be ruled out.

\item \textbf{Stress caused by electrostatics:} \rucl in these measurements is a thin, charged flake immersed in a biased electrolyte. Accordingly, the electric fields may subject the flake to stress that causes changes in its Raman spectrum. Further, this stress could be consistent with the mottled appearance of the flakes. However, if the stress is electrostatic, then it should appear at both positive and negative electrolyte bias voltage. Instead, the response of the material is asymmetric with respect to gate voltage.

\item \textbf{Charge localization due to lattice deformation:} The electrolyte bias adds charge to \ruclnospace , but the material could remain insulating because the charge is localized by lattice deformations and cannot participate in transport. However, charge localization by a lattice distortion should be continuously dependent on the amount of charge added, and there are only sharp changes in the Raman spectrum as a function of electrolyte bias. Additionally, lattice deformations associated with charge defects, like those in color centers in alkali halides, show the presence of new, broad peaks associated with the lattice deformations \cite{Porto2006,Benedek1967,Moller1972}. Such features are not present in either Raman spectrum for electrolyte-biased \ruclnospace .
\end{itemize}

\subsection{The case for a phase transition}

The above discussion shows that many of the experimental artifacts or less interesting explanations for the behavior of electrolyte-biased \rucl can be ruled out. What remains is the possibility of structural or electronic phase transition. The transition between states having different Raman spectra is hysteretic and has an interface, showing the transition is first order. Further, there are several different \rucl structures that are nearly degenerate in energy \cite{Kim2016} that gating could drive transitions between. Additionally, there is evidence that charge doping in \rucl creates charge order \cite{Koitzsch2017}. If the charge ordering couples to the lattice, then an electronic transition could explain the observed behavior. Future work should explore these possibilities.

\section{Future work}

Future investigation of electrolyte-biased \rucl should focus on x-ray diffraction measurements. Brighter x-rays from a synchrotron light source would be able to resolve scattering planes with a component parallel to the c axis, and therefore better characterize the structure of \rucl in either state. Further, diffuse x-ray scattering, which measures correlations in electron density across unit cells, could characterize the role of stress in the transition.

If additional x-ray measurements do not show a change in the structure of \ruclnospace , then the possibility of an electronic transition should be considered. Charge-ordered materials have different transport properties depending on the orientation of the stripes, so if the transport properties of electrolyte-biased \rucl are anisotropic, then the transition may be structural instead of electronic.

Another possibility for investigating doped \rucl comes from creating van der Waals heterostructures with other 2D materials. Recent work shows that when \rucl is put in contact with graphene, the graphene acquires excess conductivity that can only be explained by the \rucl layers closest to the graphene becoming conducting \cite{Zhou2018b}. DFT calculations also predict superconductivity at the interface between \rucl and graphene under certain conditions \cite{Biswas2019}. While this system is new and more complex, it could offer another route to studying doping in \rucl.

\begin{appendices}

\chapter{Procedures}

I used the following procedures during my time at Stanford. With minor adjustments, they can be used to make transport devices for just about any exfoliated two-dimensional material.

\section{Producing samples}

\subsubsection{Cleaving wafers for exfoliation}
\begin{enumerate}
	\item Find a new 90 nm SiO2 wafer and score it using a scribe along one of the crystal axes.
	\item Prop the scored portion of the wafer on a broken wirebonder tip.
	\item Press on either side of the score using a closed pair of tweezers such that the wafer cleaves along the crystal axis to which the score is parallel.
	\item Repeat until you have several 10 mm x 10 mm chips.
	\item Remove dust from chips using the air gun.
\end{enumerate}

\subsubsection{Exfoliation of graphene}
\begin{enumerate}
	\item Take a 6” piece of high gloss scotch tape (comes in a red box) and fold the ends over to use as handles. Place sticky side-up on the lab bench.
	\item Place the graphite crystal exfoliation-side down near the edge of the sticky side of the tape. Orient any tears or imperfections on the graphite crystal perpendicular to the long direction of the tape. Press GENTLY with closed tweezers to make good contact between the tape and the crystal.
	\item Using the handle, raise the tape and adhered graphite until it’s approximately perpendicular to the lab bench. GENTLY press the edge of the adhered graphite crystal closest to the handle to ensure the edge is well adhered, then use the handle and your tweezers to peel the graphite crystal from the tape. The crystal should leave behind a shiny patch of graphene on the tape. Return the crystal to its case.
	\item Using the handles, fold the tape to copy the exfoliated patch of graphene onto another section of tape.
	\item Make copies from the copied section from step 4 onto clean areas of the tape until the copy from step 4 is slightly hazy. Return the tape to the lab bench sticky side-up.
	\item Place a clean 90 nm SiO2 chip onto a glass slide. Select a hazy region of the copy from step 4 and place this region face down onto the chip. Avoid creating bubbles. Apply pressure with your thumb. Trim excess tape from the glass slide.
	\item Heat the glass slide/chip/tape complex on hot plate at 100C for 2 minutes. Then cool to room temperature (using an air gun or just by setting on the lab bench).
	\item Peel the tape from the glass slide and the chip at a high angle (as close to anti-parallel) as possible. Immobilized the chip using tweezers. Make sure to peel slowly (should take about 60 seconds).
	\item Store exfoliated flakes in dry box.
\end{enumerate}

\subsubsection{Exfoliation of boron nitride}
\begin{enumerate}	
	\item Take a 6” piece of high gloss scotch tape (comes in a red box) and fold the ends over to use as handles. Place sticky side-up on the lab bench.
	\item Sprinkle a small number of hBN crystals onto the tape.
	\item Repeatedly exfoliate the crystals onto the tape until the tape has a uniform covering of sparkly crystals. The tape should look like it’s coated with rainbow glitter (or makeup).
	\item Place several clean 90 nm SiO2 chips onto a glass slide. Press the tape over the chips, applying pressure and avoiding bubbles. The relative position of the tape and the chips is not important for this step. Trim the excess tape.
	\item Heat the glass slide/chip/tape complex on hot plate at 100C for 2 minutes. Then cool to room temperature (using an air gun or just by setting on the lab bench).
	\item Peel the tape from the glass slide and the chip at a high angle (as close to anti-parallel) as possible. Immobilize the chip using tweezers. Make sure to peel slowly (should take about 60 seconds).
	\item Store exfoliated flakes in dry box.
\end{enumerate}

\subsubsection{Exfoliation of ruthenium chloride}
\begin{itemize}
	\item Follow the same instructions as for boron nitride.
\end{itemize}

\subsubsection{Finding and naming flakes}
\begin{enumerate}
	\item Orient the chips so that a cleaved corner is in the bottom right-hand side of the chip.
	\item Look for flakes under the microscope. I’ve found that looking at 10x and changing the F-stop to the smallest setting helps the flakes show up.
	\item Save a 10x and 50x high contrast image of boron nitride flakes using the following naming convention: ``dsbjr-bn-DATE OF FIRST EXFOLIATION IN THIS SERIES-CHIP NUMBER FLAKE LETTER-X LOCATION Y LOCATION-MAGNIFICATION''. For example, the 10x image of the second hBN flake on chip 3 in the exfoliation series begun on 1 April 2016 located two millimeters from the left edge of the chip and one millimeter from the top of the chip would have the name ``dsbjr-bn-2016-04-01-3b-xp2ym1-10x.''
	\item Save a 10x and 50x high contrast image of graphene flakes using the following naming convention: ``dsbjr-gph-DATE OF FIRST EXFOLIATION IN THIS SERIES-NUMBER OF LAYERS (mono/bi/tri)-CHIP NUMBER FLAKE LETTER-X LOCATION Y LOCATION-MAGNIFICATION''. For example, the 50x image of the third flake that is a bilayer on chip 9 in the exfoliation series begun on 1 April 2016 located three millimeters from the right edge of the chip and two millimeters from the bottom of the chip would have the name ``dsbjr-gph-2016-04-01-bi-9c-xm3yp2-50x.''
\end{enumerate}

\subsubsection{Making PPC/GEL-PAK stamps}
\begin{enumerate}
	\item If not available, mix up some PPC solution in anisole. Dissolve Aldrich 389021-25G poly(propylene carbonate) average mn 50k at 11 wt% in anisole. Use a stirrer bar to mix on a hotplate at room temperature. Note that this dissolution is slow and may require over 24 hours.
	\item Spin polypropylene carbonate (11 wt% in anisole) onto 5 mm x 5 mm diced chips at 1500 rpm for 60 seconds. Bake the chips at 80C for 5 minutes or until the edge beads dry.
	\item Clean a glass slide using acetone and isopropyl alcohol.
	\item Cut a small section of gel-pak material and place it onto the cleaned slide near the upper-left corner.
	\item Examine the gel-pak material under the microscope and, using a razor, cut away any areas that are poorly adhered to the glass slide or that have some surface contamination. It’s okay if the remaining gel-pak is small (in fact, a small stamp is probably best). Cover the gel-pak bearing slides to protect from dust.
	\item Warm up the UV-ozone lamp for five minutes.
	\item UV-ozone the gel-pak slide for five minutes. During this time, use a razor to lift the edges of the PPC from a chip.
	\item Remove the gel-pak slide from the UV-ozone. Using tweezers, peel the edge beads of the PPC from the chip and gently lay the clear window of PPC over the gel-pak stamp.
	\item Bake the stamp at 80C for five minutes. Keep covered.
	\item Remove and examine under the microscope. Discard any undesirable stamps.
\end{enumerate}

\subsubsection{Preparing device substrates}
\begin{enumerate}
	\item Use 300 nm SiO2 chips with distance markings.
	\item Sequentially rinse the chips in acetone, isopropyl alcohol, methanol, and deionized water, blowing dry between each solvent.
	\item Anneal the chips in open air at 500C for one hour.
\end{enumerate}

\subsubsection{Stacking graphene and boron nitride}
\begin{enumerate}
	\item Find and select all exfoliated flakes. Compare their sizes to be sure the device will behave as expected (use transparent layers in gimp to select orientations and stacking order).
	\item Place the hBN-bearing chip on the transfer station and heat to 35C. Navigate to the desired flake.
	\item Mount the desired stamp at a slight angle (maybe 1-2 degrees).
	\item Using the coarse and fine focus knobs, roll the stamp into contact with the hBN flake. Roll on near a corner of the stamp, but not at the exact corner. Minimizing the contact area between the stamp and the chip helps avoid delamination.
	\item Peel the stamp from the chip. Note that some group members have experienced delamination when peeling. If your stamp appears to be stuck, you can try to snatch it off by turning the coarse focus to pull it up quickly. Higher temperatures make the stamp less sticky, but also make delamination more likely. Use your discretion and judgement here.
	\item Unmount the stamp and heat on a hotplate for 1 minute at 80C. Take a photo of the flake on the stamp. Label appropriately and save in the dropbox folder.
	\item Repeat the above steps for the subsequent graphene, hBN, and optional graphite backgate. Take photos at each stage.
	\item Place the substrate chip (300 nm oxide) on the transfer station and heat to 80C. SLOWLY roll the assembled stack on the stamp onto the substrate. Remove and discard the stamp.
	\item Anneal the stack/substrate in 10:1 oxygen:argon at 500C for 1 hour.
	\item The direction you roll the stamp onto the flake can be important. I think that sides of the flake with greater edge length can be easier to start the pick-up with.
	\item Graphene flakes with long/thin tails can be a problem. Sometimes during the pick-up these tails can twist and fold on themselves, ruining your perfect device. Try to roll the stamp onto the flake such that these tails come up first – wider graphene areas or multiple layer graphene/thin graphite is more rigid and not subject to the same problem.
	\item To reduce the chance of delamination, I try to minimize the amount of time the stamp is in contact with the chip during peel-off. What I end up doing is rolling on slowly, then rolling off slowly, only to the point that flake has been removed. After the flake is removed but there’s still stamp in contact with the chip, I snatch that part off like mentioned above. Any wrinkles introduced by the snatch typically come out during the anneal at 80C.
	\item If you’re delaminating, the best thing to do is put the stamp back in contact with the surface and try to go slowly. That said, I haven’t really been successful rescuing a stack when the stamp delaminates. It’s in your best interest not to delaminate in the first place.
\end{enumerate}

\subsubsection{Making open graphene samples}	
It’s hard (impossible?) to pick up monolayer graphene with a PPC stamp. So if you want graphene on the top of a device (for scanning probe measurements), you’ll need to flip over your stack. Do this by depositing the hBN-graphene stack onto another film, and make a new stamp using that film to deposit the device on a substrate.
\begin{enumerate}
	\item On a clean Si substrate (5mm x 5mm x 500 um), spin 11 wt% PPC in anisole at 1500 rpm for 1 min, then bake at 80C for a few minutes until the edge beads are dry (they’ll look wrinkly).
	\item One the same substrate, spin 950/A2 PMMA at 4000 rpm for 1 min, then bake at 80C for a few minutes until the film is dry.
	\item Pick up hBN-graphene as described previously.
	\item Place the PMMA on PPC substrate on the transfer station and heat to 60C.
	\item Gently deposit the hBN-graphene stack onto the PMMA on PPC film.
	\item Remove the PMMA on PPC film substrate and scratch along the edges with a razor to lift the edge of the film on all four sides.
	\item Cut a small square of the blue tape slightly larger than the 5 mm x 5 mm substrate. Fold it half in both directions and cut out the center to form a window larger than the flat area of the stamp.
	\item Press the square of tape onto the PMMA on PPC film substrate, making sure the hole in the tape is centered above the flat area of the film and that the tape adheres to the scratched film along the edges.
	\item Peel the tape and the PMMA on PPC film off the substrate. Take care not to wrinkle or crease the film, as it now hosts your precious heterostructure.
	\item Prepare a stamp though the UV ozone step, then lay the film onto the stamp and back at 80C for a few minutes.
	\item Verify the PMMA on PPC film is adhered to the stamp using an optical microscope, and that the heterostructure isn’t caught in a wrinkle or crease.
	\item Place the final substrate on the transfer stage and heat to 160C.
	\item Bring the PMMA on PPC stamp into contact with the substrate for deposition. The higher temperature is necessary to get the PMMA film to adhere.
	\item Once the heterostructure is deposited and the PMMA film adhered, rinse in acetone to remove the PPC, and then anneal in Ar/0¬2 to clean.
\end{enumerate}

\section{Designing devices}

\subsubsection{Atomic force microscopy}
\begin{enumerate}
	\item Take detailed AFMs of the stack. Make sure to take an AFM of the intended active area and the edges to measure dielectric (hbN flake) thickness. Record both topography and phase images. High contrast in phase images indicates that features are on the surface rather than buried in the heterostructure (e.g., leftover ppc residue).
	\item Flatten and correct the images using gwyddion. I typically remove horizontal scars and then use the three point leveling function with points of several pixel radius as close to the graphene as I can. I try to avoid the median height leveling function as it introduces artifacts into the image. You can use the line leveling function if you want to remove whole lines that look out of place. Then change the scale such that there’s contrast where the graphene is.
\end{enumerate}

\subsubsection{Device Design}
Overall Summary: Decide where to put the device based on optical images and AFM. Define an inner region where the sensitive writes will go, and an outer region where we’ll connect the traces from the inner region to bond pads. Drop alignment marks in the inner region to help with the alignment of sensitive writes. Then define a topgate (if desired), an etch, and patterns for ohmic contacts.

\begin{enumerate}
	\item Take images at 10x, 20x, 50x, 100x, and 150x (you only need the 10x and 50x, but the remainder are good to have). Make sure the 50x includes four alignment marks that circumscribe the heterostructure.
	\item Half the size of the images (2048 to 1024), as it’s hard for design cad to handle large images.
	\item Open the 50x and the processed/leveled AFM image as layers in gimp. Scale/translate/rotate the AFM image until it matches the features of the 50x image. Bubbles that remain in the heterostructure are good markers to use to make sure everything is lined up properly.
	\item Flatten the image and export it.
	\item In design cad, open a new file and load the 10x and 50x+AFM images. Scale them as needed (typically the 10x is scaled by 6 and the 50x by 1.2). Align the images as necessary. Note that 1 unit in design cad is 1 um.
	\item In each of layers 1 through 4, place polygons defining the alignment marks on the 50x image. There should be a 15 um x 15 um dashed box around each of the marks, and then a solid polygon that matches the outline of the alignment mark inside this box.
	\item In layer 6, define your beam dump. I typically draw another box over one of the alignment marks.
	\item Select a region of the heterostructure to use for your device. It should be as free of defects as possible (bubbles, specks, etc).
	\item In approximately a 40-50 um x 40-50 um region around the active area, in layer 6, define four new alignment marks (5 um x 500 nm crosses) in the corners.  You may want four more if you’re doing a topgate. Draw a box around this region as a guide to the eye so you can know where to put polygons for the inner write.
	\item Define a topgate (if desired): In layer 7, use the rectangle tool to define the active region. Rotate as necessary and record the angle. Then add a lead ending in an L-shape that goes to the edge of the inner write region.
	\item Define an etch pattern. In layer 8:
	\begin{enumerate}
		\item Define your etch pattern by making two smaller rectangles (or more if you’re ambitious), rotating them to match the topgate, and positioning them as voltage contacts. Then, define another rectangle at 90 degrees relative to the voltage contacts for the current contacts. All these rectangles should extend beyond the bounds of the topgate. Note that the current lead rectangle should fit entirely within the topgate rectangle. Add all these rectangles together using the Boolean add function to get one polygon.
		\item Given the way PMMA works as an e-beam resist, we don’t actually want to write this polygon. We want to write the inverse of it. To do that, we’ll need to define two other polygons, each of which forms a half clamshell around the active area of the device and cuts through the center of the polygon defined in 11a. Make one of these clamshells, then subtract from it the polygon defined in 11a, leaving a clamshell with one edge that’s half of the polygon. Move the copied polygon such that it lines up with the edges in the first clamshell. Then draw another clamshell and subtract the new polygon. The two clamshell polygons form your etch pattern.
		\item Increase the perimeter of what will eventually be the graphene you make ohmic contact to. 
		\item Make sure you remove all the graphene from your device to avoid shorts. If necessary, in layer 9, draw polygons that will cover the rest of the graphene.
	\end{enumerate}
	\item Define an ohmic pattern.
	\begin{enumerate}
		\item In layer 10, draw quadrilaterals that will be where you evaporate gold to make initial contact to the graphene (and graphite backgate if present).
		\item In layer 11, draw leads connecting these quadrilaterals to the ends L-shaped ends that approach the edge of your inner write region. Make sure these are separate polygons.
	\end{enumerate}
	\item Define bond pads.
	\begin{enumerate}
		\item In layer 13, define bond pads around the edges of the device. I typically fit these on the 10x image and make them 100 um x 200 um, with 100 um between them.
		\item In layer 12, define traces that overlap the bond pads and the L-shaped leads in the inner region.
	\end{enumerate}
\end{enumerate}

\subsubsection{Design principles}
\begin{enumerate}
	\item Avoid bumps or specks in the AFM image. If you must, it’s probably okay to put an impurity in the center of a hall bar because we care mostly about the edge states. But make sure there’s nothing on your edges.
	\item Make sure there is at least 1 um of graphene perimeter underneath your ohmic contacts. 
	\item The etch is design to eat hBN, not graphene or graphite. So if you have a thick bit of graphene that you’re trying to etch, all you’ll end up doing is exposing it. The exposed graphite can make things awkward when you define your bond pads as it will short them to each other. Be careful not to short your leads to each other.
	\item Expect the windows for the alignment marks to be covered in gold when you’re doing your ohmic deposition. Make sure there’s enough area between them and the closest lead that two leads won’t get shorted together. In fact, sometimes it’s best to put your ohmic leads through/over an alignment mark windows to avoid shorting.
	\item Hall bar width should be at least 1 um. Shorter than that and you start to run into edge effects. The minimum length is probably about 4 um.
	\item The best resolution you’ll get with the NovaNano is probably around 200 nm. Don’t try to define any features smaller than that, and make sure that polygons in successive layers won’t short if you’re shifted by something of that order.
	\item Use every square nanometer of space on your active area that you can.
	\item Avoid sending any traces over bumps or obvious imperfections.
\end{enumerate}

\section{Fabrication}

\subsubsection{E-beam lithography tips}
\begin{enumerate}
	\item Use a beam dump (a random polygon written somewhere in the write field you don’t care about). Write this as the first polygon in any pattern. I think there might be some random motion that happens when the beam first starts. But I’m not sure – this could be superstition.
	\item Write from the smallest features in a pattern to the largest. For example, in the etch design laid out above, you would want to write the beam dump first, then layer 8, then layer 9, if necessary. For the ohmic contacts/bond pads (typically done in one write), you’d want to write the beam dump first, then the small quadrilaterals (layer 10), then the inner traces (layer 11), then the outer traces (layer 12), and finally the bond pads (layer 13) at the end. Each successive layer is less vulnerable to lateral shifts, so writing them in that order helps make sure things are connected at the very end.
	\item Write at as high a magnification as possible. I’m not sure if this is superstition or it just makes me feel better, but when I added alignment marks and worked at around 1000x – 2000x for my etch and inner ohmics I got better results.
	\item Any time you change the beam (either spot size or magnification), redo an alignment step. This can help take care of transient beam changes that emerge during the write.
	\item For small features (small etch pattern and small ohmics), I use center to center distance and line spacing at 20 nm / 20 nm. This spacing is the most accurate. For larger features (i.e., outer traces and bond pads), I use 100 nm / 100 nm. Otherwise your writes take forever.
	\item Write small features at spot size 3, and larger features at whatever spot size has the highest current. Not sure if this is just superstition, but it’s what I was taught.
	\item Don’t measure beam current on an alignment mark (can be up to 50% lower than you’ll get when writing on your device). I measure beam current in a Faraday cup on the SEM mount. You may read a different specimen current during writing, but it will always be less than what you’ve measured in the cup. I haven’t had any problems measuring beam current that way.
	\item Write each layer at about 50x less than the maximum magnification NPGS allows. By increasing the write field (lowering the magnification), you can avoid errors NPGS throws about polygons being outside the write area.
	\item Check the relative displacement of your markers during the alignment step. For larger patterns, I would expect some kind of self-consistent translation and rotation, rather than some arbitrary displacement. For small patterns, at higher magnification there may be some arbitrary displacements. I’m not sure why, but I suspect that maybe because the alignment in the first step is necessarily bad (the alignment marks are nebulously defined and hard to align to), maybe there’s some shift in where the alignment marks are dropped.
	\item Do a test write each time you go to do a write. This can show you your errors. Pay attention to how fast polygons are written, and which polygons actually show up in the write field. I would expect the alignment marks and topgate to be written in less than 1 second, the etch to be written in up to 10 seconds depending on how big your second etch (layer 9) is, and the bond pads and ohmics to take up to 15 minutes. Something less than this may indicate you have your beam currents set incorrectly.
	\item Make sure the beam is not exposing your sample when you drive to the origin of your pattern (which should be in the center of the pattern to make sure you’re writing at the highest magnification). Watch the specimen current to confirm – it should be less than 10 pA if you aren’t writing. You can do this by switching the beam blanker to EXT when you’re staring at alignment mark with the detector unpaused. The screen should go dark and return only when you switch the beam blanker back to ON. Navigate to the center of your pattern with the beam blanker on EXT. Bear in mind that if you have the beam paused and set the scan mode to external in FEI without the beam blanker on EXT (and sometimes even if the beam blanker is on EXT!), you can expose right in the center of your pattern. Don’t do this – you’ll cross link stuff and be pissed off later.
	\item Common NPGS errors:
	\begin{enumerate}
		\item PG(6) – caused by intersecting polygons. You can fix this in-situ by opening design cad lite, drawing over your offending polygon, and deleting your old one.
		\item PG(-1) – caused by the calculated alignment transformation throwing polygons outside your write field. Fix by decreasing your magnification to increase your write field size.
		\item If you put your alignment polygons and pattern polygons in the same .dc2 file, make sure you have four alignment marks. Even if the marks aren’t in the correct shape or in the correct place, NPGS will interpret the first four layers as alignment marks. If you need fewer than four, make a separate alignment .dc2 file.
		\item Invalid transformation matrix – caused by the alignment transformation throwing your alignment mark window outside the write field. Fix by reducing your magnification.
	\end{enumerate}
\end{enumerate}

\subsubsection{Alignment marks}
\begin{enumerate}
	\item Bake the heterostructure for 2 minutes at 180C to remove any water from the surface (may be superstition).
	\item Spin 950/A5 PMMA on the stack at 4K RPM.
	\item Bake at 180C for 20 minutes.
	\item Condition the column.(let it run at highest current at 30 kV). This may also be superstition, but I think charging the column can rearrange whatever dust or junk is on there and minimize the drift during actual writing.
	\item Set beam parameters to spot size 3, 10 kV.
	\item Using the large alignment marks (100 um separation) as alignment markers, write the beam dump and the alignment marks for the inner write field.
	\item Develop by dunking in 3:1 Water:Isopropyl Alcohol for 1 minute and blowing dry. Do this as gently as possible. Examine under microscope to make sure you have what you’re expecting.
	\item UV ozone for 90 seconds or so before putting in the evaporator (may be superstition).
	\item Evaporate 5 nm Ti/some amount of gold in the KJL or the Moore e-beam evaporator. I don’t think you have to go that thick with these. Probably 50 nm of gold is okay – you just need to be able to see what you put down underneath a film of PMMA in the SEM.
	\item Liftoff by submerging in acetone. This can be as short as 15 minutes or it can take up to an hour, but it shouldn’t take much longer than that.
\end{enumerate}

\subsubsection{Topgate}
\begin{enumerate}
	\item Bake the heterostructure for 2 minutes at 180C to remove any water from the surface (may be superstition).
	\item Spin 950/A5 PMMA on the stack at 4K RPM.
	\item Bake at 180C for 20 minutes.
	\item Condition the column (let it run at highest current at 30 kV). This may also be superstition, but I think charging the column can rearrange whatever dust or junk is on there and minimize the drift during actual writing.
	\item Set beam parameters to spot size 3, 10 kV.
	\item Using the large alignment marks (100 um separation) as alignment markers, write the beam dump and the alignment marks for the inner write field.
	\item Develop by dunking in 3:1 Water:Isopropyl Alcohol for 1 minute and blowing dry. Do this as gently as possible. Examine under microscope to make sure you have what you’re expecting.
	\item UV ozone for 90 seconds or so before putting in the evaporator (may be superstition).
	\item Evaporate Ti/Au in e-beam evaporator. 5 nm of Cr and an amount of Au equal to your tallest feature (or the entire heterostructure) plus 20 nm.
	\item Liftoff by submerging in acetone. This can be as short as 15 minutes or it can take up to an hour, but it shouldn’t take much longer than that.
\end{enumerate}

\subsubsection{Etch}
\begin{enumerate}
	\item Bake at 180C for 2 minutes.
	\item Spin 950/A4 PMMA on the stack at 4K RPM.
	\item Bake at 180C for 20 minutes.
	\item Set beam parameters to spot size 3, 30 kV.
	\item Write using NPGS.
	\item Develop by dunking in 3:1 Water:Isopropyl Alcohol for 1 minute and blowing dry. Examine under microscope.
	\item Write using the dgg hBN etch in the oxford plasma pro. Use cycles of two minutes of etching followed by one minute of cooldown. You eat about 20 nm of hBN in a minute. Typically 4 cycles of this etch (8 minutes total) is enough to eat through any reasonable heterostructure. I know the resist can take up to 12 minutes of etching. I don’t know much about going beyond this, but at 12 minutes the resist is pretty thin so I wouldn’t recommend doing it.
\end{enumerate}

\subsubsection{Ohmics}
\begin{enumerate}
	\item Bake at 180C for 2 minutes.
	\item Spin 950/A5 PMMA on the stack at 4K RPM
	\item Bake at 180C for 20 minutes.
	\item Set beam parameters to spot size 3, 10 kV.
	\item Write using NPGS.
	\item Develop by dunking in 3:1 Water:Isopropyl Alcohol for 1 minute and blowing dry. Examine under microscope.
	\item UV ozone for 30 seconds (may be superstition).
	\item Evaporate Cr/Au in e-beam evaporator. 5 nm of Cr and an amount of Au equal to your tallest feature (or the entire heterostructure) plus 20 nm. Pressure is critical here – do an overnight pumpdown and shoot off a bunch of Cr (10 nm) before opening the shutter to get the lowest pressure you can. Minimize the time between the Cr evaporation and the Au evaporation to minimize the amount that the Cr layer oxidizes. Shoot for the amount of time it takes the crucible to go dark plus two minutes. When heating the gold, let it outgas as it heats before opening the shutter. Evaporate Cr slowly (maybe 0.3 – 0.5 \AA /s) and Au more quickly (2 – 3 \AA /s).
\end{enumerate}

\section{Applying ionic liquid}
\begin{enumerate}
	\item Wearing gloves, dip the tip of a sterilized 250 mL pipet tip into the ionic liquid. There should be a drop of ionic liquid that clings to the outside of the pipet tip.
	\item Under a stereoscope, deposit the drop of ionic liquid onto a clean area of the substrate away from the active area of the device.
	\item Using a loop of gold wirebonding wire approximately 100 $\mu$m in diameter, pick up a small amount of the deposited drop.
	\item Press the ionic liquid in the wire loop onto the active area of the device. With practice, you can reliably deposit droplets of ionic liquid that are smaller than 100 $\mu$m in dimension.
\end{enumerate}

\section{Miscellaneous tips}
\begin{enumerate}
	\item Sometimes the anneal can spray contaminants all over your device, which sucks. To avoid this, anneal the tube and boat in open air at 1100 C for an hour. It gives off blackbody radiation, which is pretty cool looking.
	\item If you crosslink some PMMA, it’s not the end of the world. First, let it sit in acetone overnight. If that doesn’t work, you can try a high temperature oxygen anneal for a couple of hours. There is a spectrum of crosslinking, but I once crosslinked PMMA on a device and a high temp O2 anneal (500 C). for about 3 hours got rid of it.
\end{enumerate}



\include{./tex/AppendixB}

\bibliography{C:/Users/dsbjr/Documents/GitHub/Dissertation/tex/dissertation.firstdraft}

\end{document}
