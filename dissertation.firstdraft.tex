%\documentclass[12pt,twoside]{report}
\documentclass[12pt]{report}

%% Last Modification by Derrick Boone 5 February 2019. Template from Emma Pease

% note that the document can be single or double sided.  
% note that the Registrar's office now allows 10pt, 11pt, or 12pt

\usepackage{suthesis-2e}
\usepackage{caption}
\usepackage{graphicx}
\usepackage{amsmath}
\usepackage{amsthm}
\usepackage{braket}
\usepackage{url}
\usepackage{gensymb}
\usepackage{textcomp}
\usepackage{braket}
\usepackage{amsfonts}
\usepackage{geometry}
\usepackage[pdfencoding=auto]{hyperref}
% default is now online June/2016 version
%\usepackage[online]{suthesis-2e}
%\usepackage[hardcopy]{suthesis-2e}
% the following is for doing engineering theses. 
% I am definitely not sure of the wording on the signature page so check
%\usepackage[engineer]{suthesis-2e}

% one can change the default font to Times Roman but note that most
% ways of creating pdf files from latex automatically embed (which btw
% is a good idea even with the standard fonts)

    \title{Electrolyte Biasing of the Proximate Kitaev Spin Liquid $\alpha$-Ruthenium (III) Chloride}
    \author{Derrick Sherrod Boone, Jr.}
    \dept{Applied Physics}
    \principaladviser{David Goldhaber-Gordon}
    \firstreader{Ian Fisher}
    \secondreader{Marc Kastner}
%% one can also have a \thirdreader and \fourthreader

%% note that certain departments and types of theses have other requirements
%% For instance theses in the departments of 
%% Asian Languages
%% French and Italian
%% Spanish and Portuguese
%% need to define the \dept, \dualthesis, and the actual language
%\dualthesis
%\languagemajor{Chinese}
%% 
%% Those for Graduate Program in Humanities need to define 
%\humanitiesthesis
%\jointprogram{Arts and Crafts}
%% 
%% For submission to a committee or program (no department)
% \committeethesis
% \programthesis
%%
%% For School of Education or Business or Law
% \educationthesis
% \businessthesis
% \lawthesis  (law actually isn't listed in the official documents, 2013/1014)

%DEFINE NEW COMMANDS
\newcommand{\rucl}{RuCl\textsubscript{3} }
\newcommand{\ruclnospace}{RuCl\textsubscript{3}}
\newtheorem{assumption}{Assumption}
\newcommand{\percmsq}{e\textsuperscript{-}/cm\textsuperscript{2}}
\newcommand{\pone}{P\textsubscript{1}}
\newcommand{\pzero}{P\textsubscript{0}}

\bibliographystyle{unsrt}

\begin{document}

% for a variety of reasons this is an all in one document; however,
% when actually doing the thesis it is strongly recommended that each
% chapter be in a separate file and use \include to include in the
% main file.

%% the \beforepreface command produces the title page
%% in the online version it skips the copyright (page 2) and signature (page 3) pages 
%% in the non-online version these would be included
    \beforepreface


%% Abstract can be any number of pages
    \prefacesection{Abstract}
A quantum spin liquid (QSL) is an exotic phase of matter identified by interacting spins that do not develop long-range magnetic order down to T = 0 K. QSLs have topological order and are therefore relevant to storing and processing quantum information. Additionally, hole-doped QSLs may be related to high temperature superconductivity, making the study of this phase relevant to a broad section of condensed matter physics.

Recently, the two-dimensional material and spin-assisted Mott insulator $\alpha$-ruthenium (III) chloride (\ruclnospace) has attracted research interest because of its relationship to spin liquids. Localized electrons in \rucl have anisotropic Ising interactions that are described by the Kitaev model – a theoretical type of spin liquid that is exactly solvable. The low-temperature antiferromagnetic order in \rucl can be suppressed by an in-plane magnetic field to create a field-induced QSL, identifying \rucl as a proximate Kitaev spin liquid.

Because \rucl is both a proximate Kitaev spin liquid and a Mott insulator, charge doping may create interesting electronic phases. Previous attempts to chemically dope \rucl have been unsuccessful. However, previously used doping techniques either disordered the lattice or added an unmeasured amount of charge, making this negative result hard to interpret. Charge doping by electrolyte biasing would eliminate these problems, and improve on previous measurements.

I use electronic transport measurements and Raman spectroscopy under electrolyte bias to show that, upon doping, \rucl undergoes a first order phase transition with only small changes in conductivity. X-ray diffraction of electrolyte-biased \rucl rules out intercalation being the cause of this phase transition, and confirms the assumption that electrochemical interactions are negligible. I conclude that this phase transition prevents \rucl from being used to study doped spin liquid physics and suggest further measurements to characterize the newly identified phase.

%% one can also have a prefacesection that is a Preface instead of
%% Acknowledgements.   The thematic purpose is the same (thanks).
    \prefacesection{Acknowledgements}
        Here is where I thank people and include some quotes.

%% afterpreface produces a table of contents and any other tables
%% wanted. At the end pagenumbering changes from roman to arabic and
%% is restarted
    \afterpreface
 
 
\chapter{Introduction}
Condensed matter physics is the rigorous study of what happens when a large number of cold atoms at high density are allowed to interact. It tells us why and at what temperature water freezes, why magnets attract some materials and not others, why glass is clear, why metal is shiny, and many other things. It is the branch of physics that reveals the richness of our physical world.

\section{Phases of matter}
Many materials that differ in their constituents and microscopic structure have similar bulk properties. For example, although water and mercury at ambient conditions have dramatically different densities and electrical conductivities, they are both nearly incompressible and deform continuously when a shear stress is applied. We capture these similarities by saying water and mercury are both in the liquid phase\footnote{Depending on the context, there may be a difference between a \textit{phase} of matter and and \textit{state} of matter. I will use phase in this dissertation as it seems to apply more generally.}. Phases of matter arise not just because of constituent particles, but also by the way those constituents are arranged.

A phase of matter has uniform equilibrium thermodynamic properties (density, magentization, etc.) and is defined by these properties being analytic functions of the thermodynamic parameters (e.g., temperature, pressure) \cite{Pathria2011}. Therefore, the properties of matter in static equilibrium in a given phase are the same for all space, and these properties are smooth functions of the parameters. For example, liquid water at a uniform temperature has the same density everywhere, and when it is heated by a small amount, its density decreases by a corresponding small amount. Phases of matter are separated by phase transitions, where the thermodynamic properties (or their derivatives with respect to a parameter) are no longer continuous\footnote{Infinite order phase transitions are a theoretical exception. See \cite{Costin1990}}. For example, when liquid water boils at ambient pressure, its temperature remains the same, but its density decreases discontinuously by a factor of $10^6$.

We can also use Landau theory \cite{Landau1937} to describe phases of matter by the symmetries of their Hamiltonian\footnote{A symmetry is an operation which leaves the Hamiltonian of the system invariant. For example, the Hamiltonian of a particle in free space $H = \Sigma_{i} \frac{p_{i}^{2}}{2m}$ is invariant under spatial translation $x \rightarrow x + a$}, and the phase transitions between them as the breaking or recovery of those symmetries. For example, when a liquid freezes into a solid crystal, the continuous translational symmetry of the liquid phase becomes a discrete translational symmetry as the molecules in the liquid assemble themselves into a liquid. Another example is a material transitioning from a non-magnetic to ferromagnetic phase. When the magnetic moments of the material align, it gains an overall macroscopic magnetization, breaking rotational symmetry. The following table lists some common phases and the symmetries they break \cite{Chaikin1995}.

\begin{center}
\resizebox{\textwidth}{!}{
	\begin{tabular}{l | l | l | l | l | l | l}
		\hline
		\hline
		\textbf{Phase} & Fluid & Nematic & Smectic-A & Crystal & Heisenberg Magnet & Superfluid \\ \hline
		\textbf{Broken Symmetry} & None & Rotational & 1D Translation & 3D Translation & Rotational & Phase\\ \hline \hline
	\end{tabular}
	}
	\captionof{table}{Selected phases and their associated broken symmetries}\label{tbl:nicetabelesstable}
\end{center}
		
However, in addition to the above examples, there are kinds of matter which maintain a single set of symmetries but nonetheless have phases separated by phase transitions. These kinds of matter possess topological order - a type of order that can define a phase of matter just like symmetry can \cite{Wen1990}.

\section{Topological Phases}

Topological order is a property of quantum systems that have both long-range entanglement and large ground state degeneracy. In these systems, there is no local order parameter like density or magnetization. Instead, there is a global topological invariant that changes discontinuously between phases \cite{Wen2017}. First, we set out to understand topology by considering the properties of a simple quantum topological system - Kitaev's toric code. Using the topological concepts we learn from the toric code, we will be able to understand the topological nature of some physical systems, including quantum hall states and the eventual subject of this dissertation: the spin liquid.

\subsection{The toric code}

The following explanation draws heavily from \cite{Kitaev2003} and \cite{topOrderEdX}.

Consider a system of spin-$\frac{1}{2}$ electrons living on the edges of a square lattice. We first define two operators:

\begin{align*}
A_s&=\prod_{j \in star(s)} \sigma^{x}_{j} 		& B_{p}&=\prod_{j \in plaquette(s)} \sigma^{z}_{j}
\end{align*}


 

Fuck all this - teach topology with the Toric code.

Interacting system spins on a 2D lattice with a special hamiltonian. The ground state of this Hamiltonian is a loop gas - as long as there are no free ends, the energy of the system is the same. Therefore, any state with loops is a good state. The vertex operator changes between degenerate states. Basically, it changes the loop configuration.

If you put this lattice on a torus, then you get loops that cannot be deformed by the vertex operator smoothly into other loops. These are the loops that go across the periodic boundary. 

Things to remember: Quantum hall states have topological order - consider the conductivity an order parameter - as you change the field this parameter changes discontinuously without a change in symmetry. The chern number you can derive from the band structure (?) and is a topological order parameter that changes. Maybe I can start with the quantum hall effect?

Start with IQHE - explain it and then show how the conductivity is an order parameter. Then show how the FQHE extends the IQHE. Then show the topological excitations in the FQHE are evidence of topological order and their presence requires the ground state to have topological order. Then show that spin liquids are a state that has topological excitations, and therefore topological order - even in the absence of symmetry breaking.

Somehow topological order is connected to fractional excitations in two dimensions. Fractional quantum hall liquids always have edge excitations.

Topological order is connected to the ground state degeneracy of a system. The ground state degeneracy is somehow related to the topological order parameter.

Consider using the Toric code.

What is condensed matter, what's a spin liquid, why gate it, etc... This is going okay so far.

\chapter{Properties of \texorpdfstring{$\alpha$-Ruthenium Trichloride}{alpha-RuCl3}}
This chapter covers the structural, electronic, and magnetic properties of \rucl and serves as a reference for further discussion of the material. This chapter also contains a discussion of the Kitaev model and how \rucl realizes this model in the lab. Finally, I discuss published measurements of \rucl and how they confirm it as a proximate Kitaev spin liquid.

\section{Synthesis and Crystal Structure}

Generally, ruthenium (III) chloride (\rucl) is a brown or black solid that is a precursor for 	ruthenium chemistry. It is commonly found as a hydrate, where water molecules are incorporated into its structure. High-purity hydrated ruthenium (III) chloride is commercially available from chemical vendors like Sigma Aldrich. We will consider only anhydrous ruthenium (III) chloride (i.e., without water) in this dissertation.

Ruthenium (III) chloride has two polymorphs, called $\alpha$ and $\beta$. Both $\alpha$ and $\beta$-ruthenium (III) chloride are composed of ruthenium atoms octahedrally coordinated by chlorine. However, in the $\alpha$ polymorph the octahedra are edge sharing, while in the $\beta$ polymorph, the octahedra are face-sharing. The edge-sharing octahedra give the $\alpha$ polymorph its interesting magnetic qualities, and therefore, unless explicitly stated otherwise, \rucl will refer to the $\alpha$ polymorph.

As stated previously, the structure of \rucl 



\section{Electronic Structure}

Discuss evidence that RuCl3 is a Mott insulator. Show plots of calculated electronic structure.

\section{Magnetic Properties}

Discuss magnetic transitions - paramagnet to antiferromagnet. Stacking faults causing additional magnetic transitions.

\section{\texorpdfstring{\rucl}{RuCl3} as a Kitaev Spin Liquid}

\subsection{The Kitaev Model}

Discuss and solve the Kitaev model.

\subsection{Spin Liquid Behavior in \texorpdfstring{\rucl}{RuCl3}}

Measurements that show \rucl is a proximate Kitaev spin liquid: neutron scattering, field-induced disordered states, half-quantized thermal hall conductance, magnetic impurity doping suppressing AF interaction.

\chapter{Electrolyte Gating}

This chapter provides an introduction to electrolyte gating. We discuss the mechanism underlying gating methods, the advantages and disadvantages of gating with an electrolyte, how to choose an electrolyte, and how to design a device for gating. We conclude with a discussion of some important gating measurements.

\section{Mechanism of field effect gating}

Condensed matter physicists and electrical engineers often add charge carriers (either electrons or holes) to a material with the hope of changing its properties - increasing or decreasing conductivity, changing an insulator into a metal, or even changing a material's physical structure. The most common and least intrusive method to add carriers uses the field effect to electrostatically gate a material. The field effect adds carriers using an electric field rather than chemical substitution or intercalation. 

We can define an electrochemical potential $\mu$ that represents the amount of energy required to add an electron to a system. Using this definition

\begin{equation}
\mu = E_{F} + qV
\end{equation}

where $E_{F}$ is the Fermi energy\footnote{This is only correct at T = 0 K. At finite temperatures, $E_{F}$ should be replaced with an electrochemical potential at zero field, which reflects the broadening of the occupation function. However, for most materials, the Fermi temperature is so high that the difference is negligible.}, $q$ is the fundamental charge, and $V$ the applied electric potential. When two systems with different electrochemical potentials are brought into contact, heat and charge will flow across the interface until the electrochemical potentials are equal. We can exploit this effect to add carriers to a material. If we apply a field to a material that decreases $\mu$, then electrically connect that material to a reservoir (i.e., electrical ground), charge carriers will flow into the material until the chemical potentials are equal. We have used the field effect to increase the carrier density in our material. What is probably the most produced device in the world, the metal-oxide-semiconductor field effect transistor (MOSFET), uses a metal ``gate'' and an insulating oxide to introduce carriers in a semiconductor, controllably changing the conductivity of the semiconductor channel in the device.

\begin{centering}
\includegraphics[width=0.5\textwidth]{C:/Users/dsbjr/Documents/GitHub/Dissertation/img/MOSFET-original.png}
  \captionsetup{width=0.75\textwidth}
  \captionof{figure}[MOSFET Diagram]{Cartoon of a metal-oxide-semiconductor field effect transistor (MOSFET). Voltage $V_{g}$ applied to the metal gate electrode creates a field which polarizes the oxide dielectric and induces charge carriers in the channel. When the channel is conducting, $V_{sd}$ applied between the source and drain electrodes drives a large current.} 
  \label{fig:ElecGate-1}
\end{centering}

We can calculate the charge added to the channel by modeling the MOSFET as a parallel plate capacitor

\begin{equation}
Q = CV_{g} = \epsilon \frac{A}{d} V_{g}
\end{equation}

Where $A$ is the area of the electrode and $d$ is the thickness of the oxide, we can divide by area to find the induced charge density $n$

\begin{equation}
n = \frac{Q}{A} = \frac{\epsilon V_{g}}{d}
\end{equation}

The strongest SiO\textsubscript{2} can withstand a field strength of approximately 1 V/nm before breakdown\footnote{Breakdown occurs when being subjected to an electric field causes a dielectric to lose its insulating properties. Breakdown is caused by a combination of how long the dielectric is subjected to the field and how long the dielectric is subjected to the field.} \cite{Palumbo2019}. Using our parallel plate capacitor model, the maximum amount of charge that can be induced this way is approximately 2 x $10^{12}$ \percmsq . For specially grown high-K dielectrics, this maximum induced charge increases to about 5 x $10^{13}$ \percmsq \cite{Robertson2004}.

While these density changes are impressive, there are use cases for field effect gating that require greater changes in density, including reaching the van Hove singularity in graphene or removing all the carriers from a metal thin film. To achieve larger changes in density, we can appeal to the technique of electrolyte gating.

\begin{centering}
\includegraphics[width=0.5\textwidth]{C:/Users/dsbjr/Documents/GitHub/Dissertation/img/ElectrolyteGating-original.png}
  \captionsetup{width=0.75\textwidth}
  \captionof{figure}[Mechanism of Electrolyte Gating]{Electrolyte gating uses an electrically polarized molten salt to accumulate ions at the surface of a material of interest. This layer of ions creates an electic double layer capacitor, inducing charge into the material.} 
  \label{fig:ElecGate-2}
\end{centering}

Electrolyte gating (\ref{fig:ElecGate-2}) replaces an oxide dielectric with an electrolyte (a molten salt) consisting only of positively and negatively charged species. When a voltage is applied to the gate electrode, the applied electric field causes the charged species to migrate: negative charges accumulate at the gate and positive charges accumulate at the channel (or vice-versa, depending on the sign of the gate voltage). This polarization of charge forms an electrical double layer capacitor at the interface of the channel. To get a sense of the possible induced charge, we can use our previous equation for density, replacing $d$ by the ionic radius of the species (typically ~1 nm), $\epsilon$ by the appropriate dielectric constant, and $V_{g}$ by the maximum voltage that can be applied to the liquid. In this case we find an induced density of 5 x $10^{14}$ \percmsq , an order of magnitude higher than that possible with the best oxide dielectrics.

Changing carrier density by an additional order of magnitude beyond oxide gating is a powerful tool, but electrolyte gating also has strengths and weaknesses that must be considered when designing and executing an experiment. These are discussed in the following section.

\section{Considerations for electrolyte gating}

A successful electrolyte gating experiment involves choosing the most appropriate electrolyte and thoughtfully designing the device. We'll examine both.

\subsection{Choosing the right electrolyte}

Electrolytes are typically composed of monovalent, complex molecules. A wide variety of cations and anions are available for use. A selection of these are presented in Figure \ref{fig:ElecGate-3}.

\begin{centering}
\includegraphics[width=0.75\textwidth]{C:/Users/dsbjr/Documents/GitHub/Dissertation/img/IonicLiquids-Petach.png}
  \captionsetup{width=0.75\textwidth}
  \captionof{figure}[Structure of Electrolyte Cations]{Selected cations and anions used in anhydrous electrolytes. a. Diethylmethyl(2-methoxyethyl)ammonium, DEME\textsuperscript{+} b. 1-butyl-methylpyrrolidinium, BMPY\textsuperscript{+} c. 1-ethyl-3-methyl-imidazolium, EMI\textsuperscript{+} d. Tetrafluoroborate, BF$_{4}^{-}$ e. Tris(pentafluoroethyl)trifluorophosphate, FAP\textsuperscript{-} f. bis(trisfluoromethylsulfonyl)imide, TFSI\textsuperscript{-}. From \cite{Petach2017}.} 
  \label{fig:ElecGate-3}
\end{centering}

The most important concern when choosing an electrolyte is chemistry. Unlike metal-oxide gating, which uses a chemically inert oxide, electrolyte gating involves intimate contact between the material of interest and the (potentially) chemically active electrolyte. Chemical reactions between the material and the electrolyte will change the nature of the material and disturb the double-layer capacitance, making the gating ineffective at best and destructive at worst. 

Applying a potential across the electrolyte creates the possiblity of not just chemical, but electrochemical (i.e., reduction-oxidation) reactions that can have similar effects to those listed above. A method to characterize the electrochemistry of an electrolyte is to measure its electrochemical stability window - the range of voltages beyond which the ions in the electrolyte will begin to react at the electrodes. The electrochemical stability sets the maximum potential that can be applied across the electrolyte, and therefore the maximum induced carrier density.

The electrochemical stability window is typically measured with cyclic voltammetry - sweeping the potential across the electrolyte at a constant rate and measuring the current. This current should be constant, and deviations suggest chemistry is occurring. A cyclic voltammetry measurement for DEME-TFSI is presented in Figure \ref{fig:ElecGate-4}. As seen in the figure, the behavior of the gate current becomes substantially nonlinear around $\pm 2$ volts. Accordingly, I have assumed the electrochemical stability window for this electrolyte and  measurement to be -2 to +2 volts. The observed electrochemical stability window for DEME-BF4 is similar.

\begin{centering}
\includegraphics[width=0.5\textwidth]{C:/Users/dsbjr/Documents/GitHub/Dissertation/img/DEME-TFSIStabilityWindow-Original.jpg}
  \captionsetup{width=0.75\textwidth}
  \captionof{figure}[Electrochemical stability window for DEME-TFSI]{Cyclic voltammetry measurement of DEME-TFSI (293 K, 1 x $10^{-6}$ Torr). The sweep rate is 13 mV/second. Current density is defined at the working electrode.} 
  \label{fig:ElecGate-4}
\end{centering}

Another concern for electrolyte gating is contamination of the electrolyte. Electrolytes are polar and hygroscopic; therefore, they will collect moisture and other contaminates from the atmosphere. When solvated in the ionic, these contaminants can create undesirable electrochemical interactions. Fortunately, the high vapor pressure of electrolytes means they can withstand vacuum, so some water can be removed by baking at elevated temperature in high vacuum. However, water remains the biggest source of contamination and exposure to air will necessarily reduce the electrochemical stability window. When possible, in my measurements I have attempted to reduce the exposure of the electrolyte to atmosphere.

Finally, electrolytes freeze below room temperature, which offers some challenges and advantages. Ions in an electrolyte must be free to move for the electrolyte to be polarized, so the electric field must be applied above the freezing temperature. The lower the freezing temperature, the lower the temperature at which the field can be applied. Because electrochemical reactions are suppressed at low temperatures, applying the field just above the freezing temperature minimizes There are often changes in the electric double layer near the freezing temperature, so it is important to avoid electrolytes with transition temperatures in a range of interest for the gated material.

\subsection{Thoughtful device design}

In a typical electrolyte gating experiment, a droplet of the electrolyte simultaneously covers a large metal coplanar gate and the material of interest. In addition to the concerns discussed previously, this configuration presents additional problems that can be minimized through device design.

Decreases in electric potential (i.e., potential drops) can occur anywhere between the biased coplanar gate and ground. For example, the dominant potential drop could occur at the gate-electrolyte interface, at the electrolyte-channel interface, or even between the coplanar gate and a substrate. Further, the electrolyte may form a passivating layer that can alter these potential drops as a function of applied gate voltage.

An effective electrolyte gating measurement has the dominant potential drop at the electrolyte-channel interface. Using a larger coplanar gate increases the likelihood that the potential will fall predominantly at the electrolyte -channel interface. Because impedance through the electrolyte goes as the square of inverse area, having a relatively large coplanar gate reduces the 


\section{Notable Electrolyte Gating Measurements}

Ambipolar gating of WS2. MoS2 metal-insulator transition. Trevor Petach's measurements.

\chapter{Electronic Transport}
This chapter covers all the transport measurements

\chapter{Raman Spectroscopy}

Raman spectroscopy uses inelastic light scattering to measure certain vibrational modes in materials. For single molecules, these modes correspond to Raman-active vibrations that change the polarizability of the molecule. For crystals, these modes are the collective motion of the lattice - the phonon modes.

I have used Raman spectroscopy to investigate the phonon modes of electrolyte-gated \ruclnospace. Because phonon modes are sensitive to bond lengths and atomic interactions, defects or distortion in the lattice should be visible as shifting or broadening of peaks in the Raman spectrum. Therefore Raman spectroscopy can identify changes in the lattice that may explain the absence of an expected electronic phase transition in electrolyte-gated \ruclnospace.

\section{Fundamentals of Raman Spectroscopy}
When a material is illuminated by monochromatic light of frequency $\omega$, the incident light is either absorbed, scattered, reflected, or transmitted. While the spectrum of the transmitted and reflected light contains only light at frequency $\omega$, the spectrum of the scattered light includes both radiation at $\omega$ and additional pairs of frequencies $\omega \pm \omega_{i}$. The frequencies $\omega_{i}$ are typically associated with transitions between rotational, vibrational, and electronic states of the constituent molecules of the material \cite{Long2002}. The scattered light typically has a randomized phase and polarization relative to the incident light.

The highest intensity radiation in the spectrum of scattered light occurs at frequency $\omega$ and is called Rayleigh scattered\footnote{Rayleigh scattering is responsible for diffuse sky radiation, or in otherwords, why the sky is blue.}. Radiation at other frequencies is Raman scattered. Raman scattering with frequency $\omega_{i} < \omega$ is called Stokes Raman scattering; Raman scattering with frequency $\omega_{i} > \omega$ is called anti-Stokes Raman scattering. Necessarily, Rayleigh scattering is an elastic scattering process ($E = \hbar \omega = \hbar \omega_{i}$), while Raman scattering is inelastic ($E_{\text{initial}} = \hbar \omega \neq E_{\text{final}} = \hbar \omega_{i}$).

A single molecule Rayleigh scatters incident light by absorbing a photon, moving from the ground state to an unstable virtual state, then decaying back to the ground state by emitting a photon with energy equal to the incident photon. In contrast, a single molecule Raman scatters incident light by absorbing a photon and moving to unstable virtual state, then decaying to an intermediate state above the ground state by emitting a lower-energy photon \footnote{This is Stokes Raman scattering. If a photon is incident on a molecule in an already excited state, and then the molecule decays back to the ground state, the emitted photon has \textit{more} energy than the incident photon. This type of scattering is anti-Stokes scattering and its magnitude can be used to determine the temperature of a material.}. The intermediate state then decays to the ground state by non-photonic processes (i.e., cooling by collision with other molecules).

Scattering from single crystals follows a similar pattern: an incident photon excites a molecule into an unstable virutal state. The Rayleigh scattering case is identical to that for a single molecule. But for Raman scattering, the intermediate state decays to the ground state by emitting a phonon - a quantized lattice vibration that carries energy away from the excited molecule. By conservation of energy, the emitted photon necessarily has less energy than the incident photon, and that difference in energy\footnote{This difference in energy is referred to as the Raman shift, and is typically expressed in wavenumber $\nu$ having units of inverse length (typically cm$^{-1}$).}. must be equal to the energy of the phonon. Accordingly, the spectrum of Raman-scattered light is rich with information about lattice vibrations. Figure \ref{fig:FundRamanSpect1} provides a visual aid for understanding these scattering processes.

\begin{centering}
\includegraphics[width=0.9\textwidth]{C:/Users/dsbjr/Documents/GitHub/Dissertation/img/RamanRayleighScatter-nanophoton.png}
  \captionsetup{width=0.75\textwidth}
  \captionof{figure}[Elastic and inelastic light scattering processes]{An energy level diagram of Rayleigh, Stokes Raman, and anti-Stokes Raman scattering. Image by Nanophoton Corporation, retrieved from \url{https://www.nanophoton.net/raman/raman-spectroscopy.html}}
  \label{fig:FundRamanSpect1}
\end{centering}

While Raman spectroscopy is an invaluable tool, it does have limitations. Only certain vibrational modes can participate in Raman scattering. To understand which modes are amenable to investigation by Raman spectroscopy, and the underpinnings of Rayleigh and Raman scattering, we appeal to a classical analysis of the theory of light scattering from molecules. The following discussion is an abbreviated version done by Long \cite{Long2002}.

A molecule exposed to radiation having an incident frequency $\omega_{1}$ will radiate with intensity $I$, given by

\begin{equation}
I = k'_{\omega} \omega_{s}^{4} p_{0}^{2} \sin^{2} \theta
\end{equation}

with

\begin{equation}
k'_{\omega} = \frac{1}{32 \pi^{2} \epsilon_{0} c^{3}}
\end{equation}

where $p_{0}$ is the magnitude of the electric dipole induced at frequency $\omega_{s}$ which is generally but not necessarily different from $\omega_{1}$. The coresponding wavenumber equations are

\begin{equation}
I = k'_{\nu} \nu_{s}^{4} p_{0}^{2} \sin^{2} \theta
\end{equation}

\begin{equation}
k'_{\nu} = \frac{\pi^{2} c}{2 \epsilon_{0}}
\end{equation}

We can write the induced electric dipole as a multipole expansion:

\begin{equation}
\mathbf{p} = \mathbf{p}^{(1)} + \mathbf{p}^{(2)} + \mathbf{p}^{(3)} + ...
\end{equation}

where

\begin{equation}
\begin{aligned}
	\mathbf{p}^{(1)} &= \boldsymbol{\alpha} \cdot \mathbf{E} \\
	\mathbf{p}^{(2)} &= \frac{1}{2} \boldsymbol{\beta} \cdot \mathbf{E} \mathbf{E} \\
	\mathbf{p}^{(3)} &= \frac{1}{6} \boldsymbol{\gamma} \cdot \mathbf{E} \mathbf{E} \mathbf{E}
\end{aligned}
\end{equation}
	
where $\boldsymbol{\alpha}$, $\boldsymbol{\beta}$, and $\boldsymbol{\gamma}$ are the polarizability, hyperpolarizability, and second order hyperpolarizability tensors of second, third, and fourth rank, respectively. An analysis including $\boldsymbol{\beta}$ and $\boldsymbol{\gamma}$ leads to hyper Raman scattering related to higher harmonics of the incident radiation that are beyond the scope of this dissertation. Accordingly, we retain only the dipole term in the multipole expansion.

A priori, there is no reason to assume the polarizability of the molecule will be constant as the constituent atoms vibrate when the molecule is excited. Accordingly, we can expand the polarizability tensor $\boldsymbol{\alpha}$ in a Taylor series of displacements from equilibrium:

\begin{equation}
\alpha_{\rho \sigma} = (\alpha_{\rho \sigma})_{0} + \sum_{k} \left( \frac{\partial \alpha_{\rho \sigma}}{\partial Q_{k}} \right)_{0} Q_{k} + \frac{1}{2} \sum_{k,l} \left( \frac{\partial^{2} \alpha_{\rho \sigma}}{\partial Q_{k} \partial Q_{l}} \right)_{0} Q_{k} Q_{l} ...
\end{equation}

where $\alpha_{\rho \sigma}$ are the components of the polarizability tensor (with $\rho$ and $\sigma$ spanning values x, y, and z), $Q_{i}$ are the normal coordinates of vibrations associated with molecular vibrational frequencies $\omega{i}$, and a subscript 0 indicates a value at equilibrium. Taking the electrical harmonicity approximation\footnote{Electrical harmonicity means that the variation of the polarizability in a vibration is proportional to the first power of $Q$, in analogy to mechanical harmonicity where a restoring force is proportional to displacement from equilibrium.}, we retain only the first power of $Q$ and can write the previous equation in the following way:

\begin{equation}
\begin{aligned}
	(\alpha_{\rho \sigma})_{k} &= (\alpha_{\rho \sigma})_{0} + (\alpha'_{\rho \sigma})_{k} Q_{k} \\
	(\alpha'_{\rho \sigma})_{k} &= \left( \frac{\partial \alpha_{\rho \sigma}}{\partial Q_{k}} \right)_{0}
\end{aligned}
\end{equation}

The components $(\alpha'_{\rho \sigma})_{k}$ are a well-formed tensor $\boldsymbol{\alpha}'_{k}$, so we can write the above equation as:

\begin{equation}
\boldsymbol{\alpha}_{k} = \boldsymbol{\alpha}_{0} + \boldsymbol{\alpha}'_{k} Q_{k}
\end{equation}


\section{Raman Spectrum of RuCl$\textsubscript{3}$}
Raman modes in RuCl
Raman spectra of RuCl

\section{Raman Spectrum of Electrolytes}
Results of tests to select DEME-BF4

\section{Raman Spectroscopy of RuCl$\textsubscript{3}$ with in-situ Electrolyte Gating}
Intercalated and unintercalated RuCl
Conclusion that in plane lattice is unperturbed

\chapter{X-Ray Diffraction}

Possible references:
%Kim, H.-S., & Kee, H.-Y. (2015). Crystal structure and magnetism in alpha-RuCl3: an ab-initio study, 155143, 1–10. https://doi.org/10.1103/PhysRevB.93.155143
Discusses different crystal structures

%Johnson, R. D., Williams, S. C., Haghighirad, A. A., Singleton, J., Zapf, V., Manuel, P., … Coldea, R. (2015). Monoclinic crystal structure of α-RuCl3 and the zigzag antiferromagnetic ground state. Physical Review B - Condensed Matter and Materials Physics, 92(23). https://doi.org/10.1103/PhysRevB.92.235119
Talks about stacking faults

\chapter{Conclusion}

This dissertation investigated \rucl doped by electrolyte biasing and found that, contrary to predictions, \rucl remained highly insulating. Further, Raman spectra for electrolyte-biased \rucl identified a hysteretic transition between two visually distinct states driven by electrolyte bias voltage. Finally, x-ray diffraction measurements showed these distinct states do not result from the cations in the electrolyte physically interacting with the \ruclnospace . In this concluding section, I review the results of these measurements and identify possible frameworks that could explain the data. I then suggest measurements that could further clarify the nature of electrolyte-biased \ruclnospace .

\section{Hypotheses}

\subsection{Unlikely explanations}
\begin{itemize}
\item \textbf{Electrochemistry:} In an electrolyte gating experiment, the first confounding effect that must be addressed is electrochemistry. However, there are several observations that rule out electrochemistry as the primary explanation of the behavior of electrolyte biased \ruclnospace . Measurements are made only within the electrochemical stability window of the electrolyte. Gate current remains low during measurements and, after an increase in gate voltage, there is a transient gate current that decays to an equilibrium gate current. The transition between the two states with distinct Raman spectra is repeatable, ruling out an irreversible chemical interaction. Finally, x-ray diffraction measurements show that the electrolyte does not penetrate the \rucl lattice, indicating that only the top and sides of the \rucl are available for chemical interaction. However, the presence of substrate features in the Raman spectrum indicates that the excitation characterizes the entire flake, which is dominated by the chemically-isolated bulk.

\item \textbf{Intercalation:} Only cations intercalate in \ruclnospace , given the planes of chlorine atoms present at each interface. However, it seems unreasonable to expect a cation as large as DEME+ to fit between the layers. Additionally, x-ray diffraction measurements show the interlayer separation remains a constant 5.7 \AA{} and does not change as a function of gate voltage. The layers do not move and there is no room for DEME+ to fit. Intercalation and subsequent exfoliation of a single top layer is unlikely (there is no reason to suspect only the top layer intercalates when previous measurements show uniform intercalation) and cannot explain the behavior of the bulk, which dominates the Raman signal. However, intercalation of only protons, which may not be observable by x-ray diffraction, cannot be ruled out.

\item \textbf{Stress caused by electrostatics:} \rucl in these measurements is a thin, charged flake immersed in a biased electrolyte. Accordingly, the electric fields may subject the flake to stress that causes changes in its Raman spectrum. Further, this stress could be consistent with the mottled appearance of the flakes. However, if the stress is electrostatic, then it should appear at both positive and negative electrolyte bias voltage. Instead, the response of the material is asymmetric with respect to gate voltage.

\item \textbf{Charge localization due to lattice deformation:} The electrolyte bias adds charge to \ruclnospace , but the material could remain insulating because the charge is localized by lattice deformations and cannot participate in transport. However, charge localization by a lattice distortion should be continuously dependent on the amount of charge added, and there are only sharp changes in the Raman spectrum as a function of electrolyte bias. Additionally, lattice deformations associated with charge defects, like those in color centers in alkali halides, show the presence of new, broad peaks associated with the lattice deformations \cite{Porto2006,Benedek1967,Moller1972}. Such features are not present in either Raman spectrum for electrolyte-biased \ruclnospace .
\end{itemize}

\subsection{The case for a phase transition}

The above discussion shows that many of the experimental artifacts or less interesting explanations for the behavior of electrolyte-biased \rucl can be ruled out. What remains is the possibility of structural or electronic phase transition. The transition between states having different Raman spectra is hysteretic and has an interface, showing the transition is first order. Further, there are several different \rucl structures that are nearly degenerate in energy \cite{Kim2016} that gating could drive transitions between. Additionally, there is evidence that charge doping in \rucl creates charge order \cite{Koitzsch2017}. If the charge ordering couples to the lattice, then an electronic transition could explain the observed behavior. Future work should explore these possibilities.

\section{Future work}

Future investigation of electrolyte-biased \rucl should focus on x-ray diffraction measurements. Brighter x-rays from a synchrotron light source would be able to resolve scattering planes with a component parallel to the c axis, and therefore better characterize the structure of \rucl in either state. Further, diffuse x-ray scattering, which measures correlations in electron density across unit cells, could characterize the role of stress in the transition.

If additional x-ray measurements do not show a change in the structure of \ruclnospace , then the possibility of an electronic transition should be considered. Charge-ordered materials have different transport properties depending on the orientation of the stripes, so if the transport properties of electrolyte-biased \rucl are anisotropic, then the transition may be structural instead of electronic.

Another possibility for investigating doped \rucl comes from creating van der Waals heterostructures with other 2D materials. Recent work shows that when \rucl is put in contact with graphene, the graphene acquires excess conductivity that can only be explained by the \rucl layers closest to the graphene becoming conducting \cite{Zhou2018b}. DFT calculations also predict superconductivity at the interface between \rucl and graphene under certain conditions \cite{Biswas2019}. While this system is new and more complex, it could offer another route to studying doping in \rucl.

\bibliography{C:/Users/dsbjr/Documents/GitHub/Dissertation/tex/dissertation.firstdraft}

\end{document}
