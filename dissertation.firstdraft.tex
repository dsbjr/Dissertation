%\documentclass[12pt,twoside]{report}
\documentclass[12pt]{report}

%% Last Modification by Derrick Boone 5 February 2019. Template from Emma Pease

% note that the document can be single or double sided.  
% note that the Registrar's office now allows 10pt, 11pt, or 12pt

\usepackage{suthesis-2e}
\usepackage{caption}
\usepackage{graphicx}
\usepackage{amsmath}
\usepackage{amsthm}
\usepackage{braket}
\usepackage{url}
\usepackage{gensymb}
\usepackage{textcomp}
\usepackage{braket}
\usepackage{amsfonts}
\usepackage{geometry}
\usepackage[pdfencoding=auto]{hyperref}
% default is now online June/2016 version
%\usepackage[online]{suthesis-2e}
%\usepackage[hardcopy]{suthesis-2e}
% the following is for doing engineering theses. 
% I am definitely not sure of the wording on the signature page so check
%\usepackage[engineer]{suthesis-2e}

% one can change the default font to Times Roman but note that most
% ways of creating pdf files from latex automatically embed (which btw
% is a good idea even with the standard fonts)

    \title{Electrolyte Biasing of the Proximate Kitaev Spin Liquid $\alpha$-Ruthenium (III) Chloride}
    \author{Derrick Sherrod Boone, Jr.}
    \dept{Applied Physics}
    \principaladviser{David Goldhaber-Gordon}
    \firstreader{Ian Fisher}
    \secondreader{Marc Kastner}
%% one can also have a \thirdreader and \fourthreader

%% note that certain departments and types of theses have other requirements
%% For instance theses in the departments of 
%% Asian Languages
%% French and Italian
%% Spanish and Portuguese
%% need to define the \dept, \dualthesis, and the actual language
%\dualthesis
%\languagemajor{Chinese}
%% 
%% Those for Graduate Program in Humanities need to define 
%\humanitiesthesis
%\jointprogram{Arts and Crafts}
%% 
%% For submission to a committee or program (no department)
% \committeethesis
% \programthesis
%%
%% For School of Education or Business or Law
% \educationthesis
% \businessthesis
% \lawthesis  (law actually isn't listed in the official documents, 2013/1014)

%DEFINE NEW COMMANDS
\newcommand{\rucl}{RuCl\textsubscript{3} }
\newcommand{\ruclnospace}{RuCl\textsubscript{3}}
\newtheorem{assumption}{Assumption}
\newcommand{\percmsq}{e\textsuperscript{-}/cm\textsuperscript{2}}
\newcommand{\pone}{P\textsubscript{1}}
\newcommand{\pzero}{P\textsubscript{0}}

\bibliographystyle{unsrt}

\begin{document}

% for a variety of reasons this is an all in one document; however,
% when actually doing the thesis it is strongly recommended that each
% chapter be in a separate file and use \include to include in the
% main file.

%% the \beforepreface command produces the title page
%% in the online version it skips the copyright (page 2) and signature (page 3) pages 
%% in the non-online version these would be included
    \beforepreface


%% Abstract can be any number of pages
    \prefacesection{Abstract}
A quantum spin liquid (QSL) is an exotic phase of matter identified by interacting spins that do not develop long-range magnetic order down to T = 0 K. QSLs have topological order and are therefore relevant to storing and processing quantum information. Additionally, hole-doped QSLs may be related to high temperature superconductivity, making the study of this phase relevant to a broad section of condensed matter physics.

Recently, the two-dimensional material and spin-assisted Mott insulator $\alpha$-ruthenium (III) chloride (\ruclnospace) has attracted research interest because of its relationship to spin liquids. Localized electrons in \rucl have anisotropic Ising interactions that are described by the Kitaev model – a theoretical type of spin liquid that is exactly solvable. The low-temperature antiferromagnetic order in \rucl can be suppressed by an in-plane magnetic field to create a field-induced QSL, identifying \rucl as a proximate Kitaev spin liquid.

Because \rucl is both a proximate Kitaev spin liquid and a Mott insulator, charge doping may create interesting electronic phases. Previous attempts to chemically dope \rucl have been unsuccessful. However, previously used doping techniques either disordered the lattice or added an unmeasured amount of charge, making this negative result hard to interpret. Charge doping by electrolyte biasing would eliminate these problems, and improve on previous measurements.

I use electronic transport measurements and Raman spectroscopy under electrolyte bias to show that, upon doping, \rucl undergoes a first order phase transition with only small changes in conductivity. X-ray diffraction of electrolyte-biased \rucl rules out intercalation being the cause of this phase transition, and confirms the assumption that electrochemical interactions are negligible. I conclude that this phase transition prevents \rucl from being used to study doped spin liquid physics and suggest further measurements to characterize the newly identified phase.

%% one can also have a prefacesection that is a Preface instead of
%% Acknowledgements.   The thematic purpose is the same (thanks).
    \prefacesection{Acknowledgements}
        Here is where I thank people and include some quotes.

%% afterpreface produces a table of contents and any other tables
%% wanted. At the end pagenumbering changes from roman to arabic and
%% is restarted
    \afterpreface
 
 
%\section*{How to read this dissertation}

This dissertation is about my attempts to dope $\alpha$-ruthenium (III) chloride, a spin-assisted Mott insulator and proximate Kitaev spin liquid which hosts a field-induced Kitaev quantum spin liquid. Doped spin liquids are a potential explanation for high $T_{c}$ superconductivity. Moreover, the original high temperature superconductors, the cuprates, can be considered as doped Mott insulators. Therefore investigation of doped $\alpha$-ruthenium (III) chloride is relevant to a broad section of condensed matter physics.

To avoid perturbing its electrical or magnetic properties, I have doped this material using electrolyte gating. I have investigated the doped system using electronic transport, Raman spectroscopy, and x-ray diffraction. I conclude that doping causes a phase transition in $\alpha$-ruthenium (III) chloride without any physical changes to the lattice, and suggest that this phase transition will prevent this material from being used to study doped spin liquid physics.

Chapters 1 through 3 provide a background on spin liquids, $\alpha$-ruthenium (III) chloride, and electrolyte gating, and could be skipped without if the reader is familiar with the field. Chapters 4 through 6 discuss transport, Raman, and x-ray measurements and report their results. The most novel data and conclusions are from Raman and discussed in Chapter 5, while the transport and x-ray data in Chapters 4 and 6 provide supporting information. Chapter 7 synthesizes the ideas in this dissertation and suggests future avenues for exploration. The appendices contain detailed experimental procedures, other (less successful) projects I've worked on, and some things I find funny.

If you're short on time, read Chapters 5 and 7 and you'll understand most of what I have to say.

\chapter{Introduction}
Condensed matter physics is the rigorous study of what happens when a large number of cold atoms at high density are allowed to interact. It tells us why and at what temperature water freezes, why magnets attract some materials and not others, why glass is clear, why metal is shiny, and many other things. It is the branch of physics that reveals the richness of our physical world.

In this introduction, we will discuss the theory of classical phases of matter and their transitions, introduce topological order and topological phases of matter, and conclude by examining a topological phase of particular interest, the quantum spin liquid.

\section{Classical phases of matter and their transitions}
Many materials that differ in their constituents and microscopic structure have similar bulk properties. For example, although water and mercury at ambient conditions have dramatically different densities and electrical conductivities, they are both nearly incompressible and deform continuously when a shear stress is applied. We capture these similarities by saying water and mercury are both in the liquid phase\footnote{Depending on the context, there may be a difference between a \textit{phase} of matter and \textit{state} of matter. I will use phase in this dissertation as it seems to apply more generally.}. Phases of matter arise not just because of properties of the matter, but also how the matter is arranged.

\subsection{What is a classical phase of matter?}

Put simply, a classical phase of matter is a thermodynamic system\footnote{A system with defined thermodynamic state variables e.g., pressure, temperature, entropy.} at equilibrium with uniform physical properties (e.g., density, magnetization). Solids, liquids, gases, and plasmas are examples of phases of matter from our everyday lives, though many more exist. While the ``uniform physical property'' definition is an intuitive way to describe a phase, a rigorous definition also includes analyticity. For a thermodynamic system to constitute a phase of matter, the physical properties of the system must also be analytic functions\footnote{An analytic function is one that is given by a locally convergent power series. This definition implies that all derivatives of the function exist - hence the description of an analytic function as smooth.} of thermodynamic variables \cite{Pathria2011}. Therefore, physical properties of matter in a given phase must be the same for all space, and these properties must be smooth functions of the parameters. For example, liquid water at a uniform temperature (at equilibrium) has the same density everywhere (uniform physical property), and when the temperature increases by a small amount, its density decreases by a correspondingly small amount (analyticity).

Now that we can define a phase of matter, we need a way to distinguish between different phases for a given thermodynamic system. A powerful way to distinguish phases is by looking at the symmetries of the microscopic constituents for a particular phase. Each separate phase has a set of symmetry operations that leave the system invariant. For example, if we imagine shrinking ourselves down to the size of a water molecule in a glass of water, then the scene around us doesn't change no matter where we look or how far we move through space - water molecules are randomly distributed around us. So a liquid has \textit{continuous rotational symmetry} and \textit{continuous translational symmetry}. These symmetries change if we freeze the water into ice. Now as we move through space the distribution of water molecules is equivalent only if we move in units of the lattice constant. Unlike fluids, solids have \textit{discrete translational symmetry}. Upon the freezing of a fluid into a solid, we say that the continuous translation symmetry of the fluid is broken. 

We can apply a similar analysis to magnetic systems. In a magnetically disordered phase, spins point in all directions and the system has both time-reversal and continuous rotational symmetries. However, if the spins align and the system becomes ferromagnetic, both of these symmetries are broken. Table \ref{tbl:PhasesSymmetries} lists some common phases and their associated broken symmetries \cite{Chaikin1995}.

\begin{center}
\resizebox{\textwidth}{!}{
	\begin{tabular}{l | l | l | l | l | l | l}
		\hline
		\hline
		\textbf{Phase} & Fluid & Nematic & Smectic-A & Crystal & Heisenberg Magnet & Superfluid \\ \hline
		\textbf{Broken Symmetry} & None & Rotational & 1D Translation & 3D Translation & Time-reversal; Rotational & Phase\\ \hline \hline
	\end{tabular}
	}
	\captionof{table}{Selected phases and their associated broken symmetries}\label{tbl:PhasesSymmetries}
\end{center}

We have discussed what a phase of matter is and how phases are distinguished: uniform and analytic equilibrium properties, and the symmetries of microscopic constituents. However, these two things alone are an incomplete picture, as they do not describe transitions between phases. To understand phase transitions, we will appeal to Landau theory.

\subsection{Landau theory and phase transitions}

As discussed previously, the physical properties of a phase of matter bust be analytic functions of thermodynamic state variables (for example, the pressure of a gas and the derivatives of the pressure as a function of temperature must all exist and be continuous). Therefore distinct phases must be separated by values of the thermodynamic state variables at which the physical properties are non-analytic. Given that the physical properties of a system are determined by its free energy, we would expect the non-analyticity to appear in the free energy. From this expectation, Landau theory makes several assumptions to formulate a theory of phas transitions \cite{Landau1969}.

\begin{assumption}
There exists a local, phenomenological order parameter $\eta$ which characterizes the non-analyticity at the phase transition. $\eta$ is zero in one phase and non-zero in the other.
\end{assumption}

This assumption is easy to understand empirically. For the example of boiling water, we might guess that the order parameter is related density, because the density can be defined at any single point and changes discontinuously by a factor of $10^6$ when moving from the liquid to the gas phase. In this example the order parameter is the deviation from the liquid density: $\eta = \rho - \rho_{\text{liquid}}$. For a magnetic transition, we might guess that the order parameter is the net magnetization $\mathbf{M}$. For the magnetically disordered phase $\mathbf{M} = 0$, while for the ferromagnetic phase $\mathbf{M}$ has some nonzero value.

For many systems, the order parameter is related to derivatives of the free energy with respect to the tuned variable \cite{Binder1987}. For boiling water,

\begin{equation}
\left. \frac{df}{dP} \right\vert_{T_{0}} = \frac{1}{\rho} \simeq \frac{1}{\eta}
\end{equation}

For magnetization,

\begin{equation}
\left. \frac{df}{d\mathbf{H}}\right\vert_{T_{0}} = \mathbf{M}
\end{equation}

These relationships are not surprising: if we write the free energy as an expansion in the tuned variable, the only place for the non-analyticity to hide is in the derivatives.

Having defined an order parameter, Landau theory makes a second assumption.

\begin{assumption}
The order parameter $\eta$ is small in the neighborhood of the phase transition and the free energy can be expanded in powers of $\eta$.
\end{assumption}

Written explicitly,

\begin{equation}
f(\phi_{i},T,\eta) = f_{0} + \alpha \eta + A \eta^{2} + B \eta^{3} + C \eta^{4} + ...
\end{equation}

Because $\eta$ is zero in one phase, it seems reasonable to assume a small $\eta$ near the phase transition. However, though it turns out to be okay, the assumed expansion is fraught with peril. Landau himself says \cite{Landau1969}

\begin{quote}
... the possibility of such an expansion is by no means obvious a priori. Moreover, since, as already mentioned, a transition point ... must be a singularity of the thermodynamic potential, there is every reason to suppose that such an expansion cannot be continued to terms of arbitrarily high order, and that the expansion coefficients can have singularities as functions of [$\phi_{i}$]... A complete elucidation of the nature of the singularity of the thermodynamic potential at the transition point offers great difficulties and has not yet been achieved. ... we shall give a theory based on the assumption that the presence of a singularity does not affect the terms of the expansion that are used.
\end{quote}

The final assumption of Landau theory relates to the symmetry of the system.

\begin{assumption}
The symmetries of the free energy expansion must match those of the Hamiltonian describing the low symmetry phase.
\end{assumption}

This assumption ensures that the free energy expansion remains physical. For example, the physical properties of a system with global spin-flip degeneracy are agnostic to which direction is defined as up or down (i.e., the definitions corresponding to positive and negative spin can be swapped without changing the system). Accordingly, only even powers of $\mathbf{M}$ can remain in the free energy expansion.

Using the three assumptions of Landau theory, we can postulate a functional form of the free energy for a system with reflection symmetry \cite{Binder1987}:

\begin{equation}
f = f_{0} + r(T-T_{c}) \eta^{2} + u \eta^{4} + \nu \eta^{6}
\end{equation}

where $T_{c}$ is the critical temperature below which a phase transition can occur. In the subsequent section we will explore the phase transitions described by this functional form.

\subsection{Classifications of phase transitions}

Our functional form of the free energy relates the order parameter and the critical temperature to the free energy of the system. Because thermodynamic systems minimize their free energy, we know that the observed order parameter is one for which the free energy is a local or global minimum.

We can consider two broad classes of phase transitions - those in which the order parameter varies discontinuously between two values, and those in which it varies smoothly. The former are called first order phase transitions, while the latter are called continuous phase transitions.

\subsubsection{First order phase transitions}

Consider our postulated free energy functional form

\begin{equation}
f = f_{0} + r(T-T_{c}) \eta^{2} + u \eta^{4} + \nu \eta^{6}
\end{equation}

If $r,\nu > 0$ while $u < 0$, the minima for the order parameter will emerge discontinuously as the temperature is lowered below $T_{c}$, as shown in Figure \ref{fig:ClassPhaseTrans1}.

\begin{centering}
\includegraphics[width=0.5\textwidth]{C:/Users/dsbjr/Documents/GitHub/Dissertation/img/FirstOrderPhaseTransition-Binder.png}
  \captionsetup{width=0.9\textwidth}
  \captionof{figure}[Free energy functional form for a first order phase transition]{Free energy functional form for a first order phase transition. At temperature  $T = T_{c}$, two free energy minima appear at $\pm \eta_{0}$ which are disconnected from the high temperature minima at $\eta = 0$.  Adapted from \cite{Binder1987}.}
  \label{fig:ClassPhaseTrans1}
\end{centering}

If we identify the order parameter $\eta$ with the first derivative of free energy with respect to a tunable parameter, then we see that, at a temperature below $T_{c}$, the first derivative of the free energy must change discontinuously. Accordingly, these transitions are called ``first order" transitions. The familiar phase transitions between solids, liquids, and gases are all first order. 

The discontinuous change in the order parameter gives rise to clear experimental signatures that identify a transition as first order. A first order transition necessarily involves an interface between two phases, hysteresis, and latent heat. Any of these phenomena is sufficient to show a transition is first order as they are all concurrent \cite{Mnyukh2011}.

\subsubsection{Continuous phase transitions}

If $r,\nu,u > 0$, the minima for the order parameter will emerge smoothly as the temperature lowered below $T_{c}$, as shown in Figure \ref{fig:ClassPhaseTrans2}.

\begin{centering}
\includegraphics[width=0.5\textwidth]{C:/Users/dsbjr/Documents/GitHub/Dissertation/img/SecondOrderPhaseTransition-Binder.png}
  \captionsetup{width=0.9\textwidth}
  \captionof{figure}[Free energy functional form for a second order phase transition]{Free energy functional form for a second order phase transition. At temperature  $T = T_{c}$, two free energy minima at $\pm \eta_{0}$ smoothly emerge from the high temperature minima at $\eta = 0$.  Adapted from \cite{Binder1987}.}
  \label{fig:ClassPhaseTrans2}
\end{centering}

For this type of phase transition, the continuous change in order parameter implies that the non-analyticity must exist in the second or higher order derivatives of the free energy with respect to the tunable parameter. Continuous phase transitions are marked by the absence of latent heat, a divergent susceptibility, an infinite correlation length, and a power law decay of correlations near the critical point \cite{Cardy1996}.

A common example of a continuous phase transition is the ferromagnetic transition of iron\footnote{This transition is not exactly continuous; Yang and colleagues measured a small latent heat and hysteresis \cite{Yang2008}. However, the first order nature of this particular transition may be due to the coupling of the order parameter to other parameters like strain.} near 800\degree C. While we typically associated hysteresis with ferromagnetism, note that the hysteresis that comes to mind is related to changing the direction of magnetization, not the transition between a phase that can be magnetized and a phase that cannot. 

\subsection{Summary}
In this section, we have defined phases of matter as thermodynamic systems with uniform and analytic physical properties and certain symmetries associated with their microscopic constituents. We also used Landau theory to understand transitions between phases based on the non-analyticity of the free energy and discontinuous (i.e., first order) or continuous changes in the order parameter. But while Landau theory is powerful, it is necessarily incomplete, as it fails to consider long-range interactions in certain materials.

Quantum materials may have phase transitions at $T = 0$K that are driven by quantum fluctuations instead of thermal fluctuations. Some quantum phase transitions occur in systems that maintain a single set of symmetries but nonetheless have phases separate by phase transitions. Rather than the local order parameter of Landau theory, these systems possess topological order - a novel type of order than also can be used to characterize a phase \cite{Wen1990}.

\section{Topological Phases}

Topological order is a property of quantum systems that have both large ground state degeneracy and long-range entanglement. In these systems, there is no local order parameter like density or magnetization. Instead, there is a global topological invariant that changes discontinuously between phases \cite{Wen2017}. First, we set out to understand topology by considering the properties of a simple quantum topological system - Kitaev's toric code. Using the topological concepts we learn from the toric code, we will be able to understand the topological nature of some physical systems, including quantum hall states and spin liquids.

\subsection{The toric code}

The following explanation draws heavily from \cite{Kitaev2003} and \cite{topOrderEdX}.

Consider a system of spin-$\frac{1}{2}$ particles living on the edges of a square lattice with periodic boundary conditions. We first define two operators:

\begin{align*}
A_s&=\prod_{j \in star(s)} \sigma^{x}_{j} 		& B_{p}&=\prod_{j \in plaquette(s)} \sigma^{z}_{j}
\end{align*}

\begin{centering}
\includegraphics[width=0.5\textwidth]{C:/Users/dsbjr/Documents/GitHub/Dissertation/img/ToricCode-TopOrderEdX.jpg}
  \captionsetup{width=0.9\textwidth}
  \captionof{figure}[Toric code operators on the lattice]{A square lattice with a spin-$\frac{1}{2}$ particles on each edge. The toric code operators $A_{s}$ and $B_{p}$ act on the spins highlighted in blue and red, respectively. Diagram from \cite{topOrderEdX}.}
  \label{fig:toricCode1}
\end{centering}

The operator $A_{s}$ multiplies the spin in the x direction for all spins connected at vertex $s$ (for star). The operator $B_{p}$ multiplies the spin in the z direction for all the spins around a square $p$ (for plaquette). Operators $A_{s}$ and $B_{s}$ are both Hermitian with eigenvalues $\pm 1$. Perhaps surprisingly, they also commute. While it is clear the operators commute for distant vertices and plaquettes, we can also see that they commute for a vertex and plaquette with overlapping bonds because a vertex and plaquette will always share exactly two edges \cite{Kitaev2003}.

With an established understanding of $A_{s}$ and $B_{p}$, let us consider the ground state of the following Hamiltonian:

\begin{equation}
H_{\text{tor}} = -A \prod_{s} A_{s} - B \prod_{p} B_{p}
\end{equation}

Because $A_{s}$ and $B_{s}$ commute, the ground state of the Hamiltonian is simply the ground state that simultaneously minimizes the energy for both operators. Consider first the minimal energy state associated with $A_{s}$ in the $\sigma^{z}$ basis. Provided that exactly two bonds in a each plaquette have the same spin, the $A_{s}$ term in the Hamiltonian will be minimized. Having only two bonds in each plaquette with the same spin defines a ``loop gas" - any state that consists of closed loops of the same spin will be a satisfactory ground state. This large number of states that minimize the energy of the Hamiltonian satisfies the first requirement for topological order: massive ground state degeneracy.

\begin{centering}
\includegraphics[width=0.8\textwidth]{C:/Users/dsbjr/Documents/GitHub/Dissertation/img/ToricCode-Fractional-TopOrderEdX.jpg}
  \captionsetup{width=0.75\textwidth}
  \captionof{figure}[Fractional excitations in the toric code]{Loops and fractional excitations in the toric code. Each end is a topological defect that carries a fractional spin. Red and blue highlights relate to the $\sigma^{z}$ and $\sigma^{x}$ operators, respectively. Diagram from \cite{topOrderEdX}.}
  \label{fig:toricCode2}
\end{centering}

Now consider the minimal energy state associated with $B_{p}$. In the $\sigma^{x}$ basis, using the above argument we find that the minimal energy state consists of loops of the same spin drawn on the dual lattice.\footnote{The dual of a lattice $R$ is the set $\hat{R}$ of all vectors $\mathbf{x} \in$ span($\Lambda$) such that $\mathbf{x} \cdot \mathbf{y}$ is an integer for all $\mathbf{y} \in \Lambda$. In this case, the dual lattice is a square lattice with the same lattice constant having a vertex at the center of each plaquette in the original lattice.} But in the $\sigma^{z}$ basis, $\sigma^{x}$ is off-diagonal. Therefore the states that minimize $B_{p}$ in the $\sigma^{z}$ basis are necessarily superposition states - in this case an equal weight superposition of all the possible states that minimize $A_{s}$ \cite{Savary2017}.\footnote{This illustrates a beautiful symmetry of the toric code. In either (or any) basis, the ground state is highly entangled, as it must be.} This kind of superposition cannot be factored into a product of states, and therefore satisfies the second requirement for topological order: long-range entanglement.

We have seen that the ground state of the toric code Hamiltonian satisfies the requirements for having topological order. But what can we observe about this system that shows it to have topological order? Further investigation of this system shows that it allows quasiparticle excitations with fractional quantum numbers.

Consider the ground state of the toric code Hamiltonian to be a vacuum state of closed spin loops. What if we were to flip a spin in the lattice? To do so, we would add some amount of energy (either $A$ or $B$ from the Hamiltonian) and break the loop into a string (see Figure~\ref{fig:toricCode2}). Any spin flip excitation must carry integer spin ($-\frac{1}{2}$ to $\frac{1}{2}$ or vice-versa is a change by an integer amount). However, when we break a loop into a string, we get \textit{two} quasiparticle excitation ``ends", each of which may diffuse around the lattice by smooth deformations\footnote{If we flip two and only two spins that share a vertex or a plaquette, we have not changed the energy of the system. Accordingly, such changes are allowed and can smoothly deform loops without breaking them} of the loops in the ground state. While flipping a spin changes the spin of the system by an integer, there are two physically separate quasiparticle excitations over which the integer spin is distributed. Each end of the string is therefore a quasiparticle excitation carrying fractional spin. Fractional excitations are the hallmark of topological order, and their presence in a system is proof that it is topologically nontrivial.

\begin{centering}
\includegraphics[width=0.8\textwidth]{C:/Users/dsbjr/Documents/GitHub/Dissertation/img/ToricCode-Torus-CC.jpg}
  \captionsetup{width=0.75\textwidth}
  \captionof{figure}[Topologically distinct loops in the toric code]{Red and blue lines show topologically distinct loops on the toric code lattice.}
  \label{fig:toricCode3}
\end{centering}

Another physical realization of the topology of the system is the four energetically degenerate but topologically distinct ground state configurations which are robust against local perturbations. Consider a loop that spans the boundaries of the lattice. This loop cannot be contracted to a point by any unitary locality-preserving operator, and is therefore topologically nontrivial. Further, a state with this loop is a ground state of the system, and therefore protected against local perturbations by the gap in the spectrum of $H_{\text{tor}}$ \cite{Bravyi2010}. There are four of these states, one for each way one can cross a periodic boundary (in the x and y directions, visualized in figure \ref{fig:toricCode3}), and one for each of the types of loops ($\sigma^{z}$ and $\sigma^{x}$). The number of boundary-spanning loops in a particular state serves at the topological invariant in this system and distinguishes topological states from one another. The topological invariant changes discontinuously as we move between states with different numbers of boundary-spanning loops. Rather than phases having different symmetries, we have phases with different topological invariants.

With the toric code, we have seen how massive ground state degeneracy and long-range entanglement give rise to fractional excitations and topological invariants that distinguish separate phases of matter. However, the toric code, while useful pedagogically, does not currently have a physical realization. To see physical evidence of topological order, we next turn to quantum Hall systems.

\subsection{Example: Quantum Hall Effects}

Here we briefly review the classical Hall effect, before describing the integer and fractional quantum Hall effects, which have topological order.

\subsubsection{The Classical Hall Effect}
A charged particle moving in a magnetic field experiences the Lorentz force, which is perpendicular to both the velocity of the particle and the magnetic field and proportional to the product of the charge, speed, and magnetic field strength \cite{Griffiths1999}:

\begin{equation}
\vec{\mathbf{F}} = q(\vec{\mathbf{v}} \times \vec{\mathbf{B}})
\end{equation}

During the diffusive transport of charge carriers (electrons or holes) in a conducting material exposed to a magnetic field, the Lorentz force gives rise to the classical Hall effect. Charge carriers accumulate along the edges of a current-carrying strip of material, creating an electric field transverse to the direction of the current and a corresponding Hall Voltage, given by:

\begin{equation}
V_{h} = \frac{I B R_{H}}{t}
\end{equation}

Where $I$,  $B$, $R_{H}$, and $t$ are the current, magnetic field strength, Hall coefficient (an intrinsic property of the material), and the thickness of the current-carrying sample, respectively. The Hall effect is often used to determine the charge and density of the charge carriers in a material \cite{Pierret2002}.

At low temperatures, low densities, and small length scales, the transport of charge carriers (assumed to be electrons for the remainder of this discussion) subject to a magnetic field is no longer dominated by scattering between electrons and impurities, nuclei, or other electrons. When electrons scatter so infrequently that they begin to display trajectories that no longer have the characteristics of a random walk (i.e., electrons move without collision long enough such that their trajectories bend due to the the Lorentz force), we say that the transport has transitioned from diffusive to ballistic. For our systems, this occurs when the average time between scattering events exceeds the cyclotron period. The delicate interaction between the kinetic and potential energy of ballistic electrons in a magnetic field gives rise to the rich physics of the quantum Hall effect \cite{Beenakker1991}.

\subsubsection{The Integer Quantum Hall Effect}

In 1980, von Klitzing, Dorda, and Pepper performed Hall measurements at high magnetic fields and low temperatures (up to 18 Tesla and as low as 1.5 Kelvin) on a two-dimensional electron gas (2DEG) at the inversion layer of a silicon-based metal oxide semi-conductor field effect transistor \cite{VonKlitzing1980}. The result of these measurements showed that the Hall voltage plateaued and the longitudinal voltage approached zero at certain discrete values of electron density (modulated by applying a gate voltage).

\begin{centering}
\includegraphics[width=0.5\textwidth]{C:/Users/dsbjr/Documents/GitHub/Dissertation/img/QHEMeasurement-vonK.png}
  \captionsetup{width=0.75\textwidth}
  \captionof{figure}[von Klitzing's measurement of the quantum Hall effect]{von Klitzing's measurement of the Hall voltage $U_{h}$ and longitudinal voltage $U_{pp}$ \cite{VonKlitzing1980}. Note that the minima in this figure at non-integer values of n result from spin and valley degeneracy.}
  \label{fig:IQHE1}
\end{centering}

We can understand the behavior of $U_{h}$ (the Hall voltage defined above as $V_{H}$) and $U_{pp}$ (the longitudinal voltage measured parallel to the direction of the current) by examining the wavefunction for electrons in a 2DEG subject to a magnetic field. Consider a system in which the magnetic field is oriented in the z-direction using the Landau gauge $\vec{\mathbf{A}} = (0,Bx,0)$. The Hamiltonian for this system\footnote{neglecting electron-electron interactions} is given by

\begin{equation}
\hat{H} = \sum_{i} \hat{H_{i}} = \frac{1}{2m_{e}} \left[ \hat{p}_{x,i}^{2} + \left(\hat{p}_{y,i} + \frac{eB}{c} \hat{x}_{i} \right)^{2} \right]
\end{equation}

Because the Hamiltonian does not explicit depend on the coordinates $y,i$, we can guess an ansatz solution of the form $\Psi(x,y) = \psi_{x}(x) e^{ik_{y}y}$ for a single electron. Solving the Schrodinger equation using this ansatz gives $\psi_{x}(x)$ as the solution to the quantum harmonic oscillator, and therefore the overall solution for a single electron becomes:

\begin{equation}
\Psi(x,y) = N e^{ik_{y}y}e^{\frac{(x-X)^{2}}{2(\hbar c/eB)}} H_{n} \left(\frac{x-X}{\sqrt{\hbar c/ eB}} \right)
\end{equation}

Where $X$ is a parameter related to the size of the cyclotron orbit, $N$ is a normalization constant, and $H_{n}$ is the nth Hermite polynomial. The ground state of the many electron wavefunction, the lowest Landau level can be written as \cite{Yoshioka2002}

\begin{equation} \label{psi-el}
\psi(z_{i},\overline{z}_{i}) = N \prod_{i > j} (z_{i} - z_{j}) e^{-\frac{eB}{4 \hbar c} \sum_{i} z_{i} \overline{z}_{i}}
\end{equation}

where $z_{i} = x_{i} + iy_{i}$ is the complex position of the \textit{i}th electron.

This wavefunction must be degenerate, as it is composed of single electron harmonic oscillator wavefunctions. In fact, the degeneracy is given by \cite{Chakraborty1995}:

\begin{equation}
\nu = \frac{n_{e}\Phi_{0}}{\Phi}
\end{equation}

where $n_{e}$ is the number of electrons and $\Phi$ and $\Phi_{0}$ are the magnetic flux and magnetic flux quantum, respectively\footnote{Here we assume that we work in a material without valley degeneracy and that the magnetic field is large enough such that the 2DEG is spin polarized (equivalently, that the Zeeman splitting is greater than $\hbar \omega_{c}$).}. Therefore, the ground state of this system is degenerate - a requirement for topological order. A discussion of entanglement in this system is beyond the scope of this dissertation.

We can see evidence of this topological order by considering the quantization of the Hall voltage in this system. The density of states is a superposition of harmonic oscillator states, so we write it\footnote{Note that this expression is exact for ideal materials at zero temperature only. Impurities and non-zero temperature will broaden the delta function peaks.} as:

\begin{equation}
D(E) \propto \sum_{n} \delta \left(E - \left(n + \frac{1}{2} \hbar \omega_{c} \right) \right)
\end{equation}

where $n$ is the index of the Landau level the electrons occupy, and $\omega_{c}$ is the cyclotron frequency given by $\omega_{c} = \frac{eB}{m_{e} c}$ \cite{Chakraborty1995}.

Consider the system at zero temperature. As the electron density is tuned, the Fermi energy in the sample passes through the delta functions in the expression for the density of states. Between the peaks in the density of states, the electrons are locked in cyclotron orbits and therefore the bulk of the system in insulating. However, the electrons on the edges of the system cannot complete a full cyclotron orbit without encountering the edge of the system. We can visualize these electrons as executing "skipping" orbits along the edges. These edge states carry current without dissipation and therefore $U_{pp}$ approaches zero and $U_{h}$ is fixed at a constant voltage. As the Fermi energy passes through each harmonic oscillator energy level, $\frac{U_{h}}{I}$ changes between different integer multiples of $R_{k} \approx 25812$ $\Omega$, the resistance quantum. We can see the hallmarks of topological order in this system by interpreting $n_{e}$ as the tunable parameter and $R_{h} = \frac{U_{h}}{I} = \nu R_{k}$\footnote{$\nu$ is also called the filling factor, because it can be interpreted as the number of Landau levels that are filled for a given electron density and magnetic flux. For filling factors having $\nu$ less than one, there is at least one flux quantum for every electron, so all the electrons are in the lowest Landau level. For $\nu$ greater than one, each flux quanta already is already associated with at least one electron, so by the Pauli exclusion principle the remaining electrons are pushed into higher Landau levels.} as the topological order parameter. As we tune the density, the topological order parameter changes discontinously between distinct quantum Hall phases.

A complete discussion entanglement and the topological properties of the IQHE requires an appeal to gauge symmetries and group theory beyond the scope of this dissertation (see section III of \cite{Wen2017} for further information). However, the topological order inherent in the IQHE manifests more easily in the fractional quantum Hall effect, where excitations are fractionalized and $\nu$ discussed above takes on non-integer values.

\subsubsection{The Fractional Quantum Hall Effect}

In 1982, Tsui, Stormer, and Gossard measured a quantized Hall voltage at $\nu = \frac{1}{3}$ \cite{Tsui1982}, which they called "striking evidence for a new electronic state..." (their results also suggested quantized Hall voltages for $\nu = \frac{2}{3}$ and $\nu = \frac{3}{2}$, but the mobility of electrons in the sample and the measurement temperature prevented confirming these states). We would not expect a gap in the density of states at filling factors corresponding to fractions. Accordingly, the analysis for the integer quantum Hall effect breaks down in the presence of Tsui's measurement.

In 1983, Laughlin, by a variational method, proposed the form of a wavefunction that accurately describes the Fractional Quantum Hall Effect (FQHE) identified by Tsui \cite{Laughlin1983}. The Laughlin wavefunction is given by

\begin{equation}
\psi = \prod_{i < j} \left( \frac{z_{i} - z_{j}}{l_{B}} \right)^{3} e^{-\frac{1}{4} \sum_{i} \frac{|z_{i}|^{2}}{l_{B}^{2}}}
\end{equation}

where $z_{i}$ is the complex position of the \textit{i}th electron as before and $l_{B}$ is the magnetic length given by $l_{B} = \sqrt{\frac{\hbar}{eB}}$.

Further measurements confirmed the presence of a series of fractional quantum Hall states occurring at filling factors of the form $\nu = \frac{1}{p}$, where where $p$ is an odd integer. The Laughlin wavefunction, generalized to the form

\begin{equation}
\psi = \prod_{i < j} \left( \frac{z_{i} - z_{j}}{l_{B}} \right)^{\frac{1}{\nu}} e^{-\frac{1}{4} \sum_{i} \frac{|z_{i}|^{2}}{l_{B}^{2}}}
\end{equation}

also describes these states. Note that $p$ must be odd such that the wavefunction is antisymmetric and describes fermions.

\begin{centering}
\includegraphics[width=0.5\textwidth]{C:/Users/dsbjr/Documents/GitHub/Dissertation/img/FQHEMeasurement-Willet.png}
  \captionsetup{width=0.75\textwidth}
  \captionof{figure}[Fractional Quantum Hall Effect Measurement]{Willet's measurement of the Hall resistivity as a function of magnetic field \cite{Willet1987}. Note that the large number of fractions.}
  \label{fig:FQHE1}
\end{centering}

The Laughlin wavefunction describes a system of electrons that avoid each other in an optimal way. For a state with $\nu = 1/p$, the wavefunction is proportional to the p$^{\text{th}}$ power of the separation between the electrons. This factor in the wavefunction yields two results. First, if an electron approaches within a magnetic length of another electron, the magnitude of the wavefunction drops dramatically, meaning electrons must stay far apart. Second, the wavefunction cannot be factored into single-particle states and therefore is massively entangled.

A fractional quantum Hall state meets the requirements for topological order and shows fractional excitations like those in the toric code. A direct measurement of the charge of quasiparticle excitations in the $\nu=\frac{1}{3}$ state shows that electrons carry a charge of $\frac{e}{3}$ \cite{Goldman1995}.

\subsubsection{Conclusion}

Through our examination of the quantum Hall effect, we have seen a physical realization of a topologically-ordered phase of matter.  For these systems, the tunable parameter is the electron density, the topological invariant is related to the filling factor, and the phases are separated by discontinuous changes in the topological invariant that we observe as changes in the quantized conductance. These phases demonstrate ground state degeneracy and entanglement, and show fractionalized excitations that are the hallmark of topological order.

\section{Spin Liquids}

Now that we have an understanding of topological order and have examined a physical realization of a topologically-ordered system, we turn our attention to spin liquids.

\subsection{What is a Spin Liquid?}

Spin liquids are challenging to define. In fact, it is easier to say what they \textit{aren't} rather than what they are. As a working definition, we can define a quantum spin liquids (QSLs) as a system with a ground state composed of a quantum superposition of well-formed, correlated, local magnetic moments that do not develop long-range order, even at T = 0 K \cite{Balents2010}. A prerequisite for avoiding long range order is geometric frustration. Accordingly, we will first discuss geometric frustration before considering the other properties of spin liquids.

\subsubsection{Geometric Frustration and Spin Liquids}

Consider a system of classical Ising spins interacting on a square lattice and a triangular lattice according to the following Hamiltonian: 

\begin{equation}
H = \sum_{\text{N.N.}} -J s_{i} \cdot s_{j}
\end{equation}

On the square lattice, given either a ferromagnetic interaction ($J > 0$) or an antiferromagnetic interaction ($J < 0$), there is a single minimal energy configuration. The nearest neighbor spins are either parallel or anti-parallel, respectively. These are the ferromagnetic and antiferromagnetic (\textit{Ne{\'e}l}) states. There is also a unique minimal energy configuration for the ferromagnetic interaction on the triangular lattice. But what is the ground state for an antiferromagnetic interaction on the triangular lattice?

\begin{centering}
\includegraphics[width=0.5\textwidth]{C:/Users/dsbjr/Documents/GitHub/Dissertation/img/Frustration-Balents2010.png}
  \captionsetup{width=0.75\textwidth}
  \captionof{figure}[Geometric Frustration on the Triangular Lattice]{Degenerate lowest energy states for the antiferromagnetic nearest neighbor interaction on the triangular lattice. Red lines highlight the interaction between parallel spins which is energetically unfavorable. From \cite{Balents2010}.}
  \label{fig:Frustration1}
\end{centering}

On the triangular lattice, it is not possible to put all three spins into their lowest-energy configuration. For a given plaquette, the system achieves the lowest energy by having two spins parallel. There are three such states\footnote{There are actually six, but global spin flip symmetry makes three of these redundant.} which are energetically degenerate, and in the thermodynamic limit the degeneracy of a plaquette generates an infinite number of energetically degenerate spin configurations. We describe such a system as being ''frustrated" because there is no state where each interaction achieves its minimal energy. This frustration prevents antiferromagnetic Ising spins on a triangular lattice from developing long-range magnetic order at any temperature \cite{Wannier1950}. Instead, as the system is cooled, thermal fluctuations drive the system between disordered, degenerate minimal energy states until the system is too cold to support spin flip excitations; the system freezes in a disordered state.

The absence of long-range magnetic order in the classical triangular Ising antiferromagnet suggests that such a system is close to developing the properties of a QSL. If we consider the quantum triangular spin-\textonehalf{} Ising antiferromagnet, we have all the ingredients necessary for a spin liquid. The ground state of such a system is an equally-weighted quantum superposition of degenerate minimal energy states. This ground state is necessarily disordered at 0 K, because it is a superposition of disordered states, and correlated, because it cannot be factored into a product of states. Note that the QSL also satisfies the requirements for topological order, and we would therefore expect it to possess fractional excitations. To understand the fractional excitations in this system, we will discuss the Resonating Valence Bond (RVB) description of a QSL.


\subsubsection{The Resonating Valence Bond Description of a Quantum Spin Liquid}

We have seen that the ground state of quantum triangular Ising antiferromagnet satisfies the definition of a QSL, and is composed of a superposition of equally-weighted degenerate states satisfying the nearest-neighbor antiferromagnetic interaction. But what are these degenerate states? What are their properties?

We can picture each degenerate state as a random configuration of spin-0 singlet pairs on the triangular lattice. These pairs, called valence bonds, are maximally entangled and therefore quantum mechanical. At $T = 0$K, the valence bonds undergo quantum fluctuations that allow the system to explore all possible valence bond pairings. We can think of the valence bonds ``resonating'' between different configurations, in analogy to the resonating bonds between carbon atoms in a benzene ring. Figure \ref{fig:RVB1} provides a visualization.

\begin{centering}
\includegraphics[width=0.7\textwidth]{C:/Users/dsbjr/Documents/GitHub/Dissertation/img/ResonatingValenceBond-Balents.png}
  \captionsetup{width=0.75\textwidth}
  \captionof{figure}[Resonating Valence Bond State on a Triangular Lattice]{Resonating valence bond state on a triangular lattice. The RVB state is the superposition of random valence bond pairings. From \cite{Balents2010}.}
  \label{fig:RVB1}
\end{centering}

From the RVB description of this spin liquid, we can understand how fractional excitations emerge. Consider flipping a single spin to break a valence bond into two unpaired spins, each of which is an excitation above the ground state. On flipping a spin, the total spin of the system changes by one, and each charge neutral excitation carries a fractional spin-\textonehalf{}. Because the system resonates between random singlet pairings, the excitations are free to diffuse across the lattice without energetic penalty and are therefore unconfined. These charge-0 spin-\textonehalf{} excitations are called spinons and confirm the QSL as a system with topological order.

\subsection{Why are Spin Liquids Interesting?} \label{whyqslinteresting}

Spin liquids are rich, novel phases of matter possessing topological order and exotic fractional excitations. Accordingly, they are inherently interesting from the perspective of fundamental physics. However, there are other compelling reasons to study spin liquids that are relevant to future technology: quantum computation and high-temperature superconductivity. 

Majorana fermions are massless, self-conjugate (i.e., their own antiparticle) fermions that occur in certain topologically ordered materials. These quasiparticles can robustly store and perform logic operations on quantum information \cite{Lian2017}, potentially meeting the practical quantum computing requirements of scalability and gates that are faster than decoherence time. Majorana fermions are predicted to be quasiparticle excitations of some spin liquids. Therefore spin liquids may provide a realization of technology necessary for quantum computation.

Additionally, spin liquids may be related to an exotic form of superconductivity. In 1986, Anderson proposed the RVB state (a spin liquid) as the explanation for high-temperature superconductivity in lanthanum copper oxide \cite{Anderson1986}. In this explanation for high $T_{c}$ supercondctivity, the singlet pairs in the RVB state hosts, upon hole doping, give rise to coherent, superconducting Cooper pairs  \cite{Chen2018}. A potential explanation for superconductivity beyond BCS is an excellent reason to study spin liquids.

\subsection{Physical realizations of spin liquids}

To this point, we have discussed QSLs as abstract entities - states of matter that exist in geometrically frustrated quantum magnets at $T = 0$K. But are there physical realizations of such an abstract construction? Published experimental results confirm there are at least two.

Herbertsmithite (ZnCu\textsubscript{3}(OD)\textsubscript{6}Cl\textsubscript{2}) is a spin-\textonehalf{} kagom\'{e}{} lattice\footnote{The kagom\'{e} lattice, like the triangular lattice, is highly frustrated for antiferromagnetic interactions} antiferromagnet. Neutron scattering measurements of single crystals of herbertsmithite reveal a continuum of spin excitations, rather than the conventional spin waves expected in an ordered ferromagnet \cite{Han2012}. This continuum is consistent with that expected for the spinon excitations of a QSL discussed previously.

$\alpha$-ruthenium (III) chloride is a two-dimensional quantum magnet with anisotropic Ising interactions between localized effective spin-\textonehalf{} magnetic moments arranged in a hexagonal lattice. This system, which we will discuss in depth in the next chapter, provides a physical realization of an exactly solvable spin liquid model.

%\chapter{Properties of \texorpdfstring{$\alpha$-Ruthenium Trichloride}{alpha-RuCl3}}
This is a deep dive, cover Hamiltonian, previous experiments, structure, symmetries, phonons, maybe even solve Kitaev spin liquid

\section{Synthesis and Crystal Structure}

Discuss synthesis from beta-RuCl3. Discuss monoclinic to rhombohedral transition.

\section{Electronic Structure}

Discuss evidence that RuCl3 is a Mott insulator. Show plots of calculated electronic structure.

\section{Magnetic Properties}

Discuss magnetic transitions - paramagnet to antiferromagnet. Stacking faults causing additional magnetic transitions.

\section{\texorpdfstring{\rucl}{RuCl3} as a Kitaev Spin Liquid}

\subsection{The Kitaev Model}

Discuss and solve the Kitaev model.

\subsection{Spin Liquid Behavior in \texorpdfstring{\rucl}{RuCl3}}

Measurements that show \rucl is a proximate Kitaev spin liquid: neutron scattering, field-induced disordered states, half-quantized thermal hall conductance, magnetic impurity doping suppressing AF interaction.

%\chapter{Electrolyte Gating}
This chapter is about electrolyte gating

\section{Introduction to Electronic Structure}

A few introductory words

\subsection{Density of States}

Derive density of states in 1, 2, and 3 dimensions

\subsection{Occupation Function and Charge Density}

Introduce Fermi and Bose occupation functions. Show they reduce to Boltzmann occupation at high temperature.

Introduce Fermi Energy.

\subsection{Electrochemical Potential and Field-Effect Gating}

Extend Fermi energy to non-zero temperature and field.

\section{Gating using Electrolytes}

\subsection{Mechanism of Electrolyte Gating}

Describe how it works with a few graphics

\subsection{Properties and Use of Electrolytes}

What kinds of different electrolytes are available? Why use a particular one? What do you need to think about when doing a measurement?

\subsection{Notable Electrolyte Gating Measurements}

Ambipolar gating of WS2. MoS2 metal-insulator transition. Trevor Petach's measurements.

%\chapter{Electronic Transport}
This chapter describes original measurements of the resistivity of bulk and exfoliated single-crystal \rucl as a function of temperature and electrolyte gate voltage. Bulk measurements are found to correspond with published literature, while measurements of the exfoliated flakes suggest an anomalously high energy density of states at the Fermi level and show substantial hysteresis in resistance when sweeping the electrolyte gate voltage. I conclude by discussing some explanations for the unusual behavior of the exfoliated flakes.

\section{Introduction to Electronic Transport Measurements}

Electronic transport measurements characterize the movement of charge through a material by looking for changes in resistivity as a function of some tuned parameter, like temperature, carrier density, or magnetic field. Because the resistivity is a empirical consequence of both the electronic structure of and scattering processes in a material, changes in resistivity as a function of the tuned parameter provide valuable information about these aspects.

We can learn about the electronic structure of a material by measuring the thermal activation energy of conduction. Consider a material with its electrochemical potential in the band gap. Because there are no partially occupied bands, the material is an insulator. However, insulators do conduct electricity - in order to do so, a carrier must be thermally excited to the bottom of the conduction band\footnote{Put plainly, a phonon scatters from an electron, increasing its energy so that it now occupies a state in the conduction band.}. From Drude theory \cite{Ashcroft1976}, we know that $\sigma \propto n$, where $n$ is the density of thermally excited electrons. We can write the probability of exciting an electron as a Boltzmann factor:

\begin{equation}
\mathbb{P}[\text{exc.}] \propto e^{-\frac{\Delta}{kT}}
\end{equation}

where $k$ is the Boltzmann constant and $\Delta = E_{c} - \mu$. Using the proportionality,

\begin{equation}
\sigma = A e^{-\frac{\Delta}{kT}}
\end{equation}

which is the Arrhenius relation. The value of $\Delta$ extracted from a resistance-temperature curve is the separation in energy space between the chemical potential and the bottom of the conduction band. Changes in $\Delta$ as a function of carrier density contain information about the electronic density of states.


\section{\texorpdfstring{\rucl}{RuCl3} Bulk Crystals}

Initial measurements of Bulk \ruclnospace . Also Weary Heart.

\section{Exfoliated \texorpdfstring{\rucl}{RuCl3} Crystals}

Use data from Rookle02, Rookle03, Rookle09 (I think the contacts were good for these samples) and then RuCl3TransportSample006

%\chapter{Raman Spectroscopy}

Raman spectroscopy uses inelastic light scattering to measure certain vibrational modes in materials. For single molecules, these modes correspond to Raman-active vibrations that change the polarizability of the molecule. For crystals, these modes are the collective motion of the lattice - the phonon modes.

I have used Raman spectroscopy to investigate the phonon modes of electrolyte-gated \ruclnospace. Because phonon modes are sensitive to bond lengths and atomic interactions, defects or distortion in the lattice should be visible as shifting or broadening of peaks in the Raman spectrum. Therefore Raman spectroscopy can identify changes in the lattice that may explain the absence of an expected electronic phase transition in electrolyte-gated \ruclnospace.

\section{Fundamentals of Raman Spectroscopy}
When a material is illuminated by monochromatic light of frequency $\omega$, the incident light is either absorbed, scattered, reflected, or transmitted. While the spectrum of the transmitted and reflected light contains only light at frequency $\omega$, the spectrum of the scattered light includes both radiation at $\omega$ and additional pairs of frequencies $\omega \pm \omega_{i}$. The frequencies $\omega_{i}$ are typically associated with transitions between rotational, vibrational, and electronic states of the constituent molecules of the material \cite{Long2002}. The scattered light typically has a randomized phase and polarization relative to the incident light.

The highest intensity radiation in the spectrum of scattered light occurs at frequency $\omega$ and is called Rayleigh scattered\footnote{Rayleigh scattering is responsible for diffuse sky radiation, or in otherwords, why the sky is blue.}. Radiation at other frequencies is Raman scattered. Raman scattering with frequency $\omega_{i} < \omega$ is called Stokes Raman scattering; Raman scattering with frequency $\omega_{i} > \omega$ is called anti-Stokes Raman scattering. Necessarily, Rayleigh scattering is an elastic scattering process ($E = \hbar \omega = \hbar \omega_{i}$), while Raman scattering is inelastic ($E_{\text{initial}} = \hbar \omega \neq E_{\text{final}} = \hbar \omega_{i}$).

A single molecule Rayleigh scatters incident light by absorbing a photon, moving from the ground state to an unstable virtual state, then decaying back to the ground state by emitting a photon with energy equal to the incident photon. In contrast, a single molecule Raman scatters incident light by absorbing a photon and moving to unstable virtual state, then decaying to an intermediate state above the ground state by emitting a lower-energy photon \footnote{This is Stokes Raman scattering. If a photon is incident on a molecule in an already excited state, and then the molecule decays back to the ground state, the emitted photon has \textit{more} energy than the incident photon. This type of scattering is anti-Stokes scattering and its magnitude can be used to determine the temperature of a material.}. The intermediate state then decays to the ground state by non-photonic processes (i.e., cooling by collision with other molecules).

Scattering from single crystals follows a similar pattern: an incident photon excites a molecule into an unstable virutal state. The Rayleigh scattering case is identical to that for a single molecule. But for Raman scattering, the intermediate state decays to the ground state by emitting a phonon - a quantized lattice vibration that carries energy away from the excited molecule. By conservation of energy, the emitted photon necessarily has less energy than the incident photon, and that difference in energy\footnote{This difference in energy is referred to as the Raman shift, and is typically expressed in wavenumber $\nu$ having units of inverse length (typically cm$^{-1}$).}. must be equal to the energy of the phonon. Accordingly, the spectrum of Raman-scattered light is rich with information about lattice vibrations. Figure \ref{fig:FundRamanSpect1} provides a visual aid for understanding these scattering processes.

\begin{centering}
\includegraphics[width=0.9\textwidth]{C:/Users/dsbjr/Documents/GitHub/Dissertation/img/RamanRayleighScatter-nanophoton.png}
  \captionsetup{width=0.75\textwidth}
  \captionof{figure}[Elastic and inelastic light scattering processes]{An energy level diagram of Rayleigh, Stokes Raman, and anti-Stokes Raman scattering. Image by Nanophoton Corporation, retrieved from \url{https://www.nanophoton.net/raman/raman-spectroscopy.html}}
  \label{fig:FundRamanSpect1}
\end{centering}

While Raman spectroscopy is an invaluable tool, it does have limitations. Only certain vibrational modes can participate in Raman scattering. To understand which modes are amenable to investigation by Raman spectroscopy, and the underpinnings of Rayleigh and Raman scattering, we appeal to a classical analysis of the theory of light scattering from molecules. The following discussion is an abridged version of that found in Long \cite{Long2002}.

A molecule exposed to radiation having an incident frequency $\omega_{1}$ will radiate with intensity $I$, given by

\begin{equation}
I = k'_{\omega} \omega_{s}^{4} p_{0}^{2} \sin^{2} \theta
\end{equation}

with

\begin{equation}
k'_{\omega} = \frac{1}{32 \pi^{2} \epsilon_{0} c^{3}}
\end{equation}

where $p_{0}$ is the magnitude of the electric dipole induced at frequency $\omega_{s}$ which is generally but not necessarily different from $\omega_{1}$. The coresponding wavenumber equations are

\begin{equation}
I = k'_{\nu} \nu_{s}^{4} p_{0}^{2} \sin^{2} \theta
\end{equation}

\begin{equation}
k'_{\nu} = \frac{\pi^{2} c}{2 \epsilon_{0}}
\end{equation}

We can write the induced electric dipole as a multipole expansion:

\begin{equation}
\mathbf{p} = \mathbf{p}^{(1)} + \mathbf{p}^{(2)} + \mathbf{p}^{(3)} + ...
\end{equation}

where

\begin{equation}
\begin{aligned}
	\mathbf{p}^{(1)} &= \boldsymbol{\alpha} \cdot \mathbf{E} \\
	\mathbf{p}^{(2)} &= \frac{1}{2} \boldsymbol{\beta} \cdot \mathbf{E} \mathbf{E} \\
	\mathbf{p}^{(3)} &= \frac{1}{6} \boldsymbol{\gamma} \cdot \mathbf{E} \mathbf{E} \mathbf{E}
\end{aligned}
\end{equation}
	
where $\boldsymbol{\alpha}$, $\boldsymbol{\beta}$, and $\boldsymbol{\gamma}$ are the polarizability, hyperpolarizability, and second order hyperpolarizability tensors of second, third, and fourth rank, respectively. An analysis including $\boldsymbol{\beta}$ and $\boldsymbol{\gamma}$ leads to hyper Raman scattering related to higher harmonics of the incident radiation that are beyond the scope of this dissertation. Accordingly, we retain only the dipole term in the multipole expansion.

A priori, there is no reason to assume the polarizability of the molecule will be constant as the constituent atoms vibrate when the molecule is excited. Accordingly, we can expand the polarizability tensor $\boldsymbol{\alpha}$ in a Taylor series of displacements from equilibrium:

\begin{equation}
\alpha_{\rho \sigma} = (\alpha_{\rho \sigma})_{0} + \sum_{k} \left( \frac{\partial \alpha_{\rho \sigma}}{\partial Q_{k}} \right)_{0} Q_{k} + \frac{1}{2} \sum_{k,l} \left( \frac{\partial^{2} \alpha_{\rho \sigma}}{\partial Q_{k} \partial Q_{l}} \right)_{0} Q_{k} Q_{l} ...
\end{equation}

where $\alpha_{\rho \sigma}$ are the components of the polarizability tensor (with $\rho$ and $\sigma$ spanning values x, y, and z), $Q_{i}$ are the normal coordinates of vibrations associated with molecular vibrational frequencies $\omega{i}$, and a subscript 0 indicates a value at equilibrium. Taking the electrical harmonicity approximation\footnote{Electrical harmonicity means that the variation of the polarizability in a vibration is proportional to the first power of $Q$, in analogy to mechanical harmonicity where a restoring force is proportional to displacement from equilibrium.}, we retain only the first power of $Q$ and can write the previous equation in the following way:

\begin{equation}
\begin{aligned}
	(\alpha_{\rho \sigma})_{k} &= (\alpha_{\rho \sigma})_{0} + (\alpha'_{\rho \sigma})_{k} Q_{k} \\
	(\alpha'_{\rho \sigma})_{k} &= \left( \frac{\partial \alpha_{\rho \sigma}}{\partial Q_{k}} \right)_{0}
\end{aligned}
\end{equation}

The components $(\alpha'_{\rho \sigma})_{k}$ are a well-formed tensor $\boldsymbol{\alpha}'_{k}$, so we can write the above equation as:

\begin{equation}
\boldsymbol{\alpha}_{k} = \boldsymbol{\alpha}_{0} + \boldsymbol{\alpha}'_{k} Q_{k}
\end{equation}

We know the time dependence of $Q_{k}$, so we can write

\begin{equation}
\boldsymbol{\alpha}_{k} = \boldsymbol{\alpha}_{0} + \boldsymbol{\alpha}'_{k}Q_{k_{0}} \cos (\omega_{k} t + \delta_{k})
\end{equation}

We also know the time dependence of the incident radiation

\begin{equation}
\mathbf{E} = \mathbf{E}_{0} \cos (\omega_{1} t)
\end{equation}

We can now write the time dependence of the first term in the multipole expansion of the induced dipole moment

\begin{equation}
\mathbf{p}^{(1)} = \boldsymbol{\alpha}_{0} \mathbf{E}_{0} \cos (\omega_{1} t) + \boldsymbol{\alpha}'_{k} \mathbf{E}_{0} Q_{k_{0}} \cos (\omega_{k} t + \delta) \cos (\omega_{1} t)
\end{equation}

Using a trigonometric identity, we can write the induced dipole moment in the following suggestive manner

\begin{equation}
\mathbf{p}^{(1)} = \mathbf{A} \cos (\omega_{1} t) + \mathbf{B} \cos ((\omega_{1} - \omega_{k}) t - \delta) + \mathbf{B} \cos ((\omega_{1} + \omega_{k}) t + \delta)
\end{equation}

with

\begin{equation}
\begin{aligned}
	\mathbf{A} &= \boldsymbol{\alpha}_{0} \mathbf{E}_{0} \\
	\mathbf{B} &= \boldsymbol{\alpha}'_{k} \mathbf{E}_{0} Q_{k_{0}}
\end{aligned}
\end{equation}

Naturally, $\mathbf{A}$ is the dipole moment associated with Rayleigh scattering, and $\mathbf{B}$ is the dipole moment associated with Raman scattering. This analysis shows us two things. First, there is a response at the incident frequency and a pair of shifted frequencies; Rayleigh and Raman scattering naturally emerge from classical light scattering. Second, Rayleigh scattering happens for all molecules, but Raman scattering is only possible when a vibration causes changes in the polarizability, i.e.,

\begin{equation}
\left( \frac{\partial \alpha_{\rho \sigma}}{\partial Q_{k}} \right)_{0} \neq 0
\end{equation}

Only some vibrations meet this criterion, and therefore only these Raman-active vibrations can Raman scatter incident light. While a full discussion of Raman selection rules is beyond the scope of this dissertation, molecules of low symmetry are more likely to have Raman-active modes, and molecules of high symmetry are less likely to have Raman-active modes.

\section{Raman Spectrum of RuCl$\textsubscript{3}$}

The Raman spectrum of bulk \rucl has been extensively reported  \cite{Sandilands2015,Sandilands2016,Glamazda2017,Mashhadi2018,Zhou2018}. My Raman measurements of \rucl are consistent with published literature and show peaks at 114, 161, 220, 267, 294, 310, and 338 cm$^{-1}$ as seen in Figure \ref{fig:RamanSpectRuCl3-1}.

\begin{centering}
\includegraphics[width=0.5\textwidth]{C:/Users/dsbjr/Documents/GitHub/Dissertation/img/BulkRuCl3RamanSpectrum-Original.jpg}
  \captionsetup{width=0.75\textwidth}
  \captionof{figure}[Raman spectrum of bulk \ruclnospace]{Raman spectrum of bulk \ruclnospace, measured at Stanford on the SNSF HORIBA Scientific LabRAM HR Evolution spectrometer.}
  \label{fig:RamanSpectRuCl3-1}
\end{centering}

These peaks correspond to vibrational modes of the lattice, which we can analyze by considering the space group\footnote{A space group is the symmetry group of a configuration in space. Each space group identifies the symmetries of a particular lattice. There are 230 possible space groups in three dimensions.} of \ruclnospace. The set of symmetries in the space group constrains the possible vibrational modes and specifies which will be Raman active. Formally, \rucl has space group C2/m. However, because interactions along the c-axis are weak, the Raman spectrum more closely matches the symmetries of the two-dimensional D$_{\text{3d}}$ space group \cite{Sandilands2015}.

Raman-active vibrations for the D\textsubscript{3d} space group present in \rucl are categorized as either A\textsubscript{g} or E\textsubscript{g}. A indicates that atoms oscillate in-phase and are singly degenerate; B indicates that atoms oscillate in a way that is doubly degenerate. The subscript g indicates that these oscillations are inversion symmetric. The D\textsubscript{3d} allows only 2 A\textsubscript{g} modes and 4 E \textsubscript{g} modes, which can be distinguished by polarization of the incident and Raman-scattered light. The modes in \rucl are classified as either A\textsubscript{g} or E\textsubscript{g} using polarization and are  highlighted in Figure \ref{fig:RamanSpectRuCl3-2}.

\begin{centering}
\includegraphics[width=0.5\textwidth]{C:/Users/dsbjr/Documents/GitHub/Dissertation/img/BulkRuCl3RamanSpectrumLabeled-Original.png}
  \captionsetup{width=0.75\textwidth}
  \captionof{figure}[Raman spectrum of bulk \ruclnospace with labeled symmetries]{Raman spectrum of bulk \ruclnospace, with Raman modes labeled as A\textsubscript{g} or E\textsubscript{g}.}
  \label{fig:RamanSpectRuCl3-2}
\end{centering}

The measured spectrum of \rucl has five modes with polarizations consistent with E\textsubscript{g} - one more than allowed by the symmetry of the D\textsubscript{3d} space group. The mode at 220 cm\textsuperscript{-1} is least likely to be a D\textsubscript{3d} mode and may instead be related to interlayer coupling because it has one of the lowest intensities \cite{Sandilands2015}.




\section{Raman Spectrum of Electrolytes}
Results of tests to select DEME-BF4

\section{Raman Spectroscopy of RuCl$\textsubscript{3}$ with in-situ Electrolyte Gating}
Intercalated and unintercalated RuCl
Conclusion that in plane lattice is unperturbed

\chapter{X-Ray Diffraction}

This chapter covers x-ray diffraction measurements of pristine and electrolyte gated \ruclnospace . I provide an introduction to the technique and cover previous x-ray diffraction measurements of \ruclnospace , then discuss the changes in x-ray diffraction patterns caused by electrolyte gating. I find that the separation between monolayers of \rucl does not change as a function of electrolyte gate voltage, ruling out ion intercalation and limiting electrochemistry to the outer layer of an exfoliated flake.

\section{Introduction}

X-ray diffraction uses coherent scattering of x-rays from a crystal\footnote{A crystal is a material whose atoms occupy points generated by a set of discrete translations operations given by a closed set of lattice vectors. More simply, a crystal is a solid made of a regular array of atoms.} to determine its structure. Because x-rays have wavelengths that approximately match the separation between atoms in solids, a crystal can serve as a diffraction grating for x-ray radiation. The resulting diffraction pattern can be used to calculate the positions of atoms in the crystal.

Consider a two-dimensional square lattice of points having a separation $d$ of order \AA . X-rays incident on this lattice with wavevector $\mathbf{k_{i}}$ will scatter with wavevector $\mathbf{k_{f}}$. Considering only elastic scattering (the vast majority of scattering events), $|\mathbf{k_{i}}| = |\mathbf{k_{f}}|$. The intensity of the scattered radiation will be highest when the difference in path length for each wave is an integer multiple of the wavelength, or when the scattered waves all have the same phase. The intensity is maximum when the incident angle ($\theta$), wavelength ($\lambda$), and separation $d$ satisfy the Bragg condition:

\begin{centering}
\includegraphics[width=0.5\textwidth]{./img/BraggCondition-Wiki.png}
  \captionsetup{width=0.75\textwidth}
  \captionof{figure}[X-ray scattering from a lattice]{X-ray scattering from a two-dimensional lattice. The Bragg condition is satisfied when twice the bolded distance ($d \sin \theta$) is an integer multiple of the wavelength.} 
  \label{fig:XrayIntro-1}
\end{centering}

\begin{equation}
2d \sin \theta = n \lambda
\end{equation}

The equation above can be used to calculate $d$ given incident x-rays of a known wavelength and the scattering angle $\theta$. The diffraction pattern for the lattice in Figure \ref{fig:XrayIntro-1} will show diffraction peaks not just at $\theta = \sin^{-1} \left( \frac{n \lambda}{2d} \right)$, but also at $\theta = \sin^{-1} \left( \frac{n \lambda}{2d\sqrt{2}} \right)$ and others. These ``extra'' peaks appear because the $d$ in the Bragg condition does not necessarily correspond to the distance between atoms, but rather to the distance between scattering planes.

\begin{centering}
\includegraphics[width=0.5\textwidth]{./img/BraggDiffraction-Cullity.png}
  \captionsetup{width=0.75\textwidth}
  \captionof{figure}[2D scattering planes labeled with Miller indices]{Bragg diffraction planes in a 2D square lattice. Each set of planes corresponds to a different spacing and will result in a different family of peaks. Scattering planes are labeled by their Miller indices. From \cite{Cullity2014}.} 
  \label{fig:XrayIntro-2}
\end{centering}

Scattering planes are labeled by Miller indices, which can be defined in various ways. Given a set of lattice vectors $\mathbf{a_{1}},\mathbf{a_{2}},\mathbf{a_{3}}$, define reciprocal lattice vectors $\mathbf{b_{i}}$ such that $\mathbf{a_{i}} \cdot \mathbf{b_{j}} = \delta_{ij}$. Then the plane corresponding to Miller indices $(hkl)$ is the plane perpendicular to the vector

\begin{equation}
\mathbf{g_{hkl}} = h \mathbf{b_{1}} + k \mathbf{b_{2}} + l \mathbf{b_{3}}
\end{equation}

and the separation between planes $d_{hkl} = \frac{1}{|\mathbf{g_{hkl}}|}$. Alternatively, we can think of Miller indices as the inverse of fractional intercepts in the unit cell, as in Figure \ref{fig:XrayIntro-3}.

\begin{centering}
\includegraphics[width=0.5\textwidth]{./img/MillerIndices-Cullity.png}
  \captionsetup{width=0.75\textwidth}
  \captionof{figure}[Unit cell with scattering planes labeled with Miller indices]{Unit cell with scattering planes labeled by Miller indices. Each plane intersects the x, y, and z axes at 1/$n$th of the unit cell dimension, where $n$ is the Miller index for the appropriate axis. From \cite{Cullity2014}.} 
  \label{fig:XrayIntro-3}
\end{centering}

To understand this interpretation of scattering planes, consider the (200) plane is Figure \ref{fig:XrayIntro-3}. Because the index $h = 2$, this plane intersects the $\mathbf{a}$ axis at $\frac{1}{2}$ of its extent along the $\mathbf{a}$ direction. $k,l = 0$, so there are no intercepts along the $\mathbf{b}$ and $\mathbf{c}$ axes. The same logic applies for the remaining scattering planes, with the caveat that a bar over an index (like in $(\bar{1} 1 0)$) means that the index is negative. The scattering intensity from planes with low Miller indices is typically higher than from those planes with high Miller indices. This difference results from low index planes having more atoms and therefore a higher electron density with which to scatter x-rays.

To this point, we have used an intuitive and physical approach to understand how x-rays and crystals interact. But underlying our this discussion is a mathematically rigorous theory of x-ray scattering. If we define a scattering wavevector

\begin{equation}
\mathbf{q} = \mathbf{k_{f} - k_{i}}
\end{equation}

then we can write the amplitude of the scattered x-rays as the sum of all the scattered waves weighted by their phase

\begin{equation}
F(\mathbf{q}) = \int \rho_{e}(\mathbf{r}) e^{i \mathbf{q} \cdot \mathbf{r}} d\mathbf{r}
\end{equation}

where $\rho_{e}(\mathbf{r})$ is the electron density. $F(\mathbf{q})$ is exactly the electron density in reciprocal space. 

We can also arrive at the scattered amplitude by considering the scattering from each atom in the unit cell. X-ray scattering from individual atoms $f(\mathbf{q})$, called atomic form factors, are tabulated \cite{Henke1993}. The scattering from an individual unit cell can be calculated by summing the form factor and weighting it by its position.

\begin{equation}
F(\mathbf{q}) = \sum_{\text{unit cell}} f(\mathbf{q}) e^{i \mathbf{q} \cdot \mathbf{r}}
\end{equation}

The scattering intensity is given by

\begin{equation}
I(\mathbf{q}) = |F(\mathbf{q})|^{2}
\end{equation}

which is measured in a scattering experiment. From the measured intensity, we can make inferences about $\rho_{e}(\mathbf{r})$ that determine the crystal structure.

\begin{centering}
\includegraphics[width=0.75\textwidth]{./img/FourRingDiffractometer.png}
  \captionsetup{width=0.75\textwidth}
  \captionof{figure}[Diagram of x-ray scattering angles]{Four-ring diffractometer with scattering angles labeled. From \cite{Clark2007}.} 
  \label{fig:XrayIntro-4}
\end{centering}

In the laboratory, the scattering vector $\mathbf{q}$ is given not by  $x,y,z$ coordinates, but instead by four angles: $\theta$, $\phi$, $\Omega$, and $\chi$. The definitions of these angles are shown in Figure \ref{fig:XrayIntro-4}. Loosely speaking, the angles $\theta$ and $\phi$ specify the orientation of the crystal, and the angles $\Omega$ and $\chi$ describe the direction of the beam.

\section{Previous measurements of \rucl}

As discussed in Chapter 2, x-ray diffraction studies of bulk single-crystal \rucl show it to be a two-dimensional material (interlayer spacing 5.7 \AA) of space group C2/m having monoclinic symmetry \cite{Johnson2015}. However, electron diffraction measurements of \rucl exfoliated by lithium intercalation are consistent with a P3\textsubscript{1}12 space group, which has trigonal symmetry. These different symmetries may be due to stacking faults introduced by the exfoliation process that cause the stacking order to change from \textit{ABC} to \textit{AB} \cite{Gronke2018}.

X-ray measurements also show that \rucl immersed in a solution of ions and subjected to an electric field will rapidly and reversibly incorporate cations into its structure. \rucl favors intercalation of cations because each \rucl monolayer is bounded by electronegative chlorine planes. X-ray diffraction measurements show that the interlayer spacing of intercalated \rucl is approximately equal to the ionic radius of the intercalated cation \cite{Schollhorn1983,Steffen1986}. The change in interlayer spacing for \rucl intercalated by selected cations is summarized in Figure \ref{fig:XrayIntro-5}.

\begin{centering}
\includegraphics[width=0.75\textwidth]{./img/RuCl3IntercalatedSpacing-original.png}
  \captionsetup{width=0.75\textwidth}
  \captionof{figure}[Intercalated \rucl interlayer spacing]{Interlayer spacing of \rucl intercalated by selected ionic species. The interlayer spacing for intercalated \rucl is almost double that of pristine \ruclnospace .} 
  \label{fig:XrayIntro-5}
\end{centering}

These results suggest that x-ray diffraction can distinguish between intercalated and unintercalated \rucl by determining the interlayer spacing.

\section{X-ray measurements of \rucl at Stanford} 

\subsubsection{Methods}
X-ray scattering measurements were made at Stanford in the Stanford Nano Shared Facilities X-ray Diffraction Laboratory on the Bruker D8 Venture single crystal x-ray diffractometer using Cu-generated 1.54 \AA{} x-rays. Single crystal samples of \rucl approximately 100 $\mu$m thick were affixed to a glass substrate with thin pieces of scotch tape before measurement. Pristine \rucl exfoliated onto a 300 nm Si/SiO\textsubscript{2} substrate were measured similarly.

Electrolyte gating samples were fabricated using techniques similar to those described in the methods sections of chapters 4 and 5. The samples were electrically contacted using silver epoxy and 8 mil insulated copper wire.

For electrolyte gating measurements, samples were mounted using tape and a goniometer stage. I selected angles $\theta$, $\phi$, $\Omega$, and $\chi$ that maximized the intensity of the interlayer scattering peak before applying ionic liquid. After finding the optimal angles and setting the diffractometer appropriately, I removed the sample and applied the ionic liquid. I subsequently reinstalled and electrically connected the sample. Potential bias was provided by a Keithley 2400 source measurement unit.

\subsection{Pristine \rucl}

Measurements of bulk single-crystal \rucl are consistent with literature, as shown in Figure \ref{fig:XrayMeas-1}. However, I find the crystal structure consistent with the trigonal symmetry found in powder diffraction measurements  \cite{Fletcher1967} and in exfoliated flakes \cite{Gronke2018}, rather than monoclinic C2/m symmetry. As suggested in the literature, stacking faults may be the cause of differing symmetries. The presence of these stacking faults is unsurprising - the bulk sample I measured is a cutting from a larger bulk crystal which was roughly handled.

\begin{centering}
\includegraphics[width=0.75\textwidth]{./img/BulkRuCl3XrayDiffraction.png}
  \captionsetup{width=0.75\textwidth}
  \captionof{figure}[Bulk \rucl diffraction pattern]{Bulk \rucl diffraction pattern as measured at Stanford. Peaks are labeled with their Miller indices.} 
  \label{fig:XrayMeas-1}
\end{centering}

In exfoliated \ruclnospace , the (003) interlayer peak remains visible, but the intensity of the remaining peaks is too small to resolve (see Figure \ref{fig:XrayMeas-2}). However, the (003) peak remains at the same value of $2\theta$ independent of exfoliations, showing that the interlayer separation remains a constant 5.7 \AA .

\begin{centering}
\includegraphics[width=0.75\textwidth]{./img/ExfoliatedRuCl3Xray-original.png}
  \captionsetup{width=0.75\textwidth}
  \captionof{figure}[Exfoliated \rucl diffraction pattern]{Exfoliated \rucl diffraction pattern as measured at Stanford. At this particular value of $\phi$, substrate peaks are not visible. However, they are present at other values of $\phi$.} 
  \label{fig:XrayMeas-2}
\end{centering}

Despite being too small to resolve on a log scale, there does appear to be at least one other x-ray peak present for exfoliated \rucl at particular values of $\phi$ with 120\degree{} symmetry. This feature is less than a factor of two above background and does not map exactly to a known scattering plane in \rucl (though being less than 2\degree{} from the (104) plane). This feature is highlighted in Figure \ref{fig:XrayMeas-3}.

\begin{centering}
\includegraphics[width=0.5\textwidth]{./img/ExfoliatedRuCl3Diffractogram-original.png}
  \captionsetup{width=0.75\textwidth}
  \captionof{figure}[Exfoliated \rucl diffractogram]{Exfoliated \rucl diffractogram (contrast enhanced) as measured at Stanford. Notable features include the \rucl interlayer separation at $\chi = 0$ and $\theta = 15.7$\degree, the ring-like features associated with diffraction from evaporated Au contacts, and the bright feature at large $\theta$ and $\chi < 0$ from the Si substrate. The feature highlighted in green I attribute to exfoliated \ruclnospace .} 
  \label{fig:XrayMeas-3}
\end{centering}

\subsection{Electrolyte-gated \rucl}

The diffraction pattern of exfoliated \rucl does not seem to change as a function of electrolyte bias. Figure \ref{fig:XrayMeas-4} shows the diffraction pattern at different electrolyte voltages - note that the interlayer separation feature remains at the same value of $2\theta$, indicating the interlayer separation remains constant.

\begin{centering}
\includegraphics[width=0.75\textwidth]{./img/ExfoliatedRuCl3InterlayerSeparationGateVoltage-original.png}
  \captionsetup{width=0.75\textwidth}
  \captionof{figure}[Electrolyte-biased exfoliated \rucl diffraction pattern]{Electrolyte-biased and exfoliated \rucl diffraction pattern at different electrolyte voltages, offset for clarity. Note that the feature corresponding to \rucl interlayer separation maintains the same value of $2\theta$ and has the same width despite the changing electrolyte bias.} 
  \label{fig:XrayMeas-4}
\end{centering}

The feature near the value of $2\theta$ for the (104) scattering plane is further attenuated after the application of the electrolyte and becomes hard to resolve. With my current data, it is hard to tell if this peak is hard to resolve because its corresponding scattering plane becomes disordered, or if it's just too faint at the available x-ray intensity.

\section{Discussion}

X-ray diffraction measurements of exfoliated \rucl show that the interlayer separation does not change as a function of electrolyte bias. Accordingly, given the large size of the ions comprising the electrolyte, we can be confident that the ions do not migrate into and out of the lattice. Therefore, the hysteretic behavior observed in electronic transport and Raman spectroscopy cannot be explained by the intercalation of DEME+.

Additionally, because the interlayer separation does not change, the chemical interaction between the electrolyte and the exfoliated \rucl is limited to the top and the sides of the material. The measured flakes of \rucl have thicknesses from between 50 nm to 100 nm, corresponding to 100 to 200 monolayers of \ruclnospace . Accordingly, any electrochemical intercation can account only for 1\% of the observed signal - a negligible amount. Therefore, these x-ray measurements both rule out intercalation by DEME+ and show that electrochemistry is negligible.

Possible references:
%Kim, H.-S., & Kee, H.-Y. (2015). Crystal structure and magnetism in alpha-RuCl3: an ab-initio study, 155143, 1–10. https://doi.org/10.1103/PhysRevB.93.155143
Discusses different crystal structures

%Johnson, R. D., Williams, S. C., Haghighirad, A. A., Singleton, J., Zapf, V., Manuel, P., … Coldea, R. (2015). Monoclinic crystal structure of α-RuCl3 and the zigzag antiferromagnetic ground state. Physical Review B - Condensed Matter and Materials Physics, 92(23). https://doi.org/10.1103/PhysRevB.92.235119
Talks about stacking faults

\chapter{Conclusion}

This dissertation investigated \rucl doped by electrolyte biasing and found that, contrary to predictions, \rucl remained highly insulating. Further, Raman spectra for electrolyte-biased \rucl identified a hysteretic transition between two visually distinct states driven by electrolyte bias voltage. Finally, x-ray diffraction measurements showed these distinct states do not result from the cations in the electrolyte physically interacting with the \ruclnospace . In this concluding section, I review the results of these measurements and identify possible frameworks that could explain the data. I then suggest measurements that could further clarify the nature of electrolyte-biased \ruclnospace .

\section{Hypotheses}

\subsection{Unlikely Explanations}
\begin{itemize}
\item \textbf{Electrochemistry:} In an electrolyte gating experiment, the first confounding effect that must be addressed is electrochemistry. However, there are several observations that rule out electrochemistry as the primary explanation of the behavior of electrolyte biased \ruclnospace . Measurements are made only within the electrochemical stability window of the electrolyte. Gate current remains low during measurements and, after an increase in gate voltage, there is a transient gate current that decays to an equilibrium gate current. The transition between the two states with distinct Raman spectra is repeatable, ruling out an irreversible chemical interaction. Finally, x-ray diffraction measurements show that the electrolyte does not penetrate the \rucl lattice, indicating that only the top and sides of the \rucl are available for chemical interaction. However, the presence of substrate features in the Raman spectrum indicates that the excitation characterizes the entire flake, which is dominated by the chemically-unaffected bulk.

\item \textbf{Intercalation:} Only cations intercalate in \ruclnospace , given the planes of chlorine atoms present at each interface. However, it seems unreasonable to expect a cation as large as DEME+ to fit between the layers. Additionally, x-ray diffraction measurements show the interlayer separation remains a constant 5.7 \AA{} and does not change as a function of gate voltage. The layers do not move and there is no room for DEME+ to fit. Intercalation and subsequent exfoliation of a single top layer is unlikely (there is no reason to suspect only the top layer intercalates when previous measurements show uniform intercalation) and cannot explain the behavior of the bulk, which dominates the Raman signal. However, intercalation of only protons, which may not be observable by x-ray diffraction, cannot be ruled out.

\item \textbf{Stress caused by electrostatics:} \rucl in these measurements is a thin, charged flake immersed in a biased electrolyte. Accordingly, the electric fields may subject the flake to stress that causes changes in its Raman spectrum. Further, this stress could be consistent with the mottled appearance of the flakes. However, if the stress is electrostatic, then it should appear at both positive and negative electrolyte bias voltage. Instead, the response of the material is asymmetric with respect to gate voltage.

\item \textbf{Charge localization:} The electrolyte bias adds charge to \ruclnospace , but the material could remain insulating because the charge is localized by lattice deformations and cannot participate in transport. However, charge localization should be continuously dependent on the amount of charge added, and there are only sharp changes in the Raman spectrum as a function of electrolyte bias. Additionally, lattice deformations associated with charge defects, like those in color centers in alkali halides, show the presence of new, broad peaks associated with the lattice deformations. Such features are not present in either Raman spectrum for electrolyte-biased \ruclnospace .
\end{itemize}

\subsection{The case for a phase transition}

The above discussion shows that many of the experimental artifacts or less interesting explanations for the behavior of electrolyte-biased \rucl can be ruled out. What remains is the possibility of structural or electronic phase transition. The transition between states having different Raman spectra is hysteretic and has an interface, showing the transition is first order. Further, there are several different \rucl structures that are nearly degenerate in energy that gating could drive transitions between. Additionally, there is evidence that charge doping in \rucl creates charge order. If the charge-ordering couples to the lattice, then an electronic transition could explain the behavior. Future work should explore these possibilities.

\section{Future Work}

Future investigation of electrolyte-biased \rucl should focus on x-ray diffraction measurements. Brighter x-rays from a synchrotron light source would be able to resolve scattering planes with a component parallel to the c axis, and therefore better characterize the structure of \rucl in either state. Further, diffuse x-ray scattering, which measures correlations in electron density across unit cells, could characterize the role of stress in the transition.

If additional x-ray measurements do not show a change in the structure of \ruclnospace , then the possibility of an electronic transition should be considered. Charge-ordered materials have different transport properties depending on the orientation of the stripes, so if the transport properties of electrolyte-biased \rucl are anisotropic, then the transition may be structural instead of electronic.

Possible references:
%Kim, H.-S., & Kee, H.-Y. (2015). Crystal structure and magnetism in alpha-RuCl3: an ab-initio study, 155143, 1–10. https://doi.org/10.1103/PhysRevB.93.155143
Discusses different crystal structures

\bibliography{C:/Users/dsbjr/Documents/GitHub/Dissertation/tex/dissertation.firstdraft}

\end{document}
